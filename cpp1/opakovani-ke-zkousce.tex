\begin{frame}[fragile]
\begin{exampleblock}{Reference}
\begin{lstlisting}
int& vratReferenci() {
	static int i = 0;
	return i;
}

// ...
int& reference = vratReferenci();
reference++;     // zmeni "static int i"
int neniReference = vratReferenci();
neniReference++; // nezmeni "static int i"
\end{lstlisting}
\end{exampleblock}
\end{frame}


\begin{frame}[fragile]
\begin{exampleblock}{Reference}
\begin{lstlisting}
int& vratReferenci(int& ref) {
	return ref;
}

// ...
int i = 0;
int& reference = vratReferenci(i);
reference++;     // zmeni "int i"
int neniReference = vratReferenci(i);
neniReference++; // nezmeni "int i"
\end{lstlisting}
\end{exampleblock}
\end{frame}


\begin{frame}[fragile]
\begin{exampleblock}{Reference}
\begin{lstlisting}
int vratReferenci(int& ref) {
	return ref;
}

// ...
int i = 0;
int& reference = vratReferenci(i);
reference++;     // nezmeni "int i"
int neniReference = vratReferenci(i);
neniReference++; // nezmeni "int i"
\end{lstlisting}
\end{exampleblock}
\end{frame}




\begin{frame}[fragile]
\begin{block}{Dynamická alokace paměti}
\begin{itemize}
\item Nové operátory new, new[], delete, delete[]
\item Staré funkce malloc, calloc, realloc, free
 \begin{itemize}
 \item Neinicializují objekty pomocí konstruktoru/nevolají destruktor $\rightarrow$ potenciální zdroj problémů
 \end{itemize}
\end{itemize}
\end{block}
\begin{exampleblock}{}
\begin{lstlisting}
// vytvoreni 1 objektu Object
Object* object = new Object;
delete object;

// vytvoreni pole 100 objektu Object
object = new Object[100];
delete[] object;
\end{lstlisting}
\end{exampleblock}
\end{frame}


\begin{frame}[fragile]
\begin{exampleblock}{Dynamická alokace paměti}
\begin{lstlisting}
// vytvoreni pole 100 ukazatelu na objekt Object
Object** object = new Object*[100];
for (int i = 0; i < 100; i++)
	object[i] = new Object;


for (int i = 0; i < 100; i++)
	delete object[i];
delete[] object;
\end{lstlisting}
\end{exampleblock}
\end{frame}

\begin{frame}[fragile]
\begin{block}{Přetypování}
\begin{itemize}
\item Nové operátory: static\_cast, dynamic\_cast, const\_cast, reinterpret\_cast 
\item Starý operátor: (typ)
\end{itemize}
\end{block}
\begin{exampleblock}{}
\begin{lstlisting}
int a = static_cast<int>(12.123);

ParentClass* b = static_cast<ParentClass*>(new SubClass);
SubClass* c = dynamic_cast<SubClass*>(b);

int d = reinterpret_cast<int>(c);
\end{lstlisting}
\end{exampleblock}
\end{frame}

\section{Objektové typy}

\begin{frame}
\begin{block}{}
\begin{center}
\Huge
Objektové typy
\end{center}
\end{block}
\end{frame}

\begin{frame}[fragile]
\begin{block}{Objektové typy}
\begin{itemize}
\item struct
\item class
\end{itemize}
\end{block}
\begin{exampleblock}{}
\begin{lstlisting}
struct Struct {
	int _i; // _i je public
};

class Clazz {
	int _i; // _i je private
};

\end{lstlisting}
\end{exampleblock}
\end{frame}


\begin{frame}[fragile]
\begin{exampleblock}{Objektové typy -- viditelnost} 
\begin{lstlisting}
class Clazz {
public:
	int _public;
	
protected:
	int _protected;
	
private:
	int _private;
	
public:
	int _publicAgain;
};

\end{lstlisting}
\end{exampleblock}
\end{frame}



\begin{frame}[fragile]
\begin{exampleblock}{Objektové typy -- statické atributy/metody} 
\begin{lstlisting}
class Clazz {
	static int _staticAttribute;
	
public:	
	static int getStaticAttribute();
};

// inicializace statického atributu
int Clazz::_staticAttribute = 123;

// definice metody
int Clazz::getStaticAttribute() {
	return _staticAttribute;
}

\end{lstlisting}
\end{exampleblock}
\end{frame}


\begin{frame}[fragile]
\begin{exampleblock}{Objektové typy -- ukazatel this} 
\begin{lstlisting}
class Clazz {
	
	void method() {
		// this je typu Clazz*
		_i = 123;
		this->_i = 123;
		(*this)._i = 123;
	}
	
	int _i;
}

\end{lstlisting}
\end{exampleblock}
\end{frame}

\newcommand{\errorcolor}{\color[rgb]{1.0,0.0,0.0}}

\begin{frame}[fragile]
\begin{exampleblock}{Objektové typy -- konstantní metody/objekty} 
\begin{lstlisting}
class Clazz {
	void method() {
		// this je typu Clazz*
		_i = 123;
	}
	void methodConst() const {
		// this je typu const Clazz*
		// _i = 123; <@\errorcolor $\errorcolor\leftarrow$ NELZE - atribut je konstatni@>
	}
	int _i;
}
// ...
Clazz clazz;
clazz.method();
clazz.methodConst();
\end{lstlisting}
\end{exampleblock}
\end{frame}


\begin{frame}[fragile]
\begin{exampleblock}{Objektové typy -- konstantní metody/objekty} 
\begin{lstlisting}
class Clazz {
	void method() {
		// this je typu Clazz*
		_i = 123;
	}
	void methodConst() const {
		// this je typu const Clazz*
		// _i = 123; $\commentcolor\leftarrow$ NELZE - atribut je konstatni
	}
	int _i;
}
// ...
const Clazz clazz;
// clazz.method(); $\commentcolor\leftarrow$ NELZE
clazz.methodConst();
\end{lstlisting}
\end{exampleblock}
\end{frame}

\begin{frame}[fragile]
\begin{exampleblock}{Objektové typy -- konstantní metody/objekty} 
\begin{lstlisting}
class Clazz {
	void method() {
		_i = 123;
	}
	void method() const {
		cout << _i << endl;
	}
	
	int _i;
}
// ...
Clazz clazz;
clazz.method(); // nastavi _i na 123
const Clazz& clazzConst = clazz;
clazzConst.method(); // vypise _i na cout\end{lstlisting}
\end{exampleblock}
\end{frame}

\begin{frame}[fragile]
\begin{block}{Objektové typy -- život objektu} 
\begin{itemize}
\item konstrukce
\begin{itemize}
\item konstruktor praprapředka
\item konstruktor prapředka
\item konstruktor předka
\item konstruktor třídy
\end{itemize}
\item objekt žije!
\item destrukce
\begin{itemize}
\item destruktor třídy
\item destruktor předka
\item destruktor prapředka
\item destruktor praprapředka
\end{itemize}
\end{itemize}

\end{block}
\end{frame}


\begin{frame}[fragile]
\begin{block}{Objektové typy -- konstruktor} 
Kompilátor v základu vyrobí:
\begin{itemize}
\item bezparametrický konstruktor,
\item kopírovací konstruktor,
\item operátor=.
\end{itemize}

Kopírovací konstruktor a operátor = vytvářejí mělkou (bitovou) kopii.

\end{block}
\end{frame}


\begin{frame}[fragile]
\begin{exampleblock}{Objektové typy -- konstruktor} 
\begin{lstlisting}
class Clazz {
	Clazz(); // konstruktor
	Clazz(int i); // parametricky konstruktor
	Clazz(int i, int j); // parametricky konstruktor
	Clazz(const Clazz& clazz); // kopirovaci konstruktor
	Clazz(Clazz&& clazz); // move konstruktor (C++11)
	
	Clazz& operator=(const Clazz& clazz); // operator=
	Clazz& operator=(Clazz&& clazz); // move assignment op. (C++11)
};

\end{lstlisting}
\end{exampleblock}
\end{frame}


\begin{frame}[fragile]
\begin{exampleblock}{Objektové typy -- konstruktor} 
\begin{lstlisting}
class Clazz {
	Clazz(); // konstruktor
};

Clazz clazz;
Clazz clazz2 = Clazz(); // nevhodne - konstr., kop. konstr., destruktor
Clazz* clazz3 = new Clazz;
Clazz* clazz4 = malloc(sizeof(Clazz)); // Nevola konstruktor
new(clazz4) Clazz;
\end{lstlisting}
\end{exampleblock}
\end{frame}


\begin{frame}[fragile]
\begin{exampleblock}{Objektové typy -- konstruktor} 
\begin{lstlisting}
class Clazz {
	Clazz(int i); // parametricky konstruktor
};

Clazz clazz(123);
Clazz clazz2 = Clazz(123); // nevhodne - konstr., kop. konstr., destruktor
Clazz* clazz3 = new Clazz(123);
\end{lstlisting}
\end{exampleblock}
\end{frame}



\begin{frame}[fragile]
\begin{exampleblock}{Objektové typy -- konstruktor} 
\begin{lstlisting}
class Clazz {
	Clazz(int i, int j); // parametricky konstruktor
};

Clazz clazz(123, 456);
Clazz* clazz2 = new Clazz(123, 456);
\end{lstlisting}
\end{exampleblock}
\end{frame}


\begin{frame}[fragile]
\begin{exampleblock}{Objektové typy -- konstruktor} 
\begin{lstlisting}
class Clazz {
	Clazz(const Clazz& clazz); // kopirovaci konstruktor
};

Clazz clazz; // Volani jineho konstruktoru
Clazz clazz2(clazz);
Clazz clazz3 = clazz;
Clazz clazz4 = Clazz(clazz); // nevhodne - kop. konstr., kop. konstr., destruktor

void funkce(Clazz clazz) { cout << clazz }
funkce(clazz); // !!! Tady taky !!!

void funkce2(Clazz& clazz) { cout << clazz }
funkce2(clazz); // Tady se nekopiruje! Je predan puvodni objekt
\end{lstlisting}
\end{exampleblock}
\end{frame}


\begin{frame}[fragile]
\begin{exampleblock}{Objektové typy -- konstruktor} 
\begin{lstlisting}
class Clazz {
	Clazz& operator=(const Clazz& clazz); // operator=
};

Clazz clazz; // Volani jineho konstruktoru
Clazz clazz2; // Volani jineho konstruktoru
Clazz* clazz3 = new Clazz; // Volani jineho konstruktoru

clazz2 = clazz;
*clazz3 = clazz;
\end{lstlisting}
\end{exampleblock}
\end{frame}



\begin{frame}[fragile]
\begin{exampleblock}{Objektové typy -- inicializační část konstruktoru} 
\begin{lstlisting}
class Clazz {
	Clazz() : _i(123), _j(456) { }
	Clazz(int i) : _i(i), _j(456) { }
	Clazz(int i, int j) : _i(i), _j(j) { }

	int _i;
	int _j;
};

\end{lstlisting}
\end{exampleblock}
\end{frame}

\begin{frame}[fragile]
\begin{exampleblock}{Objektové typy -- delegování konstruktoru} 
\begin{lstlisting}
class Clazz {
	Clazz() : Clazz(123) { }; 
	
	Clazz(int i) { ... } 
};

\end{lstlisting}
\end{exampleblock}
\end{frame}


\begin{frame}[fragile]
\begin{block}{Objektové typy -- destruktor} 
\begin{itemize}
\item Uvolňuje paměťové prostředky.
\item
Pokud třída obsahuje virtuální metody, pak by měl být destruktor virtuální.
\end{itemize}
\end{block}
\begin{exampleblock}{}
\begin{lstlisting}
class Clazz {
	~Clazz();
};

{ 
	Clazz clazz;
} // destruktor clazz

Clazz* clazz2 = new Clazz; 
delete clazz2; // destruktor clazz2

\end{lstlisting}
\end{exampleblock}
\end{frame}


\begin{frame}[fragile]
\begin{block}{Objektové typy -- dědičnost} 
\begin{itemize}
\item Vícenásobná dědičnost -- možnost dědit z libovolného množství předků.
\item Struct i class jsou ekvivalentní objektové typy.
\item Problém diamantu -- virtuální dědění.
\item Specifikace viditelnosti -- ovlivní viditelnost u zděděných složek (atributů a metod).
\end{itemize}
\end{block}
\begin{exampleblock}{}
\begin{lstlisting}
class Clazz { };
class SubClazz : public Clazz { };
class SubSubClazz : protected SubClazz { };
class SubSubSubClazz : private SubSubClazz { };

class OhCrap : Clazz, SubClazz, SubSubClazz, SubSubSubClazz { };
\end{lstlisting}
\end{exampleblock}
\end{frame}

\begin{frame}[fragile]
\begin{exampleblock}{Objektové typy -- dědičnost} 
\begin{lstlisting}
class Clazz { };
class SubClazz : public Clazz { };
\end{lstlisting}
\end{exampleblock}
\begin{block}{}

Co vidím uvnitř SubClazz?
\begin{itemize}
\item private, protected i public složky ze SubClazz
\item protected a public složky z Clazz
\end{itemize}

Co vidím vně SubClazz?
\begin{itemize}
\item public složky SubClazz
\item public složky Clazz
\end{itemize}
\end{block}
\end{frame}

\begin{frame}[fragile]
\begin{exampleblock}{Objektové typy -- dědičnost} 
\begin{lstlisting}
class Clazz { };
class SubClazz : protected Clazz { };
\end{lstlisting}
\end{exampleblock}
\begin{block}{}
Co vidím uvnitř SubClazz?
\begin{itemize}
\item private, protected i public složky ze SubClazz
\item protected a public složky z Clazz
\end{itemize}

Co vidím vně SubClazz?
\begin{itemize}
\item public složky SubClazz
\end{itemize}
\end{block}
\end{frame}


\begin{frame}[fragile]
\begin{exampleblock}{Objektové typy -- dědičnost -- konstruktor předka} 
\begin{lstlisting}
class Clazz { 
	Clazz();
	Clazz(int i);
};

class SubClazz : public Clazz { 
	SubClazz(); // vola Clazz::Clazz()
	SubClazz(int i); // vola Clazz::Clazz()
};
\end{lstlisting}
\end{exampleblock}
\end{frame}


\begin{frame}[fragile]
\begin{exampleblock}{Objektové typy -- dědičnost -- konstruktor předka} 
\begin{lstlisting}
class Clazz { 
	//Clazz(); // definovanim parametrickeho konstruktoru kompilator tento sam nevytvori
	Clazz(int i);
};

class SubClazz : public Clazz { 
	SubClazz(); // error, parametr volat neumim!
	SubClazz(int i); // error, parametr volat neumim!
};
\end{lstlisting}
\end{exampleblock}
\end{frame}


\begin{frame}[fragile]
\begin{exampleblock}{Objektové typy -- dědičnost -- konstruktor předka} 
\begin{lstlisting}
class Clazz { 
	//Clazz(); // definovanim parametrickeho konstruktoru kompilator tento sam nevytvori
	Clazz(int i);
};

class SubClazz : public Clazz { 
	SubClazz() : Clazz(123) { }
	SubClazz(int i) : Clazz(i) { }
};
\end{lstlisting}
\end{exampleblock}
\end{frame}



\begin{frame}[fragile]
\begin{exampleblock}{Objektové typy -- dědičnost -- konstruktor předka} 
\begin{lstlisting}
class Clazz { 
	//Clazz(); // definovanim parametrickeho konstruktoru kompilator tento sam nevytvori
	Clazz(int i);
};

class SubClazz : public Clazz { 
	SubClazz() : Clazz(123), _i(456), _j(789) { }
	SubClazz(int i) : Clazz(i), _i(i), _j(789) { }
	
	int _i;
	int _j;
};
\end{lstlisting}
\end{exampleblock}
\end{frame}


\begin{frame}[fragile]
\begin{exampleblock}{Objektové typy -- dědičnost -- virtuální dědičnost} 
\begin{lstlisting}
class Clazz {
	int _a;
};

class SubClazz1 : public virtual Clazz { 
};

class SubClazz2 : public virtual Clazz {
};

class DiamondIsNotAProblemClazz : public SubClazz1, public SubClazz2 { 
};
\end{lstlisting}
\end{exampleblock}
\end{frame}



\begin{frame}[fragile]
\begin{block}{Objektové typy -- polymorfizmus -- volání metod}
Metody:
\begin{itemize}
\item časná vazba (výchozí),
\item pozdní vazba.
\end{itemize} 
\end{block}
\end{frame}


\begin{frame}[fragile]
\begin{exampleblock}{Objektové typy -- polymorfizmus -- časná vazba} 
\begin{lstlisting}
class Clazz {
	void method() { cout << "Clazz"; }
};

class SubClazz : public Clazz { 
	void method() { cout << "SubClazz"; }
};

Clazz clazz;
clazz.method(); // "Clazz"
SubClazz subclazz;
subclazz.method(); // "SubClazz"

Clazz& clazzRef = subclazz;
clazzRef.method(); // "Clazz" !!!
Clazz* clazzPtr = &subclazz;
clazzPtr->method(); // "Clazz" !!!
\end{lstlisting}
\end{exampleblock}
\end{frame}

\begin{frame}[fragile]
\begin{exampleblock}{Objektové typy -- polymorfizmus -- pozdní vazba} 
\begin{lstlisting}
class Clazz {
	virtual void method() { cout << "Clazz"; }
};

class SubClazz : public Clazz { 
	virtual void method() override { cout << "SubClazz"; }
};

Clazz clazz;
clazz.method(); // "Clazz"
SubClazz subclazz;
subclazz.method(); // "SubClazz"

Clazz& clazzRef = subclazz;
clazzRef.method(); // "SubClazz"
Clazz* clazzPtr = &subclazz;
clazzPtr->method(); // "SubClazz"
\end{lstlisting}
\end{exampleblock}
\end{frame}


\begin{frame}[fragile]
\begin{exampleblock}{Objektové typy -- polymorfizmus -- pozdní vazba} 
\begin{lstlisting}
class Clazz {
	virtual void method() { cout << "Clazz"; }
};

class SubClazz : public Clazz { 
	virtual void method() override { cout << "SubClazz"; }
};

Clazz clazz;
clazz.method(); // "Clazz"
SubClazz subclazz;
subclazz.method(); // "SubClazz"

clazz = subclazz;
clazz.method(); // "Clazz" !!! Dochazi k oriznuti objektu
\end{lstlisting}
\end{exampleblock}
\end{frame}



\begin{frame}[fragile]
\begin{exampleblock}{Objektové typy -- polymorfizmus -- čistě virtuální metody} 
\begin{lstlisting}
class Clazz {
	virtual void method() = 0;
};

class SubClazz : public Clazz { 
	virtual void method() override { cout << "SubClazz"; }
};

//Clazz clazz; // error - nelze vytvorit objekt od abstraktni tridy
//clazz.method(); // error
SubClazz subclazz;
subclazz.method(); // "SubClazz"
\end{lstlisting}
\end{exampleblock}
\end{frame}

\section{Přetěžování operátorů}

\begin{frame}
\begin{block}{}
\begin{center}
\Huge
Přetěžování operátorů
\end{center}
\end{block}
\end{frame}

\begin{frame}[fragile]
\begin{block}{Přetěžování operátorů} 
\begin{itemize}
\item Definování jak se bude daný operátor chovat pro náš objektový typ.
\item 4 skupiny operátorů:
\begin{itemize}
\item nejdou přetížit,
\item jdou metodou,
\item jdou metodou i funkcí,
\item jdou funkcí/statickou metodou.
\end{itemize}
\end{itemize}
\end{block}
\end{frame}



\begin{frame}[fragile]
\begin{block}{Přetěžování operátorů} 
Nejdou:
\begin{itemize}
\item tečkové operátory -- ., .*, ::
\item ternární operátor -- ? :
\item přetypovací operátory -- static\_cast, \ldots
\item další -- typeof, sizeof
\end{itemize}
\end{block}
\end{frame}

\begin{frame}[fragile]
\begin{block}{Přetěžování operátorů} 
Metodou:
\begin{itemize}
\item =
\item ->
\item ->*
\item $[$\,$]$
\item ()
\item op. přetypování (typ)
\end{itemize}
\end{block}
\end{frame}

\begin{frame}[fragile]
\begin{block}{Přetěžování operátorů} 
Metodou i funkcí: \\
+ - * / ++ -\,- > < >= <= == >\,> <\,< \& | $\tilde{}$ += -= *= /= $\ldots$
\end{block}
\end{frame}

\begin{frame}[fragile]
\begin{block}{Přetěžování operátorů} 
Funkcí/statickou metodou:
\begin{itemize}
\item new
\item new$[$\,$]$
\item delete
\item delete$[$\,$]$
\end{itemize}
\end{block}
\end{frame}

\begin{frame}[t,fragile]
\begin{block}{Přetěžování operátorů} 
Operátory se přetěžují jako:
\begin{lstlisting}
navratovyTyp operator@(parametry)
\end{lstlisting}

Uvažujte výraz:
\begin{lstlisting}
int a = 5, b = 10;
int c = a + b;
\end{lstlisting}
\end{block}
\begin{block}{}
Platí:
\begin{itemize}
\item operátor + je binární operátor
\item operandy jsou \emph{a}, \emph{b}
\item změní se proměnná \emph{a}? NE -> konstanta
\item změní se proměnná \emph{b}? NE -> konstanta
\end{itemize}
\end{block}
\end{frame}

\begin{frame}[fragile]
\begin{block}{Přetěžování operátorů} 
Operátory se přetěžují jako:
\begin{lstlisting}
navratovyTyp operator@(parametry)
\end{lstlisting}

Uvažujte výraz:
\begin{lstlisting}
int a = 5, b = 10;
int c = a + b;
\end{lstlisting}
\end{block}
\begin{exampleblock}{}

\begin{lstlisting}
int operator+(int a, int b) {
	return int(a->state + b->state);
}

struct int {
	int operator+(int b) {
		return int(this->state + b->state);
	}
};
\end{lstlisting}
\end{exampleblock}
\end{frame}



\begin{frame}[fragile]
\begin{block}{Přetěžování operátorů} 
Operátory se přetěžují jako:
\begin{lstlisting}
navratovyTyp operator@(parametry)
\end{lstlisting}

Uvažujte výraz:
\begin{lstlisting}
int a = 5, b = 10;
int c = a + b;
\end{lstlisting}
\end{block}
\begin{exampleblock}{}

\begin{lstlisting}
int operator+(const int& a, const int& b) {
	return int(a->state + b->state);
}

struct int {
	int operator+(const int& b) const {
		return int(this->state + b->state);
	}
};
\end{lstlisting}
\end{exampleblock}
\end{frame}



\begin{frame}[fragile]
\begin{exampleblock}{Přetěžování operátorů} 
\begin{lstlisting}
Komplex a = 5, b = 10;
Komplex c = a + b;

Komplex operator+(const Komplex& a, const Komplex& b) {
	return Komplex(a->re + b->re, a->im + b->im);
}

struct Komplex {
	Komplex operator+(const Komplex& b) const {
		return Komplex(this->re + b->re, this->im + b->im);
	}
};
\end{lstlisting}
\end{exampleblock}
\end{frame}




\begin{frame}[fragile]
\begin{exampleblock}{Přetěžování operátorů} 
\begin{lstlisting}
struct Komplex {
	friend Komplex operator+(const Komplex& a, const Komplex& b) {
		return Komplex(a->re + b->re, a->im + b->im);
	}
};
\end{lstlisting}
\end{exampleblock}
\begin{block}{}
Friend + definice v těle způsobí:
\begin{itemize}
\item nejedná se o metodu, ale o funkci!
\item má přístup k privátním a chráněným složkám
\end{itemize}
\end{block}
\end{frame}


\begin{frame}[fragile]
\begin{block}{Přetěžování operátorů} 
Mění některé operátory stav objektu (operandu)?
\begin{itemize}
\item ++, -\/-
\item =, +=, -=, *=, /=, $\ldots$
\item $[$\,$]$ -- pokud vrací referenci/ukazatel na stav objektu
\item () -- pokud mění stav objektu
\end{itemize}
\end{block}
\end{frame}

\begin{frame}[fragile]
\begin{block}{Přetěžování operátorů} 
Proč vracet statický objekt? Možnosti:
\begin{itemize}
\item statický objekt -- dochází ke kopírování, vytvoření dočasného objektu a jeho následné destrukci (pokud nedojde k optimalizaci od kompilátoru)
\item reference
\begin{itemize}
\item ale kde ji vezmu?
\item obyčejná proměnná zanikne s koncem bloku -- nelze
\item statická (nebo globální) proměnná bude existovat dál, ale nebude to fungovat ve vícevláknových aplikacích -- nelze
\item dynamicky alokovaná proměnná -- ztratím ukazatel a informaci o alokaci a do konce programu bude paměť zabraná -- nelze
\end{itemize}
\item ukazatel (dyn. alokovaná paměť) -- změna sémantiky operátoru -- nelze
\end{itemize}
Lépe to udělat prostě nejde.
\end{block}
\end{frame}

\section{Šablony}

\begin{frame}
\begin{block}{} 
\begin{center}
\Huge
Šablony
\end{center}
\end{block}
\end{frame}



\begin{frame}[fragile]
\begin{block}{Šablony} 
Šablony představují způsob generického (obecného) programování. Šablonou může být objektový typ, metoda, funkce.
\\~\\
Princip funkce (v případě objektového typu):
\begin{itemize}
\item šablona třídy (to co já napíšu -- Sablona<T, U>)
\item instance šablony třídy -- kompilátor narazí na konkrétní použití šablony (zná jaké parametry bude dosazovat) a vytvoří konkrétní datový typ (Sablona<int, string>)
\item objekt od instance šablony třídy -- konkrétní objekt -- Sablona<int, string> objekt;
\end{itemize}
\end{block}
\end{frame}



\begin{frame}[fragile]
\begin{exampleblock}{Šablony} 
\begin{lstlisting}
// zakladni syntax:
template<definiceParametruSablony>
definiceTypuNeboMetodyNeboFunkce;

// template/class oznacuji obecny typ
// jde o ekvivalentni klicova slova, class je starsi zpusob zapisu
template<typename T, class U, typename V>
struct TemplateStruct {
	T _t;
	U _u;
	V _v;
};

TemplateStruct<int, int, double> objekt;
objekt._t = 123;
objekt._u = 456;
objekt._v = 78.9;
\end{lstlisting}
\end{exampleblock}
\end{frame}

\begin{frame}[fragile]
\begin{exampleblock}{Šablony} 
\begin{lstlisting}
// parametry mohou byt typove/hodnotove
template<typename T, int Hodnota, double X>
struct TemplateStruct {
	
	void method() { cout << X; }
	
	T _t = Hodnota;
};
\end{lstlisting}
\end{exampleblock}
\end{frame}


\begin{frame}[fragile]
\begin{exampleblock}{Šablony} 
\begin{lstlisting}
// parametry mohou mit vychozi hodnoty
template<typename T = int, int Hodnota = 15, double X = 2.34>
struct TemplateStruct {
	
	void method() { cout << X; }
	
	T _t = Hodnota;
};
\end{lstlisting}
\end{exampleblock}
\end{frame}



\begin{frame}[fragile]
\begin{exampleblock}{Šablony} 
\begin{lstlisting}
template<typename T>
struct TemplateStruct {
	void method();
	
	T _t;
};

template<typename T>
void TemplateStruct<T>::method() {
	_t = 123;
}
\end{lstlisting}
\end{exampleblock}
\end{frame}


\begin{frame}[fragile]
\begin{exampleblock}{Šablony -- nested templates} 
\begin{lstlisting}
template<typename T>
struct TemplateStruct {

	template<typename U>
	void method(U u);
	
	T _t;
};

template<typename T>
template<typename U>
void TemplateStruct<T>::method(U u) {
	cout << u;
	_t = 123;
}
\end{lstlisting}
\end{exampleblock}
\end{frame}



\begin{frame}[fragile]
\begin{exampleblock}{Šablony -- dependent names} 
\begin{lstlisting}
struct Concrete { struct Nested { }; };

template<typename T>
struct TemplateStruct {
	typename T::Nested _usageOfDependentTypeRequiresTypename;
};
\end{lstlisting}
\end{exampleblock}
\end{frame}



\begin{frame}[fragile]
\begin{exampleblock}{Šablony -- dependent names} 
\begin{lstlisting}
template<typename T>
struct S {
    template<typename U> void foo(){}
};
 
template<typename T>
void bar()
{
    S<T> s;
    s.foo<T>(); // error: < parsed as less than operator
    s.template foo<T>(); // OK
}
\end{lstlisting}
\end{exampleblock}
\end{frame}


\begin{frame}[fragile]
\begin{exampleblock}{Šablony -- explicit specialization (function)} 
\begin{lstlisting}
template<typename T>
void function(T param) { /* ... */ }

template<>
void function(int param) { /* ... */ }

template<>
void function(double param) { /* ... */ }
\end{lstlisting}
\end{exampleblock}
\end{frame}

\begin{frame}[fragile]
\begin{exampleblock}{Šablony -- explicit specialization (class)} 
\begin{lstlisting}
template<typename T>
class Clazz { 
	T method() { /* ... */ }
	
	T _attribute;
};

template<>
class Clazz<int> {
	int method() { /* ... */ }
	
	int _attribute;
	int _anotherAttribute;
};

\end{lstlisting}
\end{exampleblock}
\end{frame}



\begin{frame}[fragile]
\begin{exampleblock}{Šablony -- partial specialization (class)} 
\begin{lstlisting}
template<typename T, typename U, int V>
class Clazz { };

template<typename T, int V>
class Clazz<T, T, V> { };

template<typename T, typename U, int V>
class Clazz<T*, U, V> { };

template<typename T>
class Clazz<T, int, 10> { };

\end{lstlisting}
\end{exampleblock}
\end{frame}


\section{Poprava tupým mečem za neznalost\ldots}


\begin{frame}
	\begin{block}{} 
		\begin{center}
			\Huge
			Poprava tupým mečem za neznalost\ldots
		\end{center}
	\end{block}
\end{frame}

\begin{frame}[fragile]
\begin{alertblock}{Dynamická alokace paměti -- C funkce} 
\begin{itemize}
	\item malloc, calloc, realloc -- nevolají konstruktor
	\item free -- nevolá destruktor
	\item používejte new / delete
\end{itemize}
\end{alertblock}

\begin{exampleblock}{Dynamická alokace paměti -- C++ operátory} 
	\begin{itemize}
		\item new -- delete, new[] -- delete[]
\begin{itemize}		\item vždy musí tvořit pár!\end{itemize}
		\item \textbf{není} korektní volat new -- delete[] nebo new[] -- delete
	\end{itemize}
\end{exampleblock}
\end{frame}


\begin{frame}[fragile]
\begin{alertblock}{Nikdy nevoláme ručně destruktor!}
\begin{lstlisting}
{
	Clazz* clazz = new Clazz;
	clazz->~Clazz(); // NE!
	delete clazz; // vola znovu destruktor nad jiz rozbitou instanci
}

Clazz& operator=(const Clazz& ref) {
	if (this != &ref) {
		this->~Clazz();  // znici take tabulku VMT!
		...
	}
	
	return *this;
}
\end{lstlisting}
\end{alertblock}
\end{frame}

\begin{frame}[fragile]
\begin{exampleblock}{Operator= hlídá přiřazení "obj = obj"}
\begin{lstlisting}
Clazz& operator=(const Clazz& ref) {
	if (this != &ref) {
		...
	}
	
	return *this;
}
\end{lstlisting}
\end{exampleblock}
\end{frame}


\begin{frame}[fragile]
\begin{exampleblock}{Přetížené operátory korektně mění/nemění objekt}
\begin{lstlisting}
class Clazz {
	bool operator==(const Clazz& ref) const;
	bool operator!=(const Clazz& ref) const;	
	bool operator>=(const Clazz& ref) const;
	bool operator<=(const Clazz& ref) const;
	
	Clazz operator+(const Clazz& ref) const;
	Clazz operator-(const Clazz& ref) const;	

	Clazz& operator=(const Clazz& ref);	
	Clazz& operator+=(const Clazz& ref);
	Clazz& operator-=(const Clazz& ref);	
\end{lstlisting}
\ldots

\end{exampleblock}
\end{frame}

\begin{frame}[fragile]
\begin{exampleblock}{Přetížené operátory korektně mění/nemění objekt}

\ldots
\begin{lstlisting}
	Clazz operator[ ](int i) const;	
	Clazz& operator[ ](int i);	
	
	template<typename T>
	T operator()(T t) const;
		
	template<typename T>	
	T operator()(T t);		
};
\end{lstlisting}

\end{exampleblock}
\end{frame}
