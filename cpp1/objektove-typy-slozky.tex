\kapitola{Datové složky a operace}

\begin{frame}[fragile]
\frametitle{Viditelnost / přístupová práva}

\begin{bitemize}
\item 3 úrovně:
\begin{itemize}
\item \lstinline|public| -- přístupné všude
\item \lstinline|protected| -- přístupné z potomků
\item \lstinline|private| -- přístupné pouze z vnitřku třídy
\end{itemize}
\item výchozí hodnota:
\begin{itemize}
\item \lstinline|struct| -- \lstinline|public|
\item \lstinline|class| -- \lstinline|private|
\end{itemize}

\item změna viditelnosti:
\begin{itemize}
\item uvozuje se blok (\lstinline|viditelnost: ...|)
\item bloky se mohou opakovat, být zpřeházené
\end{itemize}
\end{bitemize}

\begin{bonusblock}{friend}
\begin{itemize}
\item přátelé třídy (\lstinline|friend|) mají přístup i k \lstinline|protected| a \lstinline|private| složkám
\end{itemize}
\end{bonusblock}
\end{frame}


\begin{frame}[fragile]
\frametitle{Viditelnost / přístupová práva}
\begin{yesblock}
\begin{lstlisting}
struct Pes {
// implicitní public: (platí pro struct)
// implicitní private: (platí pro class)

  string jmeno; // je public

private:
  int id; // je private
  int vek; // je private
 
protected:
  int pocetChlupu; // je protected

public:
  int pocetNohou; // je public
};
\end{lstlisting}
\end{yesblock}
\end{frame}

\pkapitola{Atributy (datové složky)}

\begin{frame}[fragile]
\frametitle{Atributy}

\begin{noteblock}{}
\begin{lstlisting}
[static] datovýTyp názevAtributu [inicializátor];
\end{lstlisting}
\end{noteblock}

\begin{bitemize}
\item \lstinline|static| -- statický atribut (váže se na třídu, ne na objekt)
\item \lstinline|[inicializátor]|\cpp{11} -- umožňuje nastavit výchozí hodnotu (nelze u~statických atributů)
\end{bitemize}

\begin{yesblock}
\begin{lstlisting}
  int pocetChlupu = 12345;  
  static int pocetPsu;
\end{lstlisting}
\end{yesblock}


\begin{bonusblock}{}
\begin{itemize}
\item Dále existují modifikátory: \lstinline|mutable|, \lstinline|volatile|
\end{itemize}
\end{bonusblock}
\end{frame}



\begin{frame}[fragile]
\frametitle{Statické atributy}
\begin{bitemize}
\item statické atributy potřebují vyhradit paměť a inicializovat -- v CPP souboru
\end{bitemize}

\begin{exampleblock}{PES.H}
\begin{lstlisting}
struct Pes {
  static int pocetPsu;
};
\end{lstlisting}
\end{exampleblock}

\begin{exampleblock}{PES.CPP}
\begin{lstlisting}
int Pes::pocetPsu = 0;
\end{lstlisting}
\end{exampleblock}

\end{frame}


\begin{frame}[fragile]
\frametitle{Statické atributy (použití)}
\begin{exampleblock}{MAIN.CPP}
\begin{lstlisting}
  Pes pes = PsiBouda::vytvorPsa();
  Pes::pocetPsu++;

  cout << "Pocet psu: " << Pes::pocetPsu;
\end{lstlisting}
\end{exampleblock}
\end{frame}

\pkapitola{Metody}


\begin{frame}[fragile]
\frametitle{Metody}

\begin{noteblock}{}
\begin{lstlisting}
deklaraceMetody:
[static] [virtual] datovýTypNávratovéHodnoty názevMetody([parametry]) [const] [override] [final] [noexcept];
\end{lstlisting}
\end{noteblock}

\begin{bitemize}
\item \lstinline|static| -- statická metoda (váže se na třídu, ne na objekt)
\item \lstinline|virtual| -- virtuální metoda (polymorfizmus)
\item \lstinline|const| -- konstantní metoda (nemění stav objektu)
\item \lstinline|override|\cpp{11} -- označuje přetíženou metodu z předka (polymorfizmus)
\item \lstinline|final|\cpp{11} -- zakazuje další přepisování metody v potomcích
\item \lstinline|noexcept|\cpp{11} -- označuje metodu, která nevyvolává výjimky
\end{bitemize}
\end{frame}

\begin{frame}[fragile]
\frametitle{Metody\ldots}
\begin{bonusblock}{}
\begin{itemize}
\item modifikátor \lstinline|inline| -- optimalizace, nevolá funkci, vkládá kód přímo na místo volání; kompilátor dosadí automaticky, pokud je metoda definována uvnitř třídy
\item modifikátor \lstinline|volatile| -- nestálé metody
\item \lstinline|deklaraceMetody = delete;| -- C++11, smazání metody
\item \lstinline|deklaraceMetody = default;| -- C++11, vynucení implementace metody kompilátorem
\end{itemize}
\end{bonusblock}
\end{frame}


\begin{frame}[fragile]
\frametitle{Deklarace a definice metody uvnitř třídy}
\begin{yesblock}
\begin{lstlisting}
struct Pes {
  // deklarace - úplný funkční prototyp, bez těla
  void stekej();

  // definice - deklarace + tělo
  void kousej(Osoba& osoba) {
    osoba.zran(KOUSNUTI_DO_KOTNIKU);    
  }
};
\end{lstlisting}
\end{yesblock}
\end{frame}


\begin{frame}[fragile]
\frametitle{Deklarace a definice metody vně třídy}
\begin{bitemize}
\item definice metody vně třídy musí specifikovat, že jde o metodu dané třídy
\begin{itemize}
\item používá se operátor ::
\item uvádějí se pouze modifikátory \lstinline|const|, \lstinline|override|, \lstinline|noexcept|
\end{itemize}
\end{bitemize}
\begin{yesblock}
\begin{lstlisting}
struct Pes {
  // deklarace
  void kousej(Osoba& osoba);
};

// definice
void Pes::kousej(Osoba& osoba) {
  osoba.zran(KOUSNUTI_DO_KOTNIKU);   
}
\end{lstlisting}
\end{yesblock}
\end{frame}


\begin{frame}[fragile]
\frametitle{Metody a this}

\begin{bitemize}
\item uvnitř instančních metod je dostupný ukazatel \lstinline|this| (odpovídá \lstinline|Třída*|)
\begin{itemize}
\item není potřeba, implicitně je možné se odkazovat na složky třídy
\end{itemize}
\end{bitemize}

\begin{yesblock}
\begin{lstlisting}
struct Pes {
  void obnovZdravi() {
    // this je Pes*, následující výrazy jsou totožné
    _zdravi = 100;
    this->_zdravi = 100;
    (*this)._zdravi = 100;
  }

private:
  int _zdravi;
};
\end{lstlisting}
\end{yesblock}
\end{frame}


\pkapitola{Speciální metody}

\begin{frame}
\frametitle{Speciální metody}
\begin{block}{}
Existuje několik speciálních metod souvisejících s vytvářením a rušením objektů:
\begin{itemize}
\item konstruktory (bez parametrů, s parametry, kopírovací, konverzní, move)
\item destruktor
\item operátor =
\end{itemize}
Základní verzi těchto metod nám vytvoří automaticky kompilátor, pokud některou z metod nadefinujeme sami, kompilátor pak nemusí nic vytvářet (chybějící bezparametrický konstruktor).
\end{block}
\end{frame}


\begin{frame}[fragile]
\frametitle{Vznik a zánik objektu}
\begin{block}{}
\lstinline|struct Pes| dědí z (\lstinline|struct Zvire| dědí z (\lstinline|struct Objekt|))
\end{block}

\begin{block}{Proces vzniku a zániku objektu Pes}
\begin{itemize}
\item konstruktor \lstinline|Objekt|
\item konstruktor \lstinline|Zvire|
\item konstruktor \lstinline|Pes|
\item[]
\begin{itemize}
\item objekt žije, lze volat metody\ldots
\end{itemize}

\item destruktor \lstinline|Pes|
\item destruktor \lstinline|Zvire|
\item destruktor \lstinline|Objekt|
\end{itemize}
\end{block}
\end{frame}








\begin{frame}[fragile]
\frametitle{Konstruktor}
\begin{noteblock}{}
\begin{lstlisting}
názevTřídy([parametry]) [inicializačníČást];

inicializačníČást:
 : atribut1(hodnota1), ...
\end{lstlisting}
\end{noteblock}

\begin{bitemize}
\item nemá návratovou hodnotu (ani \lstinline|void|)!
\item inicializační část -- pro inicializaci datových složek
\begin{itemize}
\item je potřeba pro inicializaci referenčních atributů
\end{itemize}
\end{bitemize}

\begin{bonusblock}{}
\begin{itemize}
\item modifikátor \lstinline|explicit| -- zakazuje implicitní konverze
\end{itemize}
\end{bonusblock}
\end{frame}


\begin{frame}[fragile]
\frametitle{Bezparametrický konstruktor}
\begin{yesblock}
\begin{lstlisting}
struct Pes {

  Pes() {
    cout << "konstruktor Pes()" << endl;
  }

};
\end{lstlisting}
\end{yesblock}
\end{frame}


\begin{frame}[fragile]
\frametitle{Parametrický konstruktor}
\begin{yesblock}
\begin{lstlisting}
struct Pes {

  Pes(int zdravi) {
    cout << "konstruktor Pes(int)" << endl;
    _zdravi = zdravi;
  }

private:
  int _zdravi;
};
\end{lstlisting}
\end{yesblock}
\end{frame}




\begin{frame}[fragile]
\frametitle{Parametrický konstruktor s inicializační částí}
\begin{yesblock}
\begin{lstlisting}
struct Pes {

  Pes(int zdravi) : _zdravi(zdravi) {
    cout << "konstruktor Pes(int)" << endl;
  }

private:
  int _zdravi;
};
\end{lstlisting}
\end{yesblock}
\end{frame}



\begin{frame}[fragile]
\frametitle{Kopírovací konstruktor}
\begin{bitemize}
\item kompilátor ho umí vytvořit automaticky
\begin{itemize}
\item vytváří mělkou kopii (\lstinline|memcpy(kopie, original, sizeof(Typ))|)
\end{itemize}
\end{bitemize}
\begin{yesblock}
\begin{lstlisting}
struct Pes {

  Pes(const Pes& pes) {
    cout << "konstruktor Pes(const Pes&)" << endl;
    _zdravi = pes._zdravi;
  }

private:
  int _zdravi;
};
\end{lstlisting}
\end{yesblock}
\end{frame}





\begin{frame}[fragile]
\frametitle{Delegování konstruktorů\cpp{11}}
\begin{bonusblock}{}
\begin{itemize}
\item delegování umožňuje zavolat jiný konstruktor před vlastním vykonáním konstruktoru (od C++11)
\item inicializační část pak musí obsahovat pouze volání delegovaného konstruktoru!
\end{itemize}
\end{bonusblock}
\vskip -2ex
\begin{bonusblock}{}
\begin{lstlisting}[basicstyle=\scriptsize]
struct Pes {
  Pes(std::string jmeno) : _jmeno(pes,_jmeno) { 
    cout << "konstruktor Pes(std::string)" << endl;
  }

  Pes(std::string jmeno, int vek) : Pes(jmeno) {
    cout << "konstruktor Pes(std::string, int)" << endl;
    _vek = vek;
  }

private:
  std::string _jmeno;
  int _vek;
};
\end{lstlisting}
\end{bonusblock}
\end{frame}



\begin{frame}[fragile]
\frametitle{operátor =}
\begin{bonusblock}{}
\begin{itemize}
\item \lstinline|operator=| přepíše obsah objektu jiným objektem
\begin{itemize}
\item kompilátor vytváří automaticky
\item funguje na stejném principu jako kopírovací konstruktor
\end{itemize}
\end{itemize}
\end{bonusblock}
\vskip -2ex
\begin{bonusblock}{}
\begin{lstlisting}[basicstyle=\small]
struct Pes {
  Pes& operator=(const Pes& pes) {
    if (this == &pes)
      return *this;

    _jmeno = pes._jmeno;
    return *this;
  }

private:
  std::string _jmeno;
};
\end{lstlisting}
\end{bonusblock}
\end{frame}




\begin{frame}[fragile]
\frametitle{Move konstruktor\cpp{11}}
\begin{bonusblock}{}
\begin{itemize}
\item C++11 zavedlo novou \uv{move} sémantiku, funguje jako optimalizace, umožňuje převzít data z dočasných objektů
\begin{itemize}
\item R-value reference (\lstinline|Typ&&|)
\item move konstruktor (\lstinline|Typ(Typ&&))|)
\item move assignment operátor (\lstinline|operator=(Typ&&)|)
\end{itemize}
\end{itemize}
\end{bonusblock}
\vskip -2ex
\begin{bonusblock}{}
\begin{lstlisting}[basicstyle=\small]
struct Pes {

  Pes(Pes&& pes) : _jmeno(std::move(pes._jmeno)) {
    cout << "konstruktor Pes(Pes&&)" << endl;
  }

private:
  std::string _jmeno;
};
\end{lstlisting}
\end{bonusblock}
\end{frame}







\begin{frame}[fragile]
\frametitle{Destruktor}
\begin{noteblock}{}
\begin{lstlisting}
[virtual] ~názevTřídy();
\end{lstlisting}
\end{noteblock}

\begin{bitemize}
\item nemá návratovou hodnotu (ani \lstinline|void|)!
\item \lstinline|virtual| -- je potřeba při využívání polymorfizmu a dědičnosti!
\end{bitemize}

\begin{yesblock}
\begin{lstlisting}
struct Pes {
  ~Pes() {
    cout << "destruktor Pes()" << endl;
  }
};
\end{lstlisting}
\end{yesblock}
\end{frame}
