\hkapitola{Vytváření objektů}

\begin{frame}[fragile]
\begin{block}{Vytváření objektů}
\begin{itemize}
\item Staticky alokované objekty (\uv{statické objekty})
\begin{itemize}
\item na stacku (lokální proměnné, parametry funkcí)
\item globální prostor
\item statické atributy tříd

\end{itemize}
\item Dynamicky alokované objekty
\begin{itemize}
\item new, delete
\end{itemize}

\end{itemize}
\end{block}
\end{frame}

\kapitola{Staticky alokované objekty}

\begin{frame}
\begin{block}{Staticky alokované objekty}
\begin{itemize}
\item vytváří a ruší je kompilátor
\begin{itemize}
\item destruktor volá kompilátor!
\end{itemize}
\item životnost 
\begin{itemize}
\item do konce bloku (lokální proměnné, parametry)
\item do konce programu (globální proměnné, statické atributy tříd)
\end{itemize}
\end{itemize}
\end{block}
\end{frame}


\begin{frame}[fragile]
\frametitle{Vytváření statických objektů}

\begin{yesblock}
\begin{lstlisting}
void main() {
  // bezparametrický konstruktor
  Pes rafan;
  // parametrický konstruktor
  Pes kousak("Kousak", 200);

  // kopírovací konstruktor
  Pes kopieKousaka(kousak);
  Pes jinaKopieKousaka = kousak;
  // operátor=
  jinaKopieKousaka = kousak;
}
\end{lstlisting}
\end{yesblock}
\end{frame}


\begin{frame}[fragile]
\frametitle{Vytváření statických objektů\ldots}
\begin{noblock}
\begin{lstlisting}
// NELZE -> jedná se o deklaraci funkce!
// funkce vracející objekt Pes, bez parametrů
Pes rafan(); 

// lze, ale nevhodné -> může vést k vytvoření dočasného objektu a kopírování
Pes kousak = Pes("Kousak", 200);

// lze, ale nevhodné -> dochází k dvojímu volání destruktoru, dojde ke zničení VMT tabulky -> přestává fungovat polymorfizmus
kousak.~Pes();
\end{lstlisting}
\end{noblock}
\end{frame}



\begin{frame}[fragile]
\frametitle{Uniform initialization\cpp{11}}
\begin{bitemize}
\item C++11
\item použití pomocí složených závorek
\item volá konstruktor nebo inicializuje veřejné datové složky
\item jednotná syntaxe u všech případů!
\end{bitemize}

\begin{yesblock}
\begin{lstlisting}
void main() {
  // bezparametrický konstruktor
  Pes rafan{};
  // parametrický konstruktor
  Pes kousak{"Kousak", 200};
  // kopie
  Pes kopieKousaka{kousak};
}
\end{lstlisting}
\end{yesblock}
\end{frame}


\pkapitola{Volání funkcí -- předávání parametrů, návratová hodnota}


\begin{frame}[fragile]
\frametitle{Parametry funkcí}

\begin{yesblock}
\begin{lstlisting}
void funkce(Pes pes) {
  pes.stekej();
}

Pes rafan{};
// dojde k vytvoření kopie (kopírovací konstruktor)
// &pes != &rafan
funkce(rafan);
\end{lstlisting}
\end{yesblock}
\end{frame}


\begin{frame}[fragile]
\frametitle{Parametry funkcí (reference)}
\begin{bitemize}
\item Reference = předání odkazem
\begin{itemize}
\item \uv{Ukazatelem na pozadí}
\end{itemize}
\end{bitemize}

\begin{yesblock}
\begin{lstlisting}
void funkce(Pes& pes) {
  pes.stekej();
}

Pes rafan{};
// objekt je předán odkazem
// &pes == &rafan
funkce(rafan);
\end{lstlisting}
\end{yesblock}
\end{frame}


\begin{frame}[fragile]
\frametitle{Parametry funkcí (ukazatel)}

\begin{yesblock}
\begin{lstlisting}
void funkce(Pes* pes) {
  pes->stekej();

  // nezmění objekt, ani proměnné rafan a ptr
  // změní pouze lokální proměnnou pes
  pes = 0xdeadbeef;
}

Pes rafan{};
// objekt je předán pomocí ukazatele (je vytvořena kopie "Pes*" = 4(8) bajtové číslo - adresa v paměti)
Pes* ptr = &rafan;
// ptr == &rafan
funkce(ptr);
// ptr == &rafan
\end{lstlisting}
\end{yesblock}
\end{frame}




\begin{frame}[fragile]
\frametitle{Parametry funkcí (ukazatel na ukazatel)}

\begin{yesblock}
\begin{lstlisting}
void funkce(Pes** pes) {
  (*pes)->stekej();

  // nezmění objekt, ani proměnnou rafan
  // změní ukazatel ptr
  *pes = 0xdeadbeef;
}

Pes rafan{};
// objekt je předán pomocí ukazatele (je vytvořena kopie "Pes*" = 4(8) bajtové číslo - adresa v paměti)
Pes* ptr = &rafan;
// ptr == &rafan
funkce(&ptr);
// ptr == 0xdeadbeef
\end{lstlisting}
\end{yesblock}
\end{frame}



\begin{frame}[fragile]
\frametitle{Návratová hodnota}
\begin{bitemize}
\item Vrácení staticky alokovaného objektu vede ke kopírování
\begin{itemize}
\item Normálně se ale využije RVO (return value optimalization -- optimalizace kompilátorem)
\item Standard C++17 definuje \uv{Guaranteed copy elision} (= RVO ve standardu)
\end{itemize}
\end{bitemize}
\vskip -1ex
\begin{yesblock}
\begin{lstlisting}
Pes funkce() {
  Pes rafan{};
  return rafan;
  // objekt rafan - zaniká, ven jde kopie
}

// pes je vytvořen pomocí kopírovacího konstruktoru
// RVO -> kopírování nemusí nastat, objekt je předán "celý"
Pes pes = funkce();

\end{lstlisting}
\end{yesblock}
\end{frame}


\begin{frame}[fragile]
\frametitle{Návratová hodnota (reference)}
\begin{block}{}
Referenci lze vracet, pokud objekt nadále existuje
\begin{itemize}
\item je dynamicky alokovaný
\item je static (pozor u vícevláknových programů)
\item je na globálním prostoru (dtto)
\end{itemize}
\end{block}


\begin{yesblock}
\begin{lstlisting}
Pes& funkce() {
  Pes* rafan = new Pes{};
  return *rafan;
  // dynamicky alokovaný objekt je platný, dokud nezavoláme delete
}

// ok, ale pozor na memory leak!
Pes& pes = funkce();

\end{lstlisting}
\end{yesblock}
\end{frame}



\begin{frame}[fragile]
\frametitle{Návratová hodnota (reference)}
\begin{bitemize}
\item Návratový typ -- reference  + staticky alokovaný lokální objekt => nefunguje
\end{bitemize}

\begin{noblock}
\begin{lstlisting}
Pes& funkce() {
  Pes rafan{};
  return rafan;
  // objekt rafan - zaniká, ven jde zmetek!
}

// pes ukazuje na rozbitou paměť!
Pes& pes = funkce();

// MSVC obsahuje rozšíření, které daný příklad korektně provede, my pojedeme dle standardu C++ a nebudeme na to spoléhat
\end{lstlisting}
\end{noblock}
\end{frame}







\begin{frame}[fragile]
\frametitle{Návratová hodnota (reference)}

\begin{bonusblock}{Prodloužení životnosti dočasných objektů}
\begin{itemize}
\item Standard C++ umožňuje prodloužit životnost objektu v návratové hodnotě
\begin{itemize}
\item pro konstantní L-reference
\item pro R-reference
\end{itemize}
\end{itemize}
\end{bonusblock}

\begin{bonusblock}{}
\begin{lstlisting}
const Pes& funkce() {
  Pes rafan;
  return rafan;
}

// ok - uplatněno prodloužení životnosti objektu dle standardu
const Pes& pes = funkce();

\end{lstlisting}
\end{bonusblock}
\end{frame}


\kapitola{Dynamicky alokované objekty}

\begin{frame}
\begin{block}{Dynamicky alokované objekty}
\begin{itemize}
\item Vznikají a zanikají na náš explicitní příkaz (new / delete)
\end{itemize}
\end{block}
\end{frame}


\begin{frame}[fragile]
\frametitle{Objekt}
%\begin{bitemize}
%\item Objekt
%\end{bitemize}

\begin{yesblock}
\begin{lstlisting}
// vytvoření proměnné typu ukazatel
Pes* rafan = nullptr;
// alokace paměti, volání konstruktoru
rafan = new Pes{};

rafan->stekej();
(*rafan).stekej();

// volání destruktoru, dealokace paměti
delete rafan;
\end{lstlisting}
\end{yesblock}
\end{frame}



\begin{frame}[fragile]
\frametitle{Pole objektů}
%\begin{bitemize}
%\item Pole objektů
%\end{bitemize}

\begin{yesblock}
\begin{lstlisting}
// vytvoření proměnné typu ukazatel
Pes* rafani = nullptr;
// alokace paměti (5 * sizeof(Pes)), 5x volání konstruktoru
rafani = new Pes[5];

rafani[0].stekej();
(rafani + 0)->stekej();
(*(rafani + 0)).stekej();

// volání destruktorů (5x), dealokace paměti
delete[] rafani;
\end{lstlisting}
\end{yesblock}
\end{frame}


\begin{frame}[fragile]
\frametitle{Pole ukazatelů na objekt}
%\begin{bitemize}
%\item Pole ukazatelů na objekt
%\end{bitemize}

\begin{yesblock}

\begin{lstlisting}[basicstyle=\small]
Pes** rafani = nullptr;
rafani = new Pes*[5]; // alokace paměti (5 * sizeof(Pes*)), nevolá konstruktor!

for (int i = 0; i < 5; i++) 
  rafani[i] = new Pes{}; // alokace paměti objektu, konstruktor

rafani[0]->stekej();
(*rafani[0]).stekej();
(**(rafani+0)).stekej();

for (int i = 0; i < 5; i++) 
  delete rafani[i]; // destruktory, dealokace paměti objektů

delete[] rafani; // dealokace paměti pole
\end{lstlisting}
\end{yesblock}
\end{frame}



\begin{frame}[fragile]
\begin{oldblock}
\begin{itemize}
\item malloc, free -- neumí volat konstruktor a destruktor
\begin{itemize}
\item nepoužívejte je
\end{itemize}
\end{itemize}
\end{oldblock}

\begin{deprecatedblock}{}
\begin{lstlisting}[basicstyle={\small},commentstyle={\dcc}]
// vytvoření proměnné typu ukazatel
Pes* rafan = nullptr;
// alokace paměti
rafan = (Pes*)malloc(sizeof(Pes));
// konstrukce
new(rafan) Pes;

// ...

// destrukce
rafan->~Pes();
// uvolnění paměti
free(rafan);
rafan = nullptr;
\end{lstlisting}
\end{deprecatedblock}
\end{frame}




\begin{frame}[fragile]
\begin{noblock}{}
%[basicstyle={\small},commentstyle={\dcc}]
\begin{lstlisting}
// nekorektní kombinace new-delete[] / new[]-delete
Pes* pes = new Pes;
delete[] pes;

pes = new Pes[2];
delete pes;
\end{lstlisting}
\end{noblock}

\begin{noblock}{}
\begin{lstlisting}
// dvojí volání destruktoru
Pes* pes = new Pes;

pes->~Pes();
delete pes;
\end{lstlisting}
\end{noblock}
\end{frame}
