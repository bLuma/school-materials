\hkapitola{STL}

\begin{frame}[fragile]
\begin{block}{STL -- Standard Template Library}
\begin{itemize}
\item představuje generickou knihovnu pro správu a zpracování dat
\item jednotlivé komponenty jsou šablony
\item základní komponenty jsou
\begin{itemize}
\item kontejnery -- slouží pro ukládání a organizaci dat
\item algoritmy -- obecné algoritmy pro zpracování dat
\item iterátory -- obecné \uv{rozhraní} mezi kontejnery a algoritmy
\item pomocné funkční objekty, adaptéry, bindery, \ldots
\end{itemize}
\end{itemize}
\end{block}


\begin{block}{Funkční objekt}
\begin{itemize}
\item vylepšení obyčejné funkce
\item objekt s přetíženým operátorem \lstinline|()|
\begin{itemize}
\item lze volat jako funkci
\item má atributy (stav)
\end{itemize}
\item od C++11 lze výhodně využívat lambda výrazy
\item v STL existuje řada univerzálních funkčních objektů a adaptérů
\end{itemize}
\end{block}
\end{frame}





\begin{frame}[fragile]
\frametitle{Kontejnery}

\begin{block}{}
\begin{itemize}
\item slouží pro ukládání dat
\item standard definuje rozhraní a složitost operací (není definována konkrétní fyzická datová struktura)
\end{itemize}
\end{block}
\vskip -1.5ex
\begin{block}{}
\begin{itemize}
\item posloupnosti -- data nemají vlastní identifikátor (klíč)
\begin{itemize}
\item \lstinline|array|\cpp{11}
\item \lstinline|vector|
\item \lstinline|deque|
\item \lstinline|list|, \lstinline|forward_list|\cpp{11}
\item adaptéry -- \lstinline|stack|, \lstinline|queue|, \lstinline|priority_queue|
\end{itemize}
\item asociativní kontejnery -- data mají vlastní identifikátor (klíč)
\begin{itemize}
\item \lstinline|set|, \lstinline|multiset|
\item \lstinline|map|, \lstinline|multimap|
\item hashovací kontejnery\cpp{11} (\lstinline|unordered_{set,map,multiset,multimap}|)
\end{itemize}
\end{itemize}
\end{block}
\end{frame}






\begin{frame}[fragile]
\frametitle{Iterátory}
\begin{block}{}
\begin{itemize}
\item představují jednotné rozhraní pro procházení dat v libovolném kontejneru
\item vycházejí z ukazatelů a logiky jejich použití (ukazatel na pole je v~zásadě iterátor)
\item obvykle realizovány jako objekty s přetíženými operátory
\item definováno několik kategorií iterátorů dle požadovaných operací
\end{itemize}
\end{block}

\begin{block}{}
\begin{itemize}
\item \lstinline|InputIterator| -- čtou data z kontejneru
\begin{itemize}
\item \lstinline|ForwardIterator|
\item \lstinline|BidirectionalIterator|
\item \lstinline|RandomAccessIterator|
\item \lstinline|ContiguousIterator|\cpp{17}
\end{itemize}
\item \lstinline|OutputIterator| -- zapisují data do kontejneru
\end{itemize}
\end{block}
\end{frame}






\begin{frame}[fragile]
\frametitle{Algoritmy}
\begin{block}{}
\begin{itemize}
\item zpracovávají data
\begin{itemize}
\item vyhledávají
\item mažou
\item upravují
\item obecně zpracovávají prvky
\item řadí
\item \ldots
\end{itemize}
\item pracují s obecnými daty a s obecnými zdrojy dat
\begin{itemize}
\item tvar dat je libovolný (int, string, složitý objekt)
\item zdroj dat je libovolný (používají se iterátory)
\item konkrétní zpracování dat je definováno pomocí operátorů nebo funkčních objektů
\end{itemize}
\end{itemize}
\end{block}
\end{frame}



\kapitola{C++11 a STL, kontejnery, iterátory, algoritmy}

\begin{frame}[fragile]
\frametitle{auto}
\begin{block}{}
\begin{itemize}
\item \lstinline|auto| umožňuje místo psaní konkrétního datového typu nechat typ odvodit kompilátor. 
\item stále se jedná o silné typování, typ proměnné není možné později změnit
\item lze s výhodou využít v mnoha situacích v následující práci s STL
\item \lstinline|auto| je možné dále specifikovat modifikátory
\end{itemize}
\end{block}	

\begin{yesblock}
\lstinline|Object obj;|
\begin{itemize}
\item \lstinline|auto v = obj; // Object v|
\item \lstinline|const auto v = obj; // const Object v|
\item \lstinline|auto& v = obj; // Object& v|
\item \lstinline|const auto& v = obj; // const Object& v|
\end{itemize}
\end{yesblock}
\end{frame}

\begin{frame}[fragile]
\frametitle{initializer\_list}
\begin{block}{}
\begin{itemize}
\item kontejnery je možné inicializovat výčtem hodnot pomocí inicializačních seznamů
\item jde o rozšíření syntaxe uniform initialization
\end{itemize}
\end{block}

\begin{yesblock}
\begin{itemize}
\item  \lstinline|vector<int> vect {10, 15, 20, 25, 30};|
\item  \lstinline|vector<int> vect = {10, 15, 20, 25, 30};|
\end{itemize}
\end{yesblock}	
\end{frame}





\begin{frame}[fragile]
\frametitle{Lambda výrazy -- anonymní funkce}
\begin{noteblock}{}
\begin{center}
[] () \{\}
\end{center}
\end{noteblock}
\end{frame}

\begin{frame}[fragile]
\frametitle{Lambda výrazy -- anonymní funkce}
\begin{noteblock}{}
\begin{center}
[ zachycení vnějších proměnných ] ( parametry funkce ) \{ tělo funkce \}
\end{center}
\end{noteblock}
\end{frame}

\begin{frame}[fragile]
\frametitle{Lambda výrazy -- zachycení vnějších proměnných}
\begin{yesblock}
\begin{lstlisting}
int x = 10;

// auto lambda0 = []() { return x; }; // nejde - x není definováno v anonymní funkci

// zachycení hodnotou
auto lambda1 = [x]() { return x; }; // ++x - nejde (je read only)
cout << lambda1(); // x = 10; out = 10;

// zachycení referencí (odkazem)
auto lambda2 = [&x]() { return ++x; };
cout << lambda2(); // x = 11; out = 11;
\end{lstlisting}
\end{yesblock}
\end{frame}


\begin{frame}[fragile]
\frametitle{Lambda výrazy -- parametry funkce}
\begin{yesblock}
\begin{lstlisting}
auto lambda0 = []() { return 1; };
cout << lambda0() << endl; // 1

auto lambda1 = [](int x) { return 2 * x; };
cout << lambda1(10) << endl; // 20

auto lambda2 = [](int x, int y, double t) { return x + y; };
cout << lambda2(10, 20, 3.1415) << endl; // 30
\end{lstlisting}
\end{yesblock}
\end{frame}

\begin{frame}[fragile]
\begin{noteblock}{specifikace návratového typu}
\begin{center}
[ ] ( ) -> návratový\_typ \{ \}
\end{center}
\end{noteblock}

\begin{noteblock}{mutable}
\begin{center}
[ ] ( ) \textit{mutable} -> návratový\_typ \{ \}
\end{center}

Umožňuje měnit vnější hodnoty zachycené hodnotou (změna se projeví pouze v těle lambdy).
\end{noteblock}

\begin{block}{zachycení vnejších hodnot}
\begin{itemize}
\item \lstinline |[=]| -- vše hodnotou
\item \lstinline |[&]| -- vše odkazem
\item \lstinline |[x, &]| -- x hodnotou, zbytek odkazem
\item \lstinline |[this]| -- zachycení this v objektu 
\end{itemize}
\end{block}
\end{frame}

\begin{frame}[fragile]
\frametitle{Předávání / uchování anonymních funkcí}
\begin{yesblock}
\begin{lstlisting}
// použitím auto
auto lambda1 = [](int value){ return value = 1; }

// použitím std::function<>
std::function<int(int)> = [](int value){ return value = 1;}
\end{lstlisting}
\end{yesblock}

\begin{block}{std::function<>}
\begin{itemize}
\item šablona s proměnným počtem parametrů
\item parametry ve tvaru \lstinline|result(param1, param2, ...)|
\begin{itemize}
\item \lstinline|function<int()>| -- funkce bez parametrů, vrací int
\item \lstinline|function<void(int)>| -- funkce s jedním int parametrem, vrací void
\item \lstinline|function<int(int)>| -- funkce s jedním int parametrem, vrací int
\item \lstinline|function<int(int, int)>| -- funkce se dvěma int parametry, vrací int
\end{itemize}

\end{itemize}
\end{block}
\end{frame}


\begin{frame}[fragile]
%\frametitle{std::function}
\begin{block}{std::function<>}
\begin{itemize}
\item umožňuje uchovávat i ukazatel na libovolnou funkci nebo metodu
\end{itemize}
\end{block}

\begin{yesblock}
\begin{lstlisting}
std::function<void(Citac&, int)> fptr = &Citac::nastavAtribut;
	
Citac citac{};
fptr(citac, 10);

\end{lstlisting}
\end{yesblock}
\end{frame}





