\hkapitola{Algoritmy}


\begin{frame}[fragile]
\frametitle{Algoritmy}
\begin{block}{}
\begin{itemize}
\item obecné algoritmy pro zpracování dat v kontejnerech (nebo i jinde -- využívá se iterátorů pro přístup k datům)
\item hlavičkové soubory \lstinline|<algorithm>| a \lstinline|<numeric>|
\end{itemize}
\end{block}
\end{frame}


\kapitola{Algoritmus copy}


\begin{frame}[fragile]
\frametitle{Algoritmus copy}
\begin{block}{}
\begin{itemize}
\item kopíruje prvky ze vstupního rozsahu na cílové umístění
\begin{itemize}
\item cílem může být jiný kontejner
\item ale i jiná část zdrojového kontejneru
\item pomocí adaptérů je možné prvky zapisovat i do proudů (soubor, konzole)
\end{itemize}
\item naivní implementace je jednoduchý cyklus a operátor= pro zkopírování prvků
\begin{itemize}
\item reálně složité šablony, které na základě typu iterátorů použijí nejvhodnější algoritmus
\item v případě spojitých oblastí paměti lze využít i \lstinline|memcpy|
\end{itemize}
\item existuje celá řada variant algoritmu
\begin{itemize}
\item \lstinline|copy_n|, \lstinline|copy_if|, \lstinline|copy_backward|

\end{itemize}
\end{itemize}
\end{block}

\begin{yesblock}
\begin{lstlisting}
copy(src.begin(), src.end(), dest.begin());
\end{lstlisting}
\end{yesblock}
\end{frame}


\begin{frame}[fragile]
\begin{block}{copy - možná implementace}
\begin{lstlisting}
template<class InputIterator, class OutputIterator>
OutputIterator copy (InputIterator first, InputIterator last, OutputIterator result)
{
	while (first != last) {
		*result = *first;
		
		++result; 
		++first;
	}
	
	return result;
}
\end{lstlisting}
\end{block}
\end{frame}

\begin{frame}[fragile]
\begin{block}{copy - možná implementace}
\begin{lstlisting}
template<class InputIterator, class OutputIterator>
OutputIterator copy (InputIterator first, InputIterator last, OutputIterator result)
{
	// iteruj, dokud nedojdeme na konec
	while (first != last) {
		// z iterátoru first vezmi objekt a ulož ho na místo kam ukazuje it. result
		*result = *first;
		
		// posuň oba iterátory na další prvek
		++result; 
		++first;
	}
	
	return result;
}
\end{lstlisting}
\end{block}
\end{frame}

\begin{frame}[fragile]
\begin{noblock}
\begin{lstlisting}
vector<int> source {0, 1, 20, 4, 50, 60};
vector<int> destination;

copy(source.begin(), source.end(), destination.begin());
\end{lstlisting}
\end{noblock}

\begin{yesblock}
\begin{lstlisting}
vector<int> source {0, 1, 20, 4, 50, 60};
vector<int> destination;

// 1) použití reserve/resize
destination.resize(source.size());
copy(source.begin(), source.end(), destination.begin());

// 2) použití vkládacího iterátorového adaptéru
copy(source.begin(), source.end(), back_inserter(destination));
\end{lstlisting}
\end{yesblock}
\end{frame}





\kapitola{Algoritmus find}


\begin{frame}[fragile]
\frametitle{Algoritmus find}
\begin{block}{}
\begin{itemize}
\item vyhledá prvek v zadaném rozsahu
\item vrací iterátor na jeho pozici nebo vrací koncový iterátor
\item prvek pozná podle shody (==) s jiným prvkem
\begin{itemize}
\item existuje predikátová verze algoritmu \lstinline|find_if|
\end{itemize}
\end{itemize}
\end{block}
\end{frame}



\begin{frame}[fragile]
\begin{block}{find - možná implementace}
\begin{lstlisting}
template<class InputIterator, class T>
InputIterator find (InputIterator first, InputIterator last, const T& val)
{
	while (first != last) {
		if (*first == val) 
			return first;
		
		++first;
	}
	
	return last;
}		
\end{lstlisting}
\end{block}
\end{frame}

\begin{frame}[fragile]
\begin{yesblock}
\begin{lstlisting}
vector<int> vect {0, 4, 45, 20, 40, 50};

auto it = find(vect.begin(), vect.end(), 40);
if (it != vect.end()) 
	cout << "prvek 40 nalezen";
\end{lstlisting}
\end{yesblock}

\begin{yesblock}
\begin{lstlisting}
vector<Osoba> vect {{"Jan", "Stastny"}, {"Petr", "Mlady"}};

auto it = find(vect.begin(), vect.end(), Osoba{"Jiri", "Veliky"});
if (it != vect.end()) 
	cout << "Osoba Jiri Veliky nalezena";
\end{lstlisting}
\end{yesblock}
\end{frame}



\begin{frame}[fragile]
\frametitle{Algoritmus find\_if}

\begin{block}{}
\begin{itemize}
\item stejný princip jako \lstinline|find|, ale využívá predikátovou funkci na porovnání prvků
\end{itemize}
\end{block}
\end{frame}


\begin{frame}[fragile]
\begin{block}{find\_if - možná implementace}
\begin{lstlisting}
template<class InputIterator, class UnaryPredicate>
InputIterator find_if (InputIterator first, InputIterator last, UnaryPredicate pred)
{
	while (first != last) {
		if (pred(*first)) 
			return first;
			
		++first;
	}
	
	return last;
}	
\end{lstlisting}
\end{block}
\end{frame}


\begin{frame}[fragile]
\frametitle{Predikát -- obyčejná funkce}

\begin{yesblock}
\begin{lstlisting}
bool jeVetsiNez40(int hodnota) {
	return hodnota > 40;
}
\end{lstlisting}
\end{yesblock}

\begin{yesblock}
\begin{lstlisting}
vector<int> vect {0, 4, 45, 20, 40, 50};

auto it = find_if(vect.begin(), vect.end(), jeVetsiNez40);
if (it != vect.end()) 
	cout << "prvek vetsi nez 40 nalezen -> " << *it;
\end{lstlisting}
\end{yesblock}
\end{frame}

\begin{frame}[fragile]
\frametitle{Predikát -- funkční objekt}
\begin{yesblock}
\begin{lstlisting}
struct jeVetsiNez40FO {
	bool operator()(int prvek) const {
		return prvek > 40;
	}
};
\end{lstlisting}
\end{yesblock}
\begin{yesblock}
\begin{lstlisting}
vector<int> vect {0, 4, 45, 20, 40, 50};

auto it = find_if(vect.begin(), vect.end(), jeVetsiNez40FO{});
if (it != vect.end()) 
	cout << "prvek vetsi nez 40 nalezen -> " << *it;
\end{lstlisting}
\end{yesblock}
\end{frame}










\begin{frame}[fragile]
\frametitle{Predikát -- standardní funkční objekty}
\begin{yesblock}
\begin{lstlisting}
vector<int> vect {0, 4, 45, 20, 40, 50};

// 1) std::binder1st, std::binder2nd - deprecated od C++11
auto it = find_if(vect.begin(), vect.end(), 
	bind2nd(greater<int>(), 40)
	);
	
// 2) std::bind (od C++11) - lepší, obecnější, trochu složitější
using std::placeholders::_1;
auto it = find_if(vect.begin(), vect.end(), 
	bind(greater<int>(), _1, 40)
	);
	
if (it != vect.end()) 
	cout << "prvek vetsi nez 40 nalezen -> " << *it;
\end{lstlisting}
\end{yesblock}
\end{frame}

\begin{frame}[fragile]
\frametitle{Predikát -- lambda výraz}
\begin{yesblock}
\begin{lstlisting}
vector<int> vect {0, 4, 45, 20, 40, 50};

auto it = find_if(vect.begin(), vect.end(), 
	[ ](int prvek) { return prvek > 40;}
	);

if (it != vect.end()) 
	cout << "prvek vetsi nez 40 nalezen -> " << *it;
\end{lstlisting}
\end{yesblock}
\end{frame}



\kapitola{Funkční objekty}


\begin{frame}[fragile]
\begin{block}{Funkční objekty, adaptéry, bindery}
\begin{itemize}
\item \lstinline|<functional>|
\item umožňují psát obecné funkce pro použití s algoritmy jako predikáty, transformační funkce, \ldots
\item systém adaptérů značně obměněn ve standardu C++11, změny pokračují v C++17
\end{itemize}
\end{block}
\end{frame}


\begin{frame}[fragile]
\frametitle{binder1st, binder2nd}

\begin{oldblock}{}
\begin{itemize}
\item \lstinline|std::binder1st, std::binder2nd| -- převádějí binární funkci na unární
\item jeden z parametrů je nahrazen konstantou, může být vypočtena i v době vytváření binderu
\item od C++11 deprecated, od C++17 odebráno, bylo nahrazeno jednotným \lstinline|std::bind|
\item pro snadnější zápis se používaly funkce \lstinline|bind1st(), bind2nd()|
\end{itemize}
\vskip 1ex

\begin{lstlisting}[basicstyle=\scriptsize]
struct Plus : std::binary_function<int, int, int> {
	int operator()(int a, int b) const {
		return a + b;
	}
};
////////////////////////////////////////////////////
auto plus5 = bind2nd(Plus{}, 5);

cout << plus5(1) << endl; // 6
cout << plus5(5) << endl; // 10
\end{lstlisting}
\end{oldblock}
\end{frame}


\begin{frame}[fragile]
\frametitle{Deprecated/removed funkční objekty}

\begin{oldblock}{}
\begin{itemize}
\item kromě \lstinline|std::binder1st, std::binder2nd| byla odebrána řada dalších funkčních objektů
\item \lstinline|std::unary_function, std::binary_function| -- definuje typy argumentů a návratové hodnoty funkce
\item \lstinline|std::ptr_fun| -- vytváří \uv{objektový} ukazatel na funkci
\item \lstinline|std::mem_fun| -- dtto na metodu (skrze ukazatel)
\item \lstinline|std::mem_fun_ref| --  dtto na metodu (skrze referenci)
\item a objekty/wrappery s nimi související
\end{itemize}
\end{oldblock}
\end{frame}


\begin{frame}[fragile]
\frametitle{Funkční objekty -- operátorové -- aritmetické}
\begin{block}{}
\begin{itemize}
\item \lstinline|std::plus| -- $x + y$
\item \lstinline|std::minus| -- $x - y$
\item \lstinline|std::multiplies| -- $x * y$
\item \lstinline|std::divides| -- $x / y$
\item \lstinline|std::modulus| -- $x \% y$
\item \lstinline|std::negate| -- $-x$
\end{itemize}
\end{block}
\end{frame}


\begin{frame}[fragile]
\frametitle{Funkční objekty -- operátorové -- porovnávací}
\begin{block}{}
\begin{itemize}
\item \lstinline|std::equal_to| -- $x == y$
\item \lstinline|std::not_equal_to| -- $x != y$
\item \lstinline|std::greater| -- $x > y$
\item \lstinline|std::less| -- $x < y$
\item \lstinline|std::greater_equal| -- $x >= y$
\item \lstinline|std::less_equal| -- $x <= y$
\end{itemize}
\end{block}
\end{frame}

\begin{frame}[fragile]
\frametitle{Funkční objekty -- operátorové -- logické a bitové}
\begin{block}{}
\begin{itemize}
\item \lstinline|std::logical_and| -- $x \&\& y$
\item \lstinline|std::logical_or| -- $x || y$
\item \lstinline|std::logical_not| -- $!x$
\item \lstinline|std::bit_and| -- $x \& y$
\item \lstinline|std::bit_or| -- $x | y$
\item \lstinline|std::bit_xor| -- $x $\^{}$ y$
\item \lstinline|std::bit_not|\cpp{14} -- \~{}$x$
\end{itemize}
\end{block}
\end{frame}




\begin{frame}[fragile]
\frametitle{Funkční objekty -- příklad}
\begin{yesblock}
\begin{lstlisting}
std::multiplies<double> multipliesDouble{};
cout << multipliesDouble(10, 3.2) << endl; 
// 32

std::equal_to<int> equalToInt{};
cout << (equalToInt(4, 4) ? "T" : "F") << endl; 
// T

std::equal_to<string> equalToString{};
cout << (equalToString("hello", "hella") ? "T" : "F") << endl; 
// F
\end{lstlisting}
\end{yesblock}
\end{frame}

\newcommand{\impbullet}{\textcolor{red}{\LARGE\textbullet}}
\newcommand{\impbulletl}{\textcolor{red}{\large\textbullet}}


\begin{frame}[fragile]
\frametitle{Funkční objekty}
\begin{block}{Function wrappers}
\begin{itemize}
\item[\impbullet] \lstinline|function|\cpp{11} -- slouží jako wrapper pro libovolný druh funkce (funkce, metoda, lambda, \ldots) dle daného předpisu
\item \lstinline|mem_fn|\cpp{11} -- vytváří funkční objekt ze specifikované metody
\end{itemize}
\end{block}

\begin{block}{Invoke}
\begin{itemize}
\item \lstinline|invoke|\cpp{17} -- vyvolání funkce/objektu se specifikovanými argumenty
\end{itemize}
\end{block}
\end{frame}


\begin{frame}[fragile]
\frametitle{Funkční objekty}

\begin{block}{Bind}
\begin{itemize}
\item[\impbullet] \lstinline|bind|\cpp{11} -- vytvoří fce objekt s možností statického nastavení některých parametrů
\item \lstinline|is_bind_expression|\cpp{11} -- indikuje, že daný objekt je výrazem \lstinline|std::bind|
\item \lstinline|is_placeholder|\cpp{11} -- indikuje, že daný objekt je standardní placeholder
\end{itemize}
\end{block}

\begin{block}{Negators}
\begin{itemize}
\item \lstinline|not_fn|\cpp{17} -- vytvoří fce objekt negující výsledek specifikované funkce
\item Do C++17 (pak deprecated)
\begin{itemize}
\item[\impbulletl] \lstinline|not1| -- vytváří fce objekt negující výsledek unární funkce
\item[\impbulletl] \lstinline|not2| -- vytváří fce objekt negující výsledek binární funkce
\end{itemize}
\end{itemize}
\end{block}
\end{frame}


\begin{frame}[fragile]
\frametitle{Funkční objekty}
\begin{block}{Searchers}
\begin{itemize}
\item \lstinline|default_searcher|\cpp{17}
\item \lstinline|boyer_moore_searcher|\cpp{17}
\item \lstinline|boyer_moore_horspool_searcher|\cpp{17}
\end{itemize}
\end{block}

\begin{block}{Reference wrapper}
\begin{itemize}
\item \lstinline|reference_wrapper|\cpp{11}
\item[\impbulletl] \lstinline|ref|\cpp{11} -- konstruuje wrapper pro referenci
\item[\impbulletl] \lstinline|cref|\cpp{11} -- konstruuje wrapper pro konstatní referenci
\end{itemize}
\end{block}
\end{frame}


\pkapitola{bind}

\begin{frame}[fragile]
\frametitle{std::bind}

\begin{block}{}
\begin{itemize}
\item umožňuje zabalit libovolnou funkci, funkční objekt, lambda výraz, metodu do funkčního objektu
\item je možné nahradit specifické parametry konkrétní hodnotou nebo referencí
\item je možné změnit pořadí parametrů
\item je možné u metody specifikovat konkrétní objekt nad kterým má být metoda volána
\item volitelné parametry se dosazují pomocí konstant \_1, \_2,~\ldots{} z~jmenného prostoru \lstinline|std::placeholders|
\end{itemize}
\end{block}
\end{frame}


\begin{frame}[fragile]
\frametitle{std::bind -- základní použití}

\begin{yesblock}
\begin{lstlisting}[basicstyle=\small]
void printVar(string variable, int value) {
	cout << variable << ": " << value << endl;
}
////////////////////////////////////////////////
auto pv1 = bind(printVar, "V1", 100);
pv1();
// V1: 100

using namespace std::placeholders;
auto pv2 = bind(printVar, "V2", _1);
pv2(200);
pv2(300);
// V2: 200
// V2: 300

auto pv3 = bind(printVar, _2, _1);
pv3(400, "V3");
// V3: 400
\end{lstlisting}
\end{yesblock}
\end{frame}



\begin{frame}[fragile]
\frametitle{std::bind -- použití na metody}

\begin{yesblock}
\begin{lstlisting}[basicstyle=\small]
using namespace std::placeholders;

struct Counter {
	int counter; 
	
	Counter(int c) : counter(c) {}

	void set(int v) { counter = v; }
	int get() const { return counter; }
};
\end{lstlisting}
\end{yesblock}
\end{frame}

\begin{frame}[fragile]
\frametitle{std::bind -- použití na metody\ldots}

\begin{yesblock}
\begin{lstlisting}[basicstyle=\small]
auto setterTo100 = bind(&Counter::set, _1, 100);
Counter c1{ 0 };
setterTo100(c1);

Counter c{ 0 };
auto cSetter = bind(&Counter::set, &c, _1);
cSetter(20);
// c.counter == 20

auto cGetter = bind(&Counter::get, &c);
cout << cGetter() << endl;
// 20
\end{lstlisting}
\end{yesblock}
\end{frame}



\begin{frame}[fragile]
\frametitle{std::bind -- použití na reference}

\begin{yesblock}
\begin{lstlisting}[basicstyle=\small]
void increment(int& value) {
	value++;
	cout << value << endl;
}
///////////////////////////////////////////

int v = 0;
// do objektu i1 se přenese kopie v
auto i1 = bind(increment, v);
i1(); // 1
i1(); // 2
cout << v << endl; // 0

// do objektu i1 se předá reference_wrapper
auto i2 = bind(increment, ref(v));
i2(); // 1
i2(); // 2
cout << v << endl; // 2
\end{lstlisting}
\end{yesblock}
\end{frame}















\kapitola{Algoritmus remove}

\begin{frame}[fragile]
\frametitle{Algoritmus remove}
\begin{block}{}
\begin{itemize}
\item slouží k odebrání prvků z rozsahu
\item odebírá prvky podle shody (==)
\item iterátory neumí přidat nebo odebrat prvky!
\begin{itemize}
\item algoritmus přeuspořádá pořadí prvků
\item vrátí iterátor na nový konec rozsahu
\item vlastní odebrání zbylých prvků je nutné provést ručně
\end{itemize}
\end{itemize}
\end{block}
\end{frame}





\begin{frame}[fragile]
\begin{block}{remove -- možná implementace}
\begin{lstlisting}
template <class ForwardIterator, class T>
ForwardIterator remove (ForwardIterator first, ForwardIterator last, const T& val)
{
	ForwardIterator result = first;
	while (first != last) {
		if (!(*first == val)) {
			*result = *first;
			++result;
		}
		++first;
	}
	
	return result;
}\end{lstlisting}
\end{block}
\end{frame}




\begin{frame}[fragile]
\begin{block}{remove -- možná implementace}
\begin{lstlisting}
template <class ForwardIterator, class T>
ForwardIterator remove (ForwardIterator first, ForwardIterator last, const T& val)
{
	ForwardIterator result = first; // vytvoř kopii iterátoru first
	while (first != last) { // projdi všechny prvky v rozsahu
		if (!(*first == val)) { // pokud nejsou shodné s odebíraným
			*result = *first; // zapiš do result
			++result; // posuň result
		}
		++first; // posuň first
	}
	
	return result;
}\end{lstlisting}
\end{block}
\end{frame}

\begin{frame}[fragile]
\frametitle{remove -- postup činnosti}
\begin{block}{}
\begin{lstlisting}
remove({1, 10, 2, 10, 3, 4, 5, 10, 6, 7}, 10);
\end{lstlisting}
\end{block}
\begin{block}<2->{}
\newcommand{\F}{{\textbf{\color[rgb]{0,0.6,0}F}}}
\newcommand{\R}{{\textbf{\color[rgb]{0.6,0,0}R}}}

\only<2>{
[\F\R 1, 10, 2, 10, 3, 4, 5, 10, 6, 7]

first = 1

[\textbf{1}, \F\R 10, 2, 10, 3, 4, 5, 10, 6, 7]
}

\only<3>{
[1, \F\R 10, 2, 10, 3, 4, 5, 10, 6, 7]

first == 10! nezapisuj

[1, \R 10, \F 2, 10, 3, 4, 5, 10, 6, 7]
}

\only<4>{
[1, \R 10, \F 2, 10, 3, 4, 5, 10, 6, 7]

first = 2

[1, \textbf{2}, \R 2, \F 10, 3, 4, 5, 10, 6, 7]
}

\only<5>{
[1, 2, \R 2, \F 10, 3, 4, 5, 10, 6, 7]

first = 10! nezapisuj

[1, 2, \R 2, 10, \F 3, 4, 5, 10, 6, 7]
}

\only<6>{
[1, 2, \R 2, 10, \F 3, 4, 5, 10, 6, 7]

first = 3

[1, 2, \textbf{3}, \R 10, 3, \F 4, 5, 10, 6, 7]
}

\only<7>{
[1, 2, 3, \R 10, 3, \F 4, 5, 10, 6, 7]

first = 4

[1, 2, 3, \textbf{4}, \R 3, 4, \F 5, 10, 6, 7]
}

\only<8>{
[1, 2, 3, 4, \R 3, 4, \F 5, 10, 6, 7]

first = 5

[1, 2, 3, 4, \textbf{5}, \R 4, 5, \F 10, 6, 7]
}

\only<9>{
[1, 2, 3, 4, 5, \R 4, 5, \F 10, 6, 7]

first == 10! nezapisuj 

[1, 2, 3, 4, 5, \R 4, 5, 10, \F 6, 7]
}

\only<10>{
[1, 2, 3, 4, 5, \R 4, 5, 10, \F 6, 7]

first = 6

[1, 2, 3, 4, 5, \textbf{6}, \R 5, 10, 6, \F 7]
}

\only<11>{
[1, 2, 3, 4, 5, 6, \R 5, 10, 6, \F 7]

first = 7

[1, 2, 3, 4, 5, 6, \textbf{7}, \R 10, 6, 7] \F
}


\only<12>{
[\textbf{1, 2, 3, 4, 5, 6, 7}, \R \textit{10, 6, 7}]
}

\end{block}
\end{frame}


\begin{frame}[fragile]
\begin{yesblock}
\begin{lstlisting}
vector<int> vect {1, 10, 2, 10, 3, 4, 5, 10, 6, 7};

auto r = remove(vect.begin(), vect.end(), 10);
// vect = [1, 2, 3, 4, 5, 6, 7, 10, 6, 7]

vect.erase(r, vect.end()); // smazání prvků 10, 6, 7
// vect = [1, 2, 3, 4, 5, 6, 7]
\end{lstlisting}
\end{yesblock}
\end{frame}



\kapitola{Přehled algoritmů}


\begin{frame}[fragile]
\frametitle{Nemodifikující algoritmy}
\begin{block}{}
\centering
\begin{tabular}{ll}
\lstinline|all_of|\cpp{11} &  \\
\lstinline|any_of|\cpp{11} & testuje splnění podmínky prvky \\
\lstinline|none_of|\cpp{11} &  \\
\hline
\lstinline|for_each| & provede operaci nad každým prvkem \\
\hline
\lstinline|for_each_n|\cpp{17} & dtto + limit pro n prvků \\
\hline
\lstinline|count| & spočítá počet prvků \\
\lstinline|count_if| & \\
\hline
\lstinline|mismatch| & vyhledá pozici neshody mezi dvěma množinami \\
\hline
\lstinline|equal| & testuje shodu dvou množin \\
\hline
\lstinline|find| & \\
\lstinline|find_if| & najde první prvek podle hodnoty/podmínky \\
\lstinline|find_if_not|\cpp{11} & \\
\hline
\lstinline|adjacent_find| & najde dva po sobě jdoucí shodné prvky \\
\lstinline|| & (nebo dle podmínky) \\
\hline
\lstinline|search| & hledá rozsah prvků \\
\lstinline|search_n| &  \\
\end{tabular}
\end{block}
\end{frame}



\begin{frame}[fragile]
\frametitle{Algoritmy modifikující}
\begin{block}{}
\centering
\begin{tabular}{ll}
\lstinline|copy| &  kopíruje prvky \\
\lstinline|copy_if|\cpp{11} & \\
\hline
\lstinline|copy_n| & dtto + specifikovaný počet prvků \\
\hline
\lstinline|copy_backward| & dtto + otáčí pořadí \\
\hline
\lstinline|move|\cpp{11} & přesouvá prvky \\
\lstinline|move_backward|\cpp{11} & \\
\hline
\lstinline|fill| & nastaví hodnotu prvků \\
\hline
\lstinline|fill_n| & dtto + specifikovaný počet prvků \\
\hline
\lstinline|transform| & transformuje prvky (aplikuje na ně funkci) \\
\hline
\lstinline|generate| & nastaví hodnotu prvků dle funkce \\
\hline
\lstinline|generate_n| & dtto + specifikovaný počet prvků \\
\end{tabular}
\end{block}
\end{frame}


\begin{frame}[fragile]
\frametitle{Algoritmy modifikující\ldots}
\begin{block}{}
\centering
\begin{tabular}{ll}
\lstinline|remove| &  odebírá prvky \\
\lstinline|remove_if| & \\
\hline
\lstinline|remove_copy| & kopíruje neodebrané prvky \\
\lstinline|remove_copy_if| & \\
\hline
\lstinline|replace| & nahrazuje hodnoty \\
\lstinline|replace_if| &  \\
\hline
\lstinline|replace_copy| & kopíruje a nahrazuje hodnoty \\
\lstinline|replace_copy_if| & \\
\hline
\lstinline|swap_ranges| & prohodí prvky z dvou rozsahů \\
\hline
\lstinline|reverse| & otáčí pořadí prvků \\
\hline
\lstinline|reverse_copy| & dtto + výsledek kopíruje \\
\hline
\lstinline|rotate| & rotuje pořadí prvků \\
\hline
\lstinline|rotate_copy| & dtto + výsledek kopíruje \\
\hline
\lstinline|random_shuffle|\cpp{03,11,14,-} & náhodně zamíchá prvky \\
\lstinline|shuffle|\cpp{17} &  \\
\end{tabular}
\end{block}
\end{frame}



\begin{frame}[fragile]
\frametitle{Algoritmy modifikující\ldots}
\begin{block}{}
\centering
\begin{tabular}{ll}
\lstinline|sample|\cpp{17} & vybere náhodný vzorek \\
\hline
\lstinline|unique| & maže sousedící duplicity \\
\hline
\lstinline|unique_copy| & dtto + výsledek kopíruje \\
\end{tabular}
\end{block}
\end{frame}



\begin{frame}[fragile]
\frametitle{Algoritmy dělící na skupiny}
\begin{block}{}
\centering
\begin{tabular}{ll}
\lstinline|is_partitioned|\cpp{11} & je skupinou dle podmínky \\
\hline
\lstinline|partition| & rozdělí prvky na dvě skupiny  \\
\hline
\lstinline|partition_copy|\cpp{11} & dtto + výsledek kopíruje \\
\hline
\lstinline|stable_partition| & dtto + zachovává relativní pořadí prvků \\
\hline
\lstinline|partition_point|\cpp{11} & vrací dělící místo v rozsahu \\
\end{tabular}
\end{block}
\end{frame}



\begin{frame}[fragile]
\frametitle{Algoritmy řadící}
\begin{block}{}
\centering
\begin{tabular}{ll}
\lstinline|is_sorted|\cpp{11} & testuje jestli je rozsah seřazený \\
\hline
\lstinline|is_sorted_until|\cpp{11} & vrací největší seřazený rozsah \\
\hline
\lstinline|sort| & řadí rozsah \\
\hline
\lstinline|partial_sort| & řadí N prvních prvků \\
\hline
\lstinline|partial_sort_copy| & dtto + výsledek kopíruje \\
\hline
\lstinline|stable_sort| & řadí rozsah, zachovává relativní pořadí prvků \\
\hline
\lstinline|nth_element| & částečně řadí rozsah dle spec. prvku \\
\end{tabular}
\end{block}
\end{frame}


\begin{frame}[fragile]
\frametitle{Algoritmy vyhledávající na seřazeném rozsahu}
\begin{block}{}
\centering
\begin{tabular}{ll}
\lstinline|lower_bound| & \\
\hline
\lstinline|upper_bound| &  \\
\hline
\lstinline|binary_search| &  \\
\hline
\lstinline|equal_range| &  \\
\end{tabular}
\end{block}
\end{frame}



\begin{frame}[fragile]
\frametitle{Množinové operace (na seřazeném rozsahu)}
\begin{block}{}
\centering
\begin{tabular}{ll}
\lstinline|merge| &  \\
\hline
\lstinline|inplace_merge| &  \\
\hline
\lstinline|includes| &  \\
\hline
\lstinline|set_difference| &  \\
\hline
\lstinline|set_intersection| &  \\
\hline
\lstinline|set_symmetric_difference| &  \\
\hline
\lstinline|set_union| & \\
\end{tabular}
\end{block}
\end{frame}



\begin{frame}[fragile]
\frametitle{Haldové operace}
\begin{block}{}
\centering
\begin{tabular}{ll}
\lstinline|is_heap| &  \\
\hline
\lstinline|is_heap_until| &  \\
\hline
\lstinline|make_heap| & \\
\hline
\lstinline|push_heap| &  \\
\hline
\lstinline|pop_heap| & \\
\hline
\lstinline|sort_heap| & \\
\end{tabular}
\end{block}
\end{frame}


\begin{frame}[fragile]
\frametitle{Operace pro minimum/maximum}
\begin{block}{}
\centering
\begin{tabular}{ll}
\lstinline|max| &  \\
\hline
\lstinline|max_element| &  \\
\hline
\lstinline|min| & \\
\hline
\lstinline|min_element| &  \\
\hline
\lstinline|minmax|\cpp{11} & \\
\hline
\lstinline|minmax_element|\cpp{11} & \\
\hline
\lstinline|clamp|\cpp{17} & \\
\hline
\lstinline|lexicographical_compare| &  \\
\hline
\lstinline|is_permutation|\cpp{11} & \\
\hline
\lstinline|next_permutation| &  \\
\hline
\lstinline|prev_permutation| &  \\
\end{tabular}
\end{block}
\end{frame}

\begin{frame}[fragile]
\frametitle{Numerické operace}
\begin{block}{}
\centering
\begin{tabular}{ll}
\lstinline|iota|\cpp{11} &  \\
\hline
\lstinline|accumulate| &  \\
\hline
\lstinline|inner_product| & \\
\hline
\lstinline|adjacent_difference| &  \\
\hline
\lstinline|partial_sum| & \\
\hline
\lstinline|reduce|\cpp{17} & \\
\hline
\ldots \\
\end{tabular}
\end{block}
\end{frame}