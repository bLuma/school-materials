\hkapitola{Jmenné prostory}

\begin{frame}[fragile]
\frametitle{Jmenné prostory}

\begin{block}{}
\begin{itemize}
\item řeší konflikty jmen
\begin{itemize}
\item v C existuje jediný globální prostor
\item knihovní funkce se prefixují, aby nedošlo ke shodě (\lstinline|glutCreateWindow, png_image_begin_read_from_file|)
\end{itemize}

\item jmenné prostory jsou pouze balíkem funkcí, typů, proměnných, \ldots
\begin{itemize}
\item neřeší se zde viditelnost (neexistuje obdoba package-private)
\item lze vytvořit anonymní jmenný prostor pro ukrytí členů jen v rámci jednoho souboru
\end{itemize}

\item není definováno striktní fyzické uspořádání souborů
\begin{itemize}
\item Java -- balíček = složka, obsahuje pouze soubory v této složce
\item C++ -- jmenný prostor může být použit kdekoliv
\item C++ -- do existujícího jmenného prostoru je možno kdykoliv cokoliv přidat
\end{itemize}

\item jmenné prostory je možné vnořovat
\end{itemize}
\end{block}
\end{frame}



\begin{frame}[fragile]
\frametitle{Vytvoření jmenného prostoru}
\begin{noteblock}{}
\begin{lstlisting}
namespace názevJmennéhoProstoru {
  [deklarace/definice složek přidávaných <@do@> jm. prostoru] 
}
\end{lstlisting}
\end{noteblock}

\begin{block}{}
\begin{itemize}
\item jm. p. je možné vytvořit na globální úrovni nebo ve jmenném prostoru
\item jm. p. je možné opakovaně definovat -- vždy se přidají nové prvky do prostoru
\item přístup k prvkům jm. p. je pomocí operátoru \lstinline|::|
\end{itemize}
\end{block}
\end{frame}




\begin{frame}[fragile]
\frametitle{Přístup k prvkům jmenného prostoru z vnitřku}
\begin{yesblock}
\begin{lstlisting}
namespace System {

  void vymazKonzoli() { ... }

  void ukonciProgram() { 
    vymazKonzoli();
  }
}
\end{lstlisting}
\end{yesblock}
\end{frame}

\begin{frame}[fragile]
\frametitle{Přístup k prvkům jmenného prostoru zvenčí}
\begin{yesblock}
\begin{lstlisting}
namespace System {

  void vymazKonzoli() { ... }

  void ukonciProgram() { ... }
}
\end{lstlisting}
\end{yesblock}

\begin{yesblock}
\begin{lstlisting}
System::vymazKonzoli();
System::ukonciProgram();
\end{lstlisting}
\end{yesblock}
\end{frame}


\begin{frame}[fragile]
\frametitle{Přístup k prvkům jmenného prostoru zvenčí -- direktiva using}
\begin{yesblock}
\begin{lstlisting}
namespace System {

  void vymazKonzoli() { ... }

  void ukonciProgram() { ... }
}
\end{lstlisting}
\end{yesblock}

\begin{yesblock}
\begin{lstlisting}
using namespace System;

vymazKonzoli();
ukonciProgram();
\end{lstlisting}
\end{yesblock}
\end{frame}



\begin{frame}[fragile]
\frametitle{Přístup k prvkům jmenného prostoru zvenčí -- deklarace using}
\begin{yesblock}
\begin{lstlisting}
namespace System {

  void vymazKonzoli() { ... }

  void ukonciProgram() { ... }
}
\end{lstlisting}
\end{yesblock}

\begin{yesblock}
\begin{lstlisting}
using System::vymazKonzoli;

vymazKonzoli();
System::ukonciProgram(); // nelze "ukonciProgram()"!
\end{lstlisting}
\end{yesblock}
\end{frame}



\begin{frame}[fragile]
\begin{noblock}
\begin{itemize}
\item deklarace/direktiva \lstinline|using| by neměla být používána v hlavičkových souborech!!!
\begin{itemize}
\item každý, kdo provede \lstinline|#include| na daný soubor bude ovlivněn usingy
\begin{itemize}
\item pozor na řetězení závislostí (hrac.h $\rightarrow$ pohyblivy\_objekt.h $\rightarrow$ objekt.h)
\end{itemize}
\item může způsobit opět konflikty jmen
\end{itemize}
\end{itemize}
\end{noblock}
\end{frame}



\begin{frame}[fragile]
\frametitle{Otevřenost jmenných prostorů}
\begin{block}{}
\begin{itemize}
\item do jmenného prostoru jde vždy přidávat nové složky
\end{itemize}
\end{block}

\begin{yesblock}
\begin{lstlisting}[basicstyle=\small]
namespace Objekty {
  Kocka micka;
}

// ...

namespace Objekty {
  Kocka mourek;
}

Objekty::micka.mnoukni();
Objekty::mourek.mnoukni();
\end{lstlisting}
\end{yesblock}
\end{frame}


\begin{frame}[fragile]
\frametitle{Vnořování jmenných prostorů}
\begin{block}{}
\begin{itemize}
\item jmenné prostory lze libovolně vnořovat
\end{itemize}
\end{block}

\begin{yesblock}
\begin{lstlisting}
namespace Hra {
  namespace Objekty {
    namespace ZiveObjekty {
      struct NPC { ... };
    }
  }
}
\end{lstlisting}
\end{yesblock}

\begin{yesblock}
\begin{lstlisting}
Hra::Objekty::ZiveObjekty::NPC obchodnik{};
\end{lstlisting}
\end{yesblock}
\end{frame}




\begin{frame}[fragile]
\frametitle{Vnořování jmenných prostorů\ldots}
\begin{block}{}
\begin{itemize}
\item od C++17 podpora pro zkrácený zápis vnořených jm. p.
\end{itemize}
\end{block}

\begin{yesblock}
\begin{lstlisting}
namespace Hra::Objekty::ZiveObjekty {
  struct NPC { ... };
}
\end{lstlisting}
\end{yesblock}

\begin{yesblock}
\begin{lstlisting}
Hra::Objekty::ZiveObjekty::NPC obchodnik{};
\end{lstlisting}
\end{yesblock}
\end{frame}



\begin{frame}[fragile]
\begin{bonusblock}{Anonymní jmenné prostory}
\begin{itemize}
\item lze použít pro skrytí jeho složek jen pro daný soubor .cpp
\end{itemize}
\end{bonusblock}

\begin{bonusblock}{}
\begin{lstlisting}
namespace {
  int _anonymous = 123;
}

// dále ve stejném souboru
cout << _anonymous;
\end{lstlisting}
\end{bonusblock}
\end{frame}


