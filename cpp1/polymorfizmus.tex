\hkapitola{Polymorfizmus}

\begin{frame}[fragile]
\begin{bitemize}
\item jedna funkce/metoda může nabývat více různých podob
\item metoda \lstinline|pracuj()| z rozhraní \lstinline|IPracujici|
\begin{itemize}
\item sekačka -- seká
\item policista -- uděluje pokuty
\item pes -- štěká
\end{itemize}
\end{bitemize}

\begin{noteblock}{Způsoby volání metod v C++}
\begin{itemize}
\item časná vazba -- výchozí, kompilátor řeší v době překladu
\item pozdní vazba -- na vyžádání (\lstinline|virtual|), řeší se v době vykonávání programu
\item []
\item Java -- automaticky vždy uplatňuje pozdní vazbu! chování v C++ a C\# je tedy odlišné
\end{itemize}
\end{noteblock}
\end{frame}







\begin{frame}[fragile]
\frametitle{Časná vazba}

\begin{yesblock}
\begin{twocols}
\begin{lstlisting}
struct Zvire {
  void vydejZvuk() {
    cout << "err";
  }
};
\end{lstlisting}

\twocolssep

\begin{lstlisting}
struct Kralik : Zvire {
  void vydejZvuk() {
    cout << "Bzzzm!";
  }
};
\end{lstlisting}
\end{twocols}
\end{yesblock}

\begin{yesblock}
\begin{lstlisting}
Zvire z{};
z.vydejZvuk(); // err

Kralik k{};
k.vydejZvuk(); // Bzzzm!
\end{lstlisting}
\end{yesblock}
\end{frame}







\begin{frame}[fragile]
\frametitle{Časná vazba\ldots}

\begin{yesblock}
\begin{twocols}
\begin{lstlisting}
struct Zvire {
  void vydejZvuk() {
    cout << "err";
  }
};
\end{lstlisting}

\twocolssep

\begin{lstlisting}
struct Kralik : Zvire {
  void vydejZvuk() {
    cout << "Bzzzm!";
  }
};
\end{lstlisting}
\end{twocols}
\end{yesblock}

\begin{noblock}
\begin{lstlisting}
Zvire& refz = k;
refz.vydejZvuk(); // err

Zvire* ptrz = &k;
ptrz->vydejZvuk(); // err
\end{lstlisting}
\end{noblock}
\end{frame}






\begin{frame}[fragile]
\frametitle{Pozdní vazba}

\begin{yesblock}
\begin{twocols}
\begin{lstlisting}
struct Zvire {
  virtual void vydejZvuk() {
    cout << "err";
  }
};
\end{lstlisting}

\twocolssep

\begin{lstlisting}
struct Kralik : Zvire {
  void vydejZvuk() {
    cout << "Bzzzm!";
  }
};
\end{lstlisting}
\end{twocols}
\end{yesblock}

\begin{yesblock}
\begin{lstlisting}
Zvire& refz = k;
refz.vydejZvuk(); // Bzzzm!

Zvire* ptrz = &k;
ptrz->vydejZvuk(); // Bzzzm!
\end{lstlisting}
\end{yesblock}
\end{frame}






\begin{frame}[fragile]
\frametitle{Pozdní vazba\ldots}

\begin{bitemize}
\item metoda musí být označena \lstinline|virtual| v předkovi
\item reference a ukazatele poté za běhu dohledají správnou metodu
\item využívá se VMT (virtual method table), kompilátor ji automaticky vytváří a vkládá na ni ukazatel do každého objektu
\end{bitemize}
\end{frame}





\begin{frame}[fragile]
\frametitle{Pozdní vazba\ldots}


\begin{yesblock}
\begin{twocols}
\begin{lstlisting}[basicstyle=\scriptsize]
struct Zvire {
  virtual void vydejZvuk() {
    cout << "err";
  }
};
\end{lstlisting}

\twocolssep

\begin{lstlisting}[basicstyle=\scriptsize]
struct Kralik : Zvire {
  void vydejZvuk() {
    cout << "Bzzzm!";
  }
};
\end{lstlisting}
\end{twocols}
\end{yesblock}

\vskip -3ex
\begin{yesblock}
\begin{twocols}
\begin{lstlisting}[basicstyle=\scriptsize]
struct Zvire {
  virtual void vydejZvuk() {
    cout << "err";
  }
};
\end{lstlisting}

\twocolssep

\begin{lstlisting}[basicstyle=\scriptsize]
struct Kralik : Zvire {
  virtual void vydejZvuk() {
    cout << "Bzzzm!";
  }
};
\end{lstlisting}
\end{twocols}
\end{yesblock}

\vskip -3ex
\begin{yesblock}
\begin{twocols}
\begin{lstlisting}[basicstyle=\small]
struct Zvire {
  virtual void vydejZvuk() {
    cout << "err";
  }
};
\end{lstlisting}

\twocolssep

\begin{lstlisting}[basicstyle=\small]
struct Kralik : Zvire {
  virtual void vydejZvuk() override {
    cout << "Bzzzm!";
  }
};
\end{lstlisting}
\end{twocols}
\end{yesblock}

\end{frame}





\begin{frame}[fragile]
\frametitle{Pozdní vazba\ldots}

\begin{block}{}
Doporučené použití:
\begin{itemize}
\item předek -- \lstinline|virtual|
\begin{itemize}
\item povinné, jinak nebude fungovat!
\end{itemize}
\item potomek -- \lstinline|virtual + override|
\begin{itemize}
\item \lstinline|virtual| -- pro přehlednost, že se jedná o virtuální metodu
\item \lstinline|override|\cpp{11} -- zajistí kontrolu kompilátorem, že předpis odpovídá virtuální metodě z předka
\end{itemize}
\end{itemize}
\end{block}
\end{frame}


\begin{frame}[fragile]
\frametitle{Pozdní vazba\ldots}

\begin{block}{}
Destruktor:
\begin{itemize}
\item časná/pozdní vazba se uplatňuje stejným způsobem na destruktor!
\begin{itemize}
\item aby bylo zajištěno, že se zavolá správný destruktor, musí být v předkovi virtuální destruktor
\item \textbf{každá třída ze které se dědí by měla mít virtuální destruktor}
\end{itemize}
\end{itemize}
\end{block}
\end{frame}



\begin{frame}[fragile]
\frametitle{Čistě virtuální metody}

\begin{bitemize}
\item virtuální metoda (\lstinline|virtual|) -- uplatnění pozdní vazby
\item čistě virtuální metoda (abstraktní, \lstinline|virtual ... = 0;|) -- pozdní vazba + chybí tělo (nutno doplnit v potomkovi)
\begin{itemize}
\item nelze vytvářet objekty od třídy obsahující čistě virtuální metody!
\end{itemize}
\end{bitemize}


\begin{yesblock}
\begin{lstlisting}
struct Zvire {
  virtual void vydejZvuk() = 0; // čistě virtuální / pure virtual
};

struct Kralik : Zvire {
  virtual void vydejZvuk() override {
    cout << "Bzzzm!";
  }
};

\end{lstlisting}
\end{yesblock}
\end{frame}





\begin{frame}[fragile]
\frametitle{Čistě virtuální metody\ldots}


\begin{noblock}
\begin{lstlisting}
struct Zvire {
  virtual void vydejZvuk() = 0; // čistě virtuální / pure virtual
};

Zvire z{}; // nelze - Zvire je abstraktní třída
\end{lstlisting}
\end{noblock}

\begin{bonusblock}{}
\begin{itemize}
\item čistě virtuální metoda může mít i tělo (definici)
\begin{itemize}
\item stále je pak čistě virtuální (abstraktní)
\item tělo lze explicitně volat z potomka \lstinline|Predek::cisteVirtualniMetoda()|
\end{itemize}
\end{itemize}
\end{bonusblock}
\end{frame}





\begin{frame}[fragile]
\frametitle{Rozhraní}

\begin{bitemize}
\item čistě virtuální metody lze použít pro realizaci typu \uv{rozhraní}
\begin{itemize}
\item nezapomínejte na virtuální destruktor!
\end{itemize}
\end{bitemize}


\begin{yesblock}
\begin{lstlisting}
struct IZvire {

  virtual ~IZvire() { }

  virtual void vydejZvuk() = 0;
  virtual void nakrm(Potravina* potravina) = 0;
  virtual void zautoc(IZvire* zvire) = 0;
};
\end{lstlisting}
\end{yesblock}
\end{frame}










