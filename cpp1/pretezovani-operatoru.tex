\nezkouskove
\hkapitola{Přetěžování operátorů}


\begin{frame}[fragile]
\frametitle{Přetěžování operátorů}
\begin{block}{}
\begin{itemize}
\item lze přetížit operátory nad uživatelskými objektovými typy
\begin{itemize}
\item použití daného operátoru nad objektem vyvolá uživatelem definovanou metodu/funkci
\item lze přetížit pouze vybrané existující operátory
\end{itemize}
\item je to pouze syntax-sugar pro volání funkcí/metod
\begin{itemize}
\item operátor + lze volat jako \lstinline|a + b|, ale i jako metodu \lstinline|a.operator+(b)|
\item v kombinaci s STL slouží k definování některých základních chování objektů
\end{itemize}
\end{itemize}
\end{block}
\end{frame}

\begin{frame}[fragile]
\begin{block}{Přetěžování operátorů} 
\begin{itemize}
\item definování, jak se bude daný operátor chovat pro náš objektový typ
\item 4 skupiny operátorů:
\begin{itemize}
\item nelze přetížit
\item lze metodou
\item lze metodou i funkcí
\item lze funkcí/statickou metodou
\end{itemize}
\end{itemize}
\end{block}
\end{frame}



\begin{frame}[fragile]
\begin{alertblock}{{\NO}~Nelze přetížit}

\begin{itemize}
\item tečkové operátory -- ., .*, ::
\item ternární operátor -- ? :
\item přetypovací operátory -- static\_cast, \ldots
\item další -- typeof, sizeof
\end{itemize}
\end{alertblock}
\end{frame}







\begin{frame}[fragile]
\frametitle{Přetížitelné operátory}
\begin{block}{Metodou} 
\begin{center}
\lstinline|= -> ->* [ ] ()| \lstinline|op. přetypování (typ)|
\end{center}
\end{block}

\begin{block}{Metodou i funkcí} 
\begin{center}
\lstinline|+ - * / ++ -- > < >= <= == >> << & || \lstinline|~ += -= *= /=| $\ldots$
\end{center}
\end{block}

\begin{block}{Funkcí/statickou metodou} 
\begin{center}
\lstinline|new new[ ] delete delete[ ]|
\end{center}
\end{block}

\end{frame}



\begin{frame}[t,fragile]
\begin{noteblock}{} 
\begin{lstlisting}[basicstyle=\small]
návratovýTyp operator@([parametryOperátoru])
\end{lstlisting}
\end{noteblock}
\vskip -1ex
\begin{block}{}
uvažujte výraz:
\begin{lstlisting}
int a = 5, b = 10;
int c = a + b;
\end{lstlisting}
\end{block}
\begin{block}{}
platí:
\begin{itemize}
\item operátor + je binární operátor
\item operandy jsou \emph{a}, \emph{b}
\item změní se proměnná \emph{a}? NE $\rightarrow$ konstanta
\item změní se proměnná \emph{b}? NE $\rightarrow$ konstanta
\end{itemize}
\end{block}
\end{frame}

\begin{frame}[fragile]
\begin{noteblock}{} 
\begin{lstlisting}[basicstyle=\small]
návratovýTyp operator@([parametryOperátoru])
\end{lstlisting}
\end{noteblock}
\vskip -1ex
\begin{block}{}
uvažujte výraz:
\begin{lstlisting}
int a = 5, b = 10;
int c = a + b;
\end{lstlisting}
\end{block}

\begin{yesblock}
\begin{lstlisting}[basicstyle=\small]
int operator+(int a, int b) {
	return {a->state + b->state};
}
\end{lstlisting}
\end{yesblock}
\begin{yesblock}
\begin{lstlisting}[basicstyle=\small]
struct int {
	int operator+(int b) {
		return {this->state + b->state};
	}
};
\end{lstlisting}
\end{yesblock}
\end{frame}



\begin{frame}[fragile]
\begin{noteblock}{} 
\begin{lstlisting}[basicstyle=\small]
návratovýTyp operator@([parametryOperátoru])
\end{lstlisting}
\end{noteblock}
\vskip -1ex
\begin{block}{}
uvažujte výraz:
\begin{lstlisting}
int a = 5, b = 10;
int c = a + b;
\end{lstlisting}
\end{block}

\begin{yesblock}
\begin{lstlisting}[basicstyle=\small]
int operator+(const int& a, const int& b) {
	return {a->state + b->state};
}
\end{lstlisting}
\end{yesblock}
\begin{yesblock}
\begin{lstlisting}[basicstyle=\small]
struct int {
	int operator+(const int& b) const {
		return {this->state + b->state};
	}
};
\end{lstlisting}
\end{yesblock}
\end{frame}



\begin{frame}[fragile]
\frametitle{Reálný příklad -- komplexní číslo}
\begin{yesblock} 
\begin{lstlisting}
Komplex a = 5, b = 10;
Komplex c = a + b;

Komplex operator+(const Komplex& a, const Komplex& b) {
	return {a->re + b->re, a->im + b->im};
}

struct Komplex {
	Komplex operator+(const Komplex& b) const {
		return {this->re + b->re, this->im + b->im};
	}
};
\end{lstlisting}
\end{yesblock}
\end{frame}




\begin{frame}[fragile]
\frametitle{Kombinace funkce + definice friend}
\begin{yesblock} 
\begin{lstlisting}[basicstyle=\small]
struct Komplex {
	friend Komplex operator+(const Komplex& a, const Komplex& b) {
		return {a->re + b->re, a->im + b->im};
	}
};
\end{lstlisting}
\end{yesblock}
\begin{block}{}
friend + definice v těle způsobí:
\begin{itemize}
\item nejedná se o metodu, ale o funkci!
\item má přístup k privátním a chráněným složkám
\end{itemize}
\end{block}
\end{frame}







\begin{frame}[fragile]
\frametitle{Operátorové metody -- konstantnost}
\begin{block}{}
\begin{itemize}
\item konstatní metoda operátoru může být vyvolána i nad konstatním objektem!
\item nekonstatní metoda operátoru může být vyvolána pouze nad nekonstatním objektem!
\end{itemize}
\end{block}

\begin{block}{}
Většina operátorů nemění stav objektu
\begin{itemize}
\item \lstinline|+ - * / |
\item \lstinline|<< >> & | | \~{}
\item \lstinline|< > >= <= == !=|
\end{itemize}
\end{block}
\end{frame}










\begin{frame}[fragile]
\frametitle{Operátorové metody -- konstantnost\ldots}
\begin{block}{} 
Operátory (potenciálně) měnící stav objektu
\begin{itemize}
\item ++, -\/-
\item =, +=, -=, *=, /=, $\ldots$
\item $[$\,$]$ -- pokud vrací referenci/ukazatel na stav objektu
\item () -- pokud mění stav objektu
\end{itemize}
\end{block}
\end{frame}









\begin{frame}[fragile]
\frametitle{Typ návratové hodnoty}
\begin{block}{}
\begin{itemize}
\item operátory porovnání -- \lstinline|bool|
\item ostatní -- typ objektu nebo jiný libovolný typ
\end{itemize}
\end{block}

\begin{block}{} 
Proč vracet staticky alokovaný objekt? Možnosti:
\begin{itemize}
\item statický objekt -- dochází ke kopírování, vytvoření dočasného objektu a jeho následné destrukci (pokud nedojde k optimalizaci od kompilátoru)
\item reference
\begin{itemize}
\item ale kde ji vezmu?
\item obyčejná proměnná zanikne s koncem bloku -- nelze
\item statická (nebo globální) proměnná bude existovat dál, ale nebude to fungovat ve vícevláknových aplikacích -- nelze
\item dynamicky alokovaná proměnná -- ztratím ukazatel a informaci o~alokaci a do konce programu bude paměť zabraná -- nelze
\end{itemize}
\item ukazatel (dyn. alokovaná paměť) -- změna sémantiky operátoru -- nelze
\end{itemize}
\end{block}
\end{frame}






\begin{frame}[fragile]
\frametitle{Využití přetížených operátorů}
\begin{block}{}
\begin{itemize}
\item \lstinline|< > <= >= == !=|
\begin{itemize}
\item řazení hodnot, asociativní kontejnery, jednoduché porovnávání
\end{itemize}

\item \lstinline|[ ]|
\begin{itemize}
\item přístup k prvkům kontejneru
\end{itemize}

\item \lstinline|()|
\begin{itemize}
\item funkční objekty (objekty určující chování algoritmů)
\end{itemize}

\item \lstinline|++ -- * -> ->*|
\begin{itemize}
\item realizace iterátoru
\end{itemize}

\item \lstinline|++ -- + - * /|
\begin{itemize}
\item matematické struktury (matice, komplexní čísla, ntice, \ldots), měnitelné hodnoty
\end{itemize}


\end{itemize}
\end{block}
\end{frame}






\begin{frame}[fragile]
\begin{center}
další příklady k nalezení např. v knize Thinking in C++ \\ ~ \\
\url{http://www.drbio.cornell.edu/pl47/programming/TICPP-2nd-ed-Vol-one-html/Chapter12.html} \\
~ \\
další zdroje \\
~ \\
\url{http://en.cppreference.com/w/cpp/language/operators} \\
~ \\
\end{center}
\end{frame}







\zkouskove