\hkapitola{Rozdíly mezi C a C++}

\begin{frame}[fragile]
\frametitle{Základní rozdíly}
\begin{bitemize}
\item znakové literály jsou typu \lstinline|char|
\item proměnné je možné deklarovat \uv{kdekoliv}
\item silnější typová kontrola
\begin{itemize}
\item \lstinline|void func();| -- v C++ funkce bez parametrů
\item \lstinline|void* prom| -- nelze přiřadit bez konverze typu
\end{itemize}
\item ukazatel nikam není \lstinline|NULL|, ale \lstinline|nullptr|\cpp{11} (rvalue, nelze přiřadit do neukazatele)
\item funkce je nutné deklarovat před jejich zavoláním
\item nový logický typ \lstinline|bool| (\lstinline|true| nebo \lstinline|false|)
\item při použití datového typu struktury se uvádí pouze název struktury (nikoliv \lstinline|struct NazevStruktury| jako v C)
\end{bitemize}
\end{frame}

\begin{frame}[fragile]
\frametitle{Dynamická alokace paměti}

\begin{oldblock}
\begin{itemize}
\item funkce \lstinline|malloc|, \lstinline|free|, \lstinline|calloc| a \lstinline|realloc| jsou nahrazeny
\begin{itemize}
\item neumí pracovat s objekty
\end{itemize}
\end{itemize}
\end{oldblock}
\begin{bitemize}
\item nové operátory \lstinline|new|, \lstinline|new[]|, \lstinline|delete| a \lstinline|delete[]|
\begin{itemize}
\item umí pracovat s objekty
\item realokaci je nutné řešit ručně (\lstinline|new, memcpy(), delete)|
\end{itemize}
\end{bitemize}
\end{frame}


\kapitola{Reference}

\begin{frame}[fragile]
\frametitle{L-value reference}

\begin{bitemize}
\item L-value reference představují ukazatele na \uv{pojmenované} proměnné
\begin{itemize}
\item ne na dočasné objekty
\end{itemize}
\item kompilátor referencování a dereferencování řeší za nás (odpadá starost s operátory * a \&)
\item po vytvoření reference nejde změnit kam ukazuje
\item veškerá manipulace je pak automaticky přeposlána na referencovaný objekt
\end{bitemize}

\begin{yesblock}
\begin{lstlisting}[basicstyle=\small]
int alfa = 100;
int& refAlfa = alfa; // není potřeba &

refAlfa++; // alfa = 101
cout << (&alfa == &refAlfa); // == true
refAlfa = 0; // alfa = 0;
\end{lstlisting}
\end{yesblock}
\end{frame}


\begin{frame}[fragile]
\frametitle{L-value reference\ldots}
\begin{bitemize}
\item lze je použít jako
\begin{itemize}
\item lokální proměnné
\item atributy tříd
\item parametry metod
\item návratové hodnoty
\end{itemize}
\end{bitemize}

\begin{twocols}
\begin{yesblock}
\begin{lstlisting}[basicstyle=\scriptsize]
// předávání hodnotou
// = kopie objektu
void nakrmKocku(Kocka k) {
  k.nakrmena(RYBA);
  // nakrmili jsme kopii micky, micka zatím chcípe hlady  
}

Kocka micka;
nakrmKocku(micka);
\end{lstlisting}
\end{yesblock}

\twocolssep

\begin{yesblock}
\begin{lstlisting}[basicstyle=\scriptsize]
// předávání odkazem
// = stejný objekt
void nakrmKocku(Kocka& k) {
  k.nakrmena(RYBA);
  // nakrmili jsme micku
}

Kocka micka;
nakrmKocku(micka);
\end{lstlisting}
\end{yesblock}

\end{twocols}
\end{frame}






\begin{frame}[fragile]
\frametitle{L-value reference\ldots}

\begin{bitemize}
\item reference je možné vytvořit na libovolný datový typ (i ukazatele)
\begin{itemize}
\item ale nelze vytvořit ukazatel na referenci
\end{itemize}
\end{bitemize}

\begin{yesblock}
\begin{lstlisting}
int i = 10;
int* ui = &i;

int*& refui = ui; // ok
//int&* urefi; // ne - ukazatel na referenci

// následující dva řádky jsou totožné
*ui = 100;
*refui = 100;
// i == *ui == *refui == 100
\end{lstlisting}
\end{yesblock}
\end{frame}





\begin{frame}[fragile]
\frametitle{L-value reference\ldots}

\begin{bitemize}
\item konstatní reference
\begin{itemize}
\item nelze přiřadit jinou hodnotu (pomocí operátoru =)
\item s objekty lze omezeně manipulovat
\item \lstinline|const type& ref;|
\end{itemize}
\item referenční funkce
\begin{itemize}
\item funkce, která vrací referenci
\item za return musí být l-hodnota
\end{itemize}
\end{bitemize}

\begin{yesblock}
\begin{lstlisting}[basicstyle=\scriptsize]
Kocka& mnoukej(Kocka& k) {
  cout << "mnau ";
  k.zamnoukano();
  return k;
}

Kocka micka;
mnoukej(mnoukej(mnoukej(micka)));
// micka 3x zamňoukala

\end{lstlisting}
\end{yesblock}
\end{frame}


\begin{frame}[fragile]
%\frametitle{R-value reference}

\begin{bonusblock}{R-value reference\cpp{11}}
\begin{itemize}
\item představuje optimalizaci využití dočasných objektů
\item zapisuje se jako: \lstinline|datovyTyp&& prom|
\item přidává move sémantiku (konstruktor, operátor =)
\begin{itemize}
\item \lstinline|std::move()|
\item \lstinline|std::forward()|
\end{itemize}
\end{itemize}
\end{bonusblock}
\end{frame}


\kapitola{Ostatní\ldots}


\begin{frame}[fragile]
\frametitle{Standardní vstupy a výstupy (obrazovka, soubor, paměť)}

\begin{bitemize}
\item pro vstup a výstup vytvořeny objekty -- proudy
\begin{itemize}
\item hlavičkový soubor \lstinline|iostream|
\item výstup na obrazovku pomocí \lstinline|std::cout << "data" << 123 << prom << ...;|
\item vstup z klávesnice pomocí \lstinline|cin >> prom1 >> prom2 >> prom3 >> ...;|
\end{itemize}
\end{bitemize}

\begin{yesblock}
\begin{lstlisting}[basicstyle=\small]
#include <iostream>

void main() {
  std::cout << "Zadej cislo: ";
  int cislo;
  std::cin >> cislo;
  std::cout << "Zadal jsi " << cislo << std::endl;
}
\end{lstlisting}
\end{yesblock}
\end{frame}



\begin{frame}[fragile]
\frametitle{Prostory jmen (namespace)}
\begin{bitemize}
\item řeší konflikty jmen
\item podobné balíčkům z Javy (ale vše uvnitř je veřejné)
\item přístup k prvkům pomocí \lstinline|::| nebo zpřístupnění pomocí operátoru \lstinline|using|
\end{bitemize}

\begin{yesblock}
\begin{lstlisting}
namespace MojeKnihovna {
  void knihovniFunkce() { ... }
}

void main() {
  // knihovniFunkce(); // NE -> neexistuje
  MojeKnihovna::knihovniFunkce();
}
\end{lstlisting}
\end{yesblock}
\end{frame}



\begin{frame}[fragile]
\frametitle{Prostory jmen (namespace)\ldots}
\begin{bitemize}
\item \lstinline|using| se nepoužívá v hlavičkových souborech!
\end{bitemize}

\begin{yesblock}
\begin{lstlisting}
namespace MojeKnihovna {
  void knihovniFunkce() { ... }
}

// ... 

using namespace MojeKnihovna;

void main() {
  knihovniFunkce();
}
\end{lstlisting}
\end{yesblock}
\end{frame}





\begin{frame}[fragile]
\frametitle{Výjimky}

\begin{bitemize}
\item ošetřování chyb pomocí výjimek
\begin{itemize}
\item blok \lstinline|try|
\item blok \lstinline|catch|
\item příkaz \lstinline|throw|
\item modifikátor \lstinline|noexcept|\cpp{11}
\end{itemize}
\end{bitemize}

\begin{yesblock}
\begin{lstlisting}[basicstyle=\scriptsize]
try {
  pripravVlakna();
  provedVypocet();
  uklidVlakna();
} catch (ThreadException& threadException) {
  cerr << "Vlakna spadla" << endl;
} catch (CalculationException& calculationException) {
  cerr << "Vypocet spadl" << endl;
}
\end{lstlisting}
\end{yesblock}
\end{frame}



\begin{frame}[fragile]
\frametitle{Šablony (generické datové typy a funkce)}

\begin{bitemize}
\item genericita na steroidech
\item \lstinline|template<typename T> ...|
\item šablony
\begin{itemize}
\item funkcí
\item metod
\item datových typů
\item šablon
\end{itemize}
\item specializace šablon
\begin{itemize}
\item parciální
\item explicitní
\end{itemize}
\end{bitemize}
\vskip -1ex
\begin{yesblock}
\begin{lstlisting}[basicstyle=\scriptsize]
template<typename T, int Size>
struct Array {
  T& get(int index) { return _array[index]; }
  T _array[Size];
}
\end{lstlisting}
\end{yesblock}
\end{frame}





\begin{frame}[fragile]
\frametitle{RTTI -- run time type information}

\begin{bitemize}
\item obdoba reflexe z Javy
\item podstatně jednodušší 
\begin{itemize}
\item v podstatě umí jen identifikovat typ a vypsat jeho jméno (tvar není standardizován)
\end{itemize}
\item definuje dva operátory
\begin{itemize}
\item \lstinline|typeid| -- vrací strukturu \lstinline|type_info| s informacemi o typu
\item \lstinline|dynamic_cast| -- chytré přetypování v rámci hierarchie dědičnosti
\end{itemize}
\item využívá přítomnost VMT v objektech (vyžaduje virtuální metodu, viz později polymorfizmus)
\end{bitemize}
\end{frame}







\begin{frame}[fragile]
\frametitle{Nové operátory pro přetypování}

\begin{bitemize}
\item \lstinline|dynamic_cast|
\begin{itemize}
\item využítá RTTI (Run Time Type Information)
\item užitečný, bezpečný, přetypování z předka na potomka (i naopak)
\end{itemize}
\end{bitemize}
\vskip -2ex
\begin{bonusblock}{}
\begin{itemize}
\item \lstinline|static_cast|
\begin{itemize}
\item běžné přetypování
\end{itemize}
\item \lstinline|const_cast|
\begin{itemize}
\item pro odstranění \lstinline|const| nebo \lstinline|volatile| modifikátorů
\end{itemize}
\item \lstinline|reinterpret_cast|
\begin{itemize}
\item nestandardní přetypování
\end{itemize}
\end{itemize}
\end{bonusblock}
\vskip -1ex
\begin{yesblock}
\begin{lstlisting}[basicstyle=\scriptsize]
Object* obj = new Kocka("Micka");
Kocka* kocka = dynamic_cast<Kocka*>(obj);
if (kocka != nullptr)
  kocka->mnoukej();
\end{lstlisting}
\end{yesblock}
\end{frame}







\nezkouskove


\kapitola{C++11, C++14, C++17}

\begin{frame}[fragile]
\frametitle{constexpr\cpp{11}}

\begin{bonusblock}{}
\begin{itemize}
\item definuje, že proměnná nebo funkce obsahuje/vrací konstatní výraz a může být využit na místě, kde se očekává konstatní výraz (definice statického pole, \ldots)
\end{itemize}
\end{bonusblock}

\begin{yesblock}
\begin{lstlisting}
constexpr int pocetBajtuVObrazu(int sirka, int vyska, int bpp) {
  return sirka * vyska * (bpp / 8);
}

unsigned char obrazoveBity[pocetBajduVObrazu(800, 600, 32)];
\end{lstlisting}
\end{yesblock}
\end{frame}



\begin{frame}[fragile]
\frametitle{auto\cpp{11}, decltype\cpp{11}}

\begin{bonusblock}{}
\begin{itemize}
\item \lstinline|auto|\cpp{11} lze použít jako zástupný symbol pro libovolný datový typ
\begin{itemize}
\item konkrétní datový typ dohledá kompilátor v době kompilace
\item stále se jedná o silně typovaný jazyk, nelze pak přiřadit jiný typ do takové proměnné
\end{itemize}
\item \lstinline|decltype|\cpp{11} slouží jako zástupný symbol datového typu definovaného podle výsledku výrazu
\end{itemize}
\end{bonusblock}

\begin{yesblock}
\begin{lstlisting}
std::map<KeyObject<Person>, Person> map;

auto iterator = map.rbegin();
// auto = std::map<KeyObject<Person>, Person>::reverse_iterator

auto v = 1; // v = int
decltype(v) w = v; // w = int
\end{lstlisting}
\end{yesblock}
\end{frame}





\begin{frame}[fragile]
\frametitle{lambda výrazy\cpp{11}}

\begin{bonusblock}{}
\begin{itemize}
\item lambda výrazy (anonymní funkce) představují možnost zapsat funkci prakticky kamkoliv do kódu
\end{itemize}
\end{bonusblock}

\begin{yesblock}
\begin{lstlisting}
auto function = [ ] (int p1, int p2) { return p1 + p2; }

function(10, 20); // = 30
\end{lstlisting}
\end{yesblock}
\end{frame}






\begin{frame}[fragile]
\frametitle{for-each\cpp{11}}
\begin{bonusblock}{}
\begin{itemize}
\item \lstinline|for (definiceProměnné : kontejner) { ... }|
\item umožňuje jednoduše procházet kontejnery/kolekce
\item funguje nad statickým polem a objekty, které definují metody \lstinline|begin(), end()|
\end{itemize}
\end{bonusblock}

\begin{yesblock}
\begin{lstlisting}
std::vector<int> container;
for (int value : container) {
  ...
}
\end{lstlisting}
\end{yesblock}
\end{frame}



\begin{frame}[fragile]
\frametitle{for-each\cpp{11}}
\begin{noblock}
\begin{lstlisting}
// for-each podle C++/CLI použitelné v MSVC
// není ve standardu C++ -> nepoužívat!
for each (int value in container) {
  ...
}
\end{lstlisting}
\end{noblock}

\begin{yesblock}
\begin{lstlisting}
// neplést for-each cyklus s for_each algoritmem!
for_each(
  container.begin(), 
  container.end(), 
  [](int value) { ... });
\end{lstlisting}
\end{yesblock}
\end{frame}



\begin{frame}[fragile]
\frametitle{for-each\cpp{11}}
\begin{block}{}
\begin{itemize}
\item lze kombinovat s \lstinline|auto|
\end{itemize}
\end{block}

\begin{yesblock}
\begin{lstlisting}
std::vector<Game::Map::Object<GUID, WorldType>> container;

for (auto& object : container) {
  ... 
}
\end{lstlisting}
\end{yesblock}
\end{frame}





\begin{frame}[fragile]
\frametitle{tuple (N-tice)\cpp{11}}

\begin{bonusblock}{}
\begin{itemize}
\item slouží pro předávání N hodnot bez nutnosti tvořit třídu
\item využívá šablon s proměnným počtem parametrů
\end{itemize}
\end{bonusblock}

\begin{yesblock}
\begin{lstlisting}[basicstyle=\small]
std::tuple<double, char, std::string> get_student(int id)
{
  if (id == 0) 
    return std::make_tuple(3.8, 'A', "Lisa Simpson");  
  if (id == 1) 
    return std::make_tuple(2.9, 'C', "Milhouse Van Houten");  
  if (id == 2) 
    return std::make_tuple(1.7, 'D', "Ralph Wiggum");
  
  throw std::invalid_argument("id");
}
\end{lstlisting}
\end{yesblock}
\end{frame}


\begin{frame}[fragile]
\frametitle{tuple (N-tice)\cpp{11}}

\begin{yesblock}
\begin{lstlisting}[basicstyle=\small]
int main()
{
    auto student0 = get_student(0);
    std::cout << "ID: 0, "
              << "GPA: " << std::get<0>(student0) << ", "
              << "grade: " << std::get<1>(student0) << ", "
              << "name: " << std::get<2>(student0) << '\n';
 
    double gpa1;
    char grade1;
    std::string name1;
    std::tie(gpa1, grade1, name1) = get_student(1);
    std::cout << "ID: 1, "
              << "GPA: " << gpa1 << ", "
              << "grade: " << grade1 << ", "
              << "name: " << name1 << '\n';
}
\end{lstlisting}
\end{yesblock}
\end{frame}









\begin{frame}[fragile]
\frametitle{Další novinky z C++11, 14}
\begin{bonusblock}{}
\begin{itemize}
\item silně typované výčty -- \lstinline|enum class NazevEnumu { ... }|
\begin{itemize}
\item jednotlivé výčtové konstanty nejsou dostupné z globálního prostoru, ale z \lstinline|NazevEnumu::konstanta|
\end{itemize}
\item nové řetězcové literály -- \lstinline|u8"Řetězec v UTF-8"|, \lstinline|u"UTF-16", U"UTF-32"|
\item RAW řetězcové literály -- {\ttfamily{}R}{\ttfamily\color[rgb]{0.627,0.126,0.941}"(retezec~"~s~'~divnoznaky)\dq}
\item uživatelské řetězcové literály
\item \lstinline|static_assert|
\item atributy -- \lstinline|[[atribut]]|
\item \ldots
\end{itemize}
\end{bonusblock}
\end{frame}






\begin{frame}[fragile]
\frametitle{Structured bindings\cpp{17}}

\begin{bonusblock}{}
\begin{itemize}
\item umožňuje jednoduše rozdělit složitý typ na elementární proměnné 
\item funguje na tuple interface, statické pole nebo pokud typ má pouze veřejné nestatické složky
\end{itemize}
\end{bonusblock}

\begin{yesblock}
\begin{lstlisting}[basicstyle=\small]
// C++14
double gpa1;
char grade1;
std::string name1;
std::tie(gpa1, grade1, name1) = get_student(1);
...

// C++17
auto& [gpa2, grade2, name2] = get_student(2);
...
\end{lstlisting}
\end{yesblock}
\end{frame}












\begin{frame}[fragile]
\frametitle{Structured bindings\cpp{17}}

\begin{bonusblock}{}
\begin{itemize}
\item lze kombinovat i s for-each
\end{itemize}
\end{bonusblock}

\begin{yesblock}
\begin{lstlisting}[basicstyle=\small]
std::map<System::GUID, Game::Map::Object<GUID, WorldType>> container;

for (const auto& [key, value] : container) {
  ...
}
\end{lstlisting}
\end{yesblock}
\end{frame}








\begin{frame}[fragile]
\frametitle{Další novinky z C++17}
\begin{bonusblock}{}
\begin{itemize}
\item \lstinline|constexpr if|
\item init-statement pro if/switch
\item automatické odvození typů pro šablony objektových typů
\item fold expressions (... v šablonách s proměnným počtem parametrů)
\item \ldots
\end{itemize}
\end{bonusblock}

\begin{noblock}
\begin{itemize}
\item C++17 není prozatím široce podporováno (zejména v MSVC)
\item Většina novinek bude dostupná nejdříve od MSVC 2017!
\item Viz \url{https://blogs.msdn.microsoft.com/vcblog/2017/05/10/c17-features-in-vs-2017-3/}
\end{itemize}
\end{noblock}
\end{frame}




\zkouskove