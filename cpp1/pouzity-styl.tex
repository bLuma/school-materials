\hkapitola{Použitý styl v přednáškách}


\begin{frame}[fragile]
%\frametitle{Styly}
\begin{block}{Základní informace} 
Lorem ipsum dolor sit amet, consectetur adipiscing elit.
\end{block}

\begin{exampleblock}{Příklad}
Nullam mattis efficitur aliquam.
\end{exampleblock}

\begin{alertblock}{Chyba / příklad s chybou / nekorektní použití} 
Sed aliquam iaculis massa, vel tincidunt lacus tincidunt eget.
\end{alertblock}

\begin{noteblock}{Poznámka (teorie, vhodné k zapamatování)}
Proin porta urna ut ipsum ornare, a ultricies elit dictum.
\end{noteblock}

\begin{deprecatedblock}{Deprecated (staré, už nepoužívané)} 
Pellentesque habitant morbi tristique senectus et netus et malesuada fames. 
\end{deprecatedblock}

\begin{bonusblock}{Bonus (nepotřebujete ke zkoušce, ale souvisí s tématem)}
Sed imperdiet pharetra est, sed ullamcorper neque. 
\end{bonusblock}
\end{frame}


\begin{frame}[fragile]
\frametitle{Ukázka}
\begin{block}{Ukazatele a dynamická paměť}
Ukázky korektní a nekorektní práce s ukazateli a dynamicky alokovanou pamětí:
\end{block}

\begin{twocols}

\begin{exampleblock}{\YES\,Good}
\begin{lstlisting}
int* pointer = nullptr;
pointer = new int;
*pointer = 123;
\end{lstlisting}
\end{exampleblock}
\twocolssep
\begin{alertblock}{\NO\,Bad}
\begin{lstlisting}
int* pointer = 0xdeadbeef;
*pointer = 123;
\end{lstlisting}
\end{alertblock}

\end{twocols}


\begin{deprecatedblock}{\WARNING\,deprecated}
\begin{lstlisting}
int* pointer = (int*)malloc(sizeof(int));
*pointer = 123;
\end{lstlisting}
\end{deprecatedblock}

\end{frame}



\begin{frame}[fragile]
\frametitle{Definice kódu}

\begin{noteblock}{}
\begin{lstlisting}[basicstyle=\small]
struct|class nazevDatovehoTypu [final] [dědičnost] {
  [složky - atributy, metody, vnořené typy]...
} [objekty];

[dědičnost]:
  : [viditelnost] [virtual] předek1, ...
\end{lstlisting}
\end{noteblock}

\begin{block}{Definuje:}
\begin{itemize}
\item Začínáme klíčovým slovem \lstinline|struct| nebo \lstinline|class|
\item dále uvedeme název datového typu (\lstinline|Pes|, \lstinline|Kocka|, \lstinline|Student|, \ldots)
\item dále může být (ale nemusí) klíčové slovo \lstinline|final|
\item dále může být definováno dědění z předků
\item[]
\item Uvnitř třídy je možné definovat větší množství složek (\ldots)
\end{itemize}

\end{block}
\end{frame}








\nezkouskove
\begin{frame}[fragile]
\begin{center}
\large Slidy označené fialovými pruhy obsahují téma, které není vyžadováno u zkoušky a zápočtu.
\end{center}
\end{frame}
\zkouskove