\hkapitola{Objektové typy}

\begin{frame}[fragile]
\frametitle{Objektové typy}
\begin{bitemize}
\item \lstinline|class| -- plnohodnotný objektový typ
\item \lstinline|struct| -- plnohodnotný objektový typ
\item \lstinline|union| -- omezený objektový typ (neumí dědičnost, polymorfizmus)
\end{bitemize}

\begin{block}{}
Základní vlastnosti:
\begin{itemize}
\item vícenásobná dědičnost
\begin{itemize}
\item neexistují rozhraní
\end{itemize}

\item polymorfizmus
\item H x CPP soubor
\begin{itemize}
\item H -- deklarace (celý typ, atributy, prototypy metod)
\item CPP -- definice (pouze těla metod, statické atributy)
\end{itemize}
\end{itemize}
\end{block}
\end{frame}


\begin{frame}[fragile]
\frametitle{struct x class}
\begin{bitemize}
\item \lstinline|struct| i \lstinline|class| jsou zaměnitelné
\item \lstinline|struct| může dědit z \lstinline|class| a naopak
\item liší se pouze ve výchozí úrovni viditelnosti složek (\lstinline|public| x \lstinline|private|)
\end{bitemize}
\end{frame}


\kapitola{Základní syntaxe objektových typů}

\begin{frame}[fragile]
\frametitle{Definice struct a class}
\begin{noteblock}{}
\begin{lstlisting}
struct|class nazevDatovehoTypu [final] [dědičnost] {
  [složky - atributy, metody, vnořené typy]
} [objekty];
\end{lstlisting}
\end{noteblock}


\begin{twocols}
\begin{yesblock}
\begin{lstlisting}
struct Pes {
  ...
};
\end{lstlisting}
\end{yesblock}

\twocolssep

\begin{yesblock}
\begin{lstlisting}
class Pes {
  ...
};
\end{lstlisting}
\end{yesblock}
\end{twocols}

\begin{block}{dopředná deklarace}
\begin{itemize}
\item pokud potřebujeme pracovat s typem, ale ne s jeho vnitřkem
\item \lstinline!struct|class nazevDatovehoTypu;!
\end{itemize}
\end{block}

\end{frame}


\begin{frame}[fragile]
\begin{noteblock}{}
\begin{lstlisting}[basicstyle=\scriptsize]
struct|class nazevDatovehoTypu [final] [dědičnost] {
  [složky - atributy, metody, vnořené typy]
} [objekty];
\end{lstlisting}
\end{noteblock}

\begin{bitemize}
\item{} [objekty] -- umožňuje vytvořit staticky alokované objekty od dané třídy v~globálním prostoru
\end{bitemize}

\begin{yesblock}
\begin{lstlisting}
struct Pes {
  ...
} punta, rafan;

void main() { 
  punta.stekej(); 
  rafan.stekej(); 
}
\end{lstlisting}
\end{yesblock}
\end{frame}


\begin{frame}[fragile]
\begin{noteblock}{}
\begin{lstlisting}[basicstyle=\scriptsize]
struct|class nazevDatovehoTypu [final] [dědičnost] {
  [složky - atributy, metody, vnořené typy]
} [objekty];
\end{lstlisting}
\end{noteblock}

\begin{bitemize}
\item{} [\lstinline|final|]\cpp{11} -- zakazuje dědit z této třídy
\item{} [dědičnost] -- specifikace předků
\begin{itemize}
\item \lstinline|: [viditelnost] [virtual] předek1, ...|
\end{itemize}
\end{bitemize}

\begin{yesblock}
\begin{lstlisting}
struct Pes : public virtual Zvire, public IIdentifikovatelny, IKopirovatelny {
  ...
};
\end{lstlisting}
\end{yesblock}
\end{frame}



\begin{frame}[fragile]
\begin{noteblock}{}
\begin{lstlisting}[basicstyle=\scriptsize]
struct|class nazevDatovehoTypu [final] [dědičnost] {
  [složky - atributy, metody, vnořené typy]
} [objekty];
\end{lstlisting}
\end{noteblock}

\begin{bitemize}
\item{} [složky] -- dále dle typu atribut, metoda, vnořený typ
\end{bitemize}

\begin{yesblock}
\begin{lstlisting}
struct Pes {

  void stekej() { ... }

  string jmeno;

};
\end{lstlisting}
\end{yesblock}
\end{frame}

