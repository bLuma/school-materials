\nezkouskove
\section{Implementace dědičnosti v C++}

\begin{frame}[fragile]
\vskip 3cm
\begin{bonusblock}{}
\vskip 2ex
\Large \centering
Implementace dědičnosti v C++
\vskip 2ex
\end{bonusblock}
\vskip 1.5cm
\begin{center}
{\small Následující část přednášky obsahuje informace k bližšímu pochopení, jak interně funguje vícenásobná dědičnost a polymorfizmus. \\ Podrobně viz \url{ https://www.usenix.org/legacy/publications/compsystems/1989/fall_stroustrup.pdf}} \\ \textbf{Nejedná se o téma potřebné do zkoušky/zápočtu!}
\end{center}

\end{frame}


\begin{frame}[fragile]
\begin{bonusblock}{Objekt}
\begin{itemize}
\item objekt je spojitá část paměti, kde jsou uloženy atributy daného objektu
\item atributy jsou uloženy v pořadí dle pořadí definování
\item pokud třída dědí z jiné třídy $\rightarrow$ v paměti jsou v souvislém bloku uloženy hodnoty atributů ze všech předků i z dané třídy
\item[]
\item metoda je realizována jako vylepšená funkce
\begin{itemize}
\item na pozadí od kompilátoru obdrží ukazatel \lstinline|this|
\end{itemize}

\end{itemize}

\end{bonusblock}
\end{frame}

\begin{frame}[fragile]
\begin{bonusblock}{}
\vskip 2ex
\large \centering
Varianta A) Jednoduchá dědičnost
\vskip 2ex
\end{bonusblock}
\end{frame}



\begin{frame}[fragile]
\begin{noteblock}{}
\begin{twocols}
\begin{lstlisting}[basicstyle=\scriptsize]
struct A {
  int a = 0xaaaaaaaa;
  int b = 0xbbbbbbbb;

  int getA() const { return a; }
  int getB() const { return b; }
};
\end{lstlisting}
\twocolssep
\begin{lstlisting}[basicstyle=\scriptsize]
struct B : A {
  int c = 0xcccccccc;
  int d = 0xdddddddd;

  int getC() const { return c; }
  int getD() const { return d; }
};
\end{lstlisting}
\end{twocols}
\end{noteblock}


\begin{overlayarea}{\textwidth}{3cm}

\begin{threecols}

\begin{yesblock}<2->
\begin{lstlisting}
A objA{};
B objB{};
\end{lstlisting}
\end{yesblock}

\threecolssep

\centering
\only<3,4>{
\begin{tabular}{r|c|}
\hline
&	objA \\
\hline 
0x00 & \ttfamily 0xaaaaaaaa \\
\hline
0x04 & \ttfamily 0xbbbbbbbb \\
\hline
\end{tabular}
}\only<5>{
\begin{tabular}{r|cccc|}
\hline
\thisse & \multicolumn{4}{c|}{objA} \\
\hline 
\bc 0x00 & \bc aa & \bc aa & \bc aa & \bc aa \\
\hline
0x04 & bb & bb & bb & bb \\
\hline
\end{tabular}
}\only<6>{
\begin{tabular}{r|cccc|}
\hline
\thisse & \multicolumn{4}{c|}{objA} \\
\hline 
\gc 0x00 & \gc aa & \gc aa & \gc aa & \gc aa \\
\hline
\bc 0x04 & \bc bb & \bc bb & \bc bb & \bc bb \\
\hline
\end{tabular}
}\only<7>{
\begin{tabular}{r|cccc|}
\hline
\thisse & \multicolumn{4}{c|}{objB} \\
\hline 
\gc 0x00 & \gc aa & \gc aa & \gc aa & \gc aa \\
\hline
\gc 0x04 & \gc bb & \gc bb & \gc bb & \gc bb \\
\hline 
\bc 0x08 & \bc cc & \bc cc & \bc cc & \bc cc \\
\hline
0x0c & dd & dd & dd & dd \\
\hline
\end{tabular}
}\only<8>{
\begin{tabular}{r|cccc|}
\hline
\thisse & \multicolumn{4}{c|}{objB} \\
\hline 
\gc 0x00 & \gc aa & \gc aa & \gc aa & \gc aa \\
\hline
\gc 0x04 & \gc bb & \gc bb & \gc bb & \gc bb \\
\hline 
\gc 0x08 & \gc cc & \gc cc & \gc cc & \gc cc \\
\hline
\bc 0x0c & \bc dd & \bc dd & \bc dd & \bc dd \\
\hline
\end{tabular}
}

\threecolssep

\begin{overlayarea}{\textwidth}{3cm}
\centering
\only<4>{
\begin{tabular}{r|c|}
\hline
& objB \\
\hline 
0x00 & \ttfamily 0xaaaaaaaa \\
\hline
0x04 & \ttfamily 0xbbbbbbbb \\
\hline 
0x08 & \ttfamily 0xcccccccc \\
\hline
0x0c & \ttfamily 0xdddddddd \\
\hline
\end{tabular}
}
\end{overlayarea}

\end{threecols}
\end{overlayarea}


\begin{overlayarea}{\textwidth}{3cm}
\begin{onlyenv}<5>
\begin{yesblock}
\begin{lstlisting}
int get<@\color{red}A@>() const {
  return * (int*) ( ((char*)this) + <@\color{red}0x00@>);
}
\end{lstlisting}
\end{yesblock}
\end{onlyenv}

\begin{onlyenv}<6>
\begin{yesblock}
\begin{lstlisting}
int get<@\color{red}B@>() const {
  return * (int*) ( ((char*)this) + <@\color{red}0x04@>);
}
\end{lstlisting}
\end{yesblock}
\end{onlyenv}

\begin{onlyenv}<7>
\begin{yesblock}
\begin{lstlisting}
int get<@\color{red}C@>() const {
  return * (int*) ( ((char*)this) + <@\color{red}0x08@>);
}
\end{lstlisting}
\end{yesblock}
\end{onlyenv}

\begin{onlyenv}<8>
\begin{yesblock}
\begin{lstlisting}
int get<@\color{red}D@>() const {
  return * (int*) ( ((char*)this) + <@\color{red}0x0c@>);
}
\end{lstlisting}
\end{yesblock}
\end{onlyenv}
\end{overlayarea}

\end{frame}




\begin{frame}[fragile]
\begin{bonusblock}{}
\vskip 2ex
\large \centering
Varianta B) Vícenásobná dědičnost
\vskip 2ex
\end{bonusblock}
\end{frame}





\begin{frame}[fragile]

\begin{noteblock}{}
\begin{threecols}
\begin{lstlisting}[basicstyle=\scriptsize]
struct A {
  int a = 0xaaaaaaaa;
  int b = 0xbbbbbbbb;
  ...
};
\end{lstlisting}
\threecolssep
\begin{lstlisting}[basicstyle=\scriptsize]
struct B {
  int c = 0xcccccccc;
  int d = 0xdddddddd;
  ...
};
\end{lstlisting}
\threecolssep
\begin{lstlisting}[basicstyle=\scriptsize]
struct C : A, B {
  int e = 0xeeeeeeee;
  int f = 0xffffffff;
  ...
};
\end{lstlisting}
\end{threecols}
\end{noteblock}

\begin{overlayarea}{\textwidth}{4cm}
\begin{onlyenv}<2-7>
\begin{twocols}
\begin{yesblock}<2->
\begin{lstlisting}
C objC{};
\end{lstlisting}
\end{yesblock}

\twocolssep
\begin{overlayarea}{\textwidth}{4cm}
\begin{onlyenv}<3>
\begin{tabular}{r|c|l}
\hline
& objC & \\
\hline 
0x00 & \ttfamily 0xaaaaaaaa & $\leftarrow$ A \\
\hline
0x04 & \ttfamily 0xbbbbbbbb & \\
\hline 
0x08 & \ttfamily 0xcccccccc & $\leftarrow$ B \\
\hline
0x0c & \ttfamily 0xdddddddd & \\
\hline
0x10 & \ttfamily 0xeeeeeeee & $\leftarrow$ C \\
\hline
0x14 & \ttfamily 0xffffffff & \\
\hline
\end{tabular}
\end{onlyenv}\begin{onlyenv}<4>
\begin{tabular}{r|cccc|}
\hline
& \multicolumn{4}{c|}{objC} \\
\hline 
0x00 & aa & aa & aa & aa \\
\hline
0x04 & bb & bb & bb & bb \\
\hline 
0x08 & cc & cc & cc & cc \\
\hline
0x0c & dd & dd & dd & dd\\
\hline
0x10 & ee & ee & ee & ee\\
\hline
0x14 & ff & ff & ff & ff\\
\hline
\end{tabular}
\end{onlyenv}\begin{onlyenv}<5>
\begin{tabular}{r|cccc|}
\hline
\thisse & \multicolumn{4}{c|}{objC} \\
\hline 
\bc 0x00 & \bc aa & \bc aa & \bc aa & \bc aa \\
\hline
\bc 0x04 & \bc bb & \bc bb & \bc bb & \bc bb \\
\hline 
0x08 & cc & cc & cc & cc \\
\hline
0x0c & dd & dd & dd & dd\\
\hline
0x10 & ee & ee & ee & ee\\
\hline
0x14 & ff & ff & ff & ff\\
\hline
\end{tabular}
\end{onlyenv}\begin{onlyenv}<6>
\begin{tabular}{r|cccc|}
\hline
\thisse & \multicolumn{4}{c|}{objC} \\
\hline 
\rc 0x00 & \rc aa & \rc aa & \rc aa & \rc aa \\
\hline
\rc 0x04 & \rc bb & \rc bb & \rc bb & \rc bb \\
\hline 
0x08 & cc & cc & cc & cc \\
\hline
0x0c & dd & dd & dd & dd\\
\hline
0x10 & ee & ee & ee & ee\\
\hline
0x14 & ff & ff & ff & ff\\
\hline
\end{tabular}
\end{onlyenv}\begin{onlyenv}<7>
\begin{tabular}{r|cccc|}
\hline
\thisse & \multicolumn{4}{c|}{objC} \\
\hline 
0x00 & aa & aa & aa & aa \\
\hline
0x04 & bb & bb & bb & bb \\
\hline 
0x08 & cc & cc & cc & cc \\
\hline
0x0c & dd & dd & dd & dd\\
\hline
\bc 0x10 & \bc ee & \bc ee & \bc ee & \bc ee \\
\hline
\bc 0x14 & \bc ff & \bc ff & \bc ff & \bc ff \\
\hline
\end{tabular}
\end{onlyenv}
\end{overlayarea}
\end{twocols}
\end{onlyenv}\begin{onlyenv}<8>
\begin{block}{}
Původní předpoklad selhal u druhého předka v objektu objC\ldots
\begin{itemize}
\item reálná implementace ale funguje
\item kompilátor umí transformovat začátek objektu dle potřeby
\item \lstinline|(C*)&objC == (B*)&objC|
\item \lstinline|(char*)((C*)&objC) != (char*)((B*)&objC)|

\end{itemize}
Ve skutečnosti tedy\ldots
\end{block}
\end{onlyenv}\begin{onlyenv}<9>
\begin{center}
\vskip -5mm
\begin{tabular}{r|cccc|}
\hline
& \multicolumn{4}{c|}{objC} \\
\hline 
0x00 & aa & aa & aa & aa \\
\hline
0x04 & bb & bb & bb & bb \\
\hline 
\bc 0x08 \thise & \bc cc & \bc cc & \bc cc & \bc cc \\
\hline
\bc 0x0c & \bc dd & \bc dd & \bc dd & \bc dd\\
\hline
0x10 & ee & ee & ee & ee\\
\hline
0x14 & ff & ff & ff & ff\\
\hline
\end{tabular}
\end{center}
\end{onlyenv}
\end{overlayarea}

\begin{overlayarea}{\textwidth}{4cm}
\begin{onlyenv}<9>
\vskip -5mm
\begin{yesblock}
\begin{lstlisting}[basicstyle=\small]
int impl_getC_classC(C* object) {
  B* temporary = (B*)((char*)object + 0x08);
  return temporary->getC();
}
\end{lstlisting}
\end{yesblock}
\end{onlyenv}\begin{onlyenv}<5-7>
\begin{yesblock}
\begin{onlyenv}<5>
\begin{lstlisting}
get<@\color{red}A@>() -> * (int*) (((char*)this) + 0x00)
get<@\color{red}B@>() -> * (int*) (((char*)this) + 0x04)
\end{lstlisting}
\end{onlyenv}

\begin{onlyenv}<6>
\begin{lstlisting}
get<@\color{red}C@>() -> * (int*) (((char*)this) + 0x00)
get<@\color{red}D@>() -> * (int*) (((char*)this) + 0x04)
\end{lstlisting}
\end{onlyenv}

\begin{onlyenv}<7>
\begin{lstlisting}
get<@\color{red}E@>() -> * (int*) (((char*)this) + 0x10)
get<@\color{red}F@>() -> * (int*) (((char*)this) + 0x14)
\end{lstlisting}
\end{onlyenv}

\end{yesblock}
\end{onlyenv}
\end{overlayarea}


\end{frame}










\begin{frame}[fragile]
\begin{bonusblock}{}
\vskip 2ex
\large \centering
Varianta C) Diamond problem -- virtuální dědičnost
\vskip 2ex
\end{bonusblock}
\end{frame}








\begin{frame}[fragile]
\begin{noteblock}{}
\begin{lstlisting}
struct A {
  int a = 0xaaaaaaaa;
  int b = 0xbbbbbbbb;
};
\end{lstlisting}
\end{noteblock}

\begin{twocols}
\begin{noteblock}{}
\begin{lstlisting}
struct B : A {
  int c = 0xcccccccc;
  int d = 0xdddddddd;
}
\end{lstlisting}
\end{noteblock}
\twocolssep
\begin{noteblock}{}
\begin{lstlisting}
struct C : A {
  int e = 0xeeeeeeee;
  int f = 0xffffffff;
}
\end{lstlisting}
\end{noteblock}
\end{twocols}

\begin{noteblock}{}
\begin{lstlisting}
struct D : B, C {
  int x = 0x99999999;
}
\end{lstlisting}
\end{noteblock}
\end{frame}



\begin{frame}[fragile]
\begin{yesblock}
\begin{lstlisting}
D objD{};
\end{lstlisting}
\end{yesblock}

\vskip 1ex
\begin{center}
\begin{onlyenv}<1>
\begin{tabular}{c|c|l}
& objD & \\
\hline
0x00 & \ttfamily 0xaaaaaaaa & $\leftarrow B, A_B$ \\
\hline
0x04 & \ttfamily 0xbbbbbbbb & \\
\hline
0x08 & \ttfamily 0xcccccccc & $\leftarrow B\ldots$ \\
\hline
0x0c & \ttfamily 0xdddddddd &  \\
\hline
0x10 & \ttfamily 0xaaaaaaaa & $\leftarrow C, A_C$ \\
\hline
0x14 & \ttfamily 0xbbbbbbbb & \\
\hline
0x18 & \ttfamily 0xeeeeeeee & $\leftarrow C\ldots$ \\
\hline
0x1c & \ttfamily 0xffffffff &  \\
\hline
0x20 & \ttfamily 0x99999999 & $\leftarrow D$ \\
\hline
\end{tabular}
\end{onlyenv}\begin{onlyenv}<2>
\begin{tabular}{c|c|l}
& objD & \\
\hline
0x00 & \ttfamily \bc 0xaaaaaaaa & $\leftarrow B, A_B$ \\
\hline
0x04 & \ttfamily \bc 0xbbbbbbbb & \\
\hline
0x08 & \ttfamily 0xcccccccc & $\leftarrow B\ldots$ \\
\hline
0x0c & \ttfamily 0xdddddddd &  \\
\hline
0x10 & \ttfamily \bc 0xaaaaaaaa & $\leftarrow C, A_C$ \\
\hline
0x14 & \ttfamily \bc 0xbbbbbbbb & \\
\hline
0x18 & \ttfamily 0xeeeeeeee & $\leftarrow C\ldots$ \\
\hline
0x1c & \ttfamily 0xffffffff &  \\
\hline
0x20 & \ttfamily 0x99999999 & $\leftarrow D$ \\
\hline
\end{tabular}
\end{onlyenv}
\end{center}
\end{frame}


\begin{frame}[fragile]
\begin{yesblock}
\begin{lstlisting}
D objD{};
D* d = &objD;
\end{lstlisting}
\end{yesblock}

\begin{noblock}
\begin{lstlisting}
A* a = (A*)d; // nelze - které A?
\end{lstlisting}
\end{noblock}

\begin{yesblock}
\begin{lstlisting}
A* a = (A*)(B*)d; // lze
\end{lstlisting}
\end{yesblock}
\end{frame}



\begin{frame}[fragile]
\begin{noteblock}{}
\begin{lstlisting}
struct A {
  int a = 0xaaaaaaaa;
  int b = 0xbbbbbbbb;
};
\end{lstlisting}
\end{noteblock}

\begin{twocols}
\begin{noteblock}{}
\begin{lstlisting}
struct B : virtual A {
  int c = 0xcccccccc;
  int d = 0xdddddddd;
}
\end{lstlisting}
\end{noteblock}
\twocolssep
\begin{noteblock}{}
\begin{lstlisting}
struct C : virtual A {
  int e = 0xeeeeeeee;
  int f = 0xffffffff;
}
\end{lstlisting}
\end{noteblock}
\end{twocols}

\begin{noteblock}{}
\begin{lstlisting}
struct D : B, C {
  int x = 0x99999999;
}
\end{lstlisting}
\end{noteblock}
\end{frame}





\begin{frame}[fragile]
\begin{yesblock}
\begin{lstlisting}
D objD{};
\end{lstlisting}
\end{yesblock}

\vskip 1ex
\begin{center}
\begin{onlyenv}<1>
\begin{tabular}{c|c|l}
& objD & \\
\hline
0x00 & \ttfamily 0x1c & $\leftarrow B$ \\
\hline
0x04 & \ttfamily 0xcccccccc & \\
\hline
0x08 & \ttfamily 0xdddddddd &  \\
\hline
0x0c & \ttfamily 0x1c & $\leftarrow C$ \\
\hline
0x10 & \ttfamily 0xeeeeeeee & \\
\hline
0x14 & \ttfamily 0xffffffff &  \\
\hline
0x18 & \ttfamily 0x99999999 & $\leftarrow D$ \\
\hline
0x1c & \ttfamily 0xaaaaaaaa & $\leftarrow A$ \\
\hline
0x20 & \ttfamily 0xbbbbbbbb & \\
\hline
\end{tabular}
\end{onlyenv}\begin{onlyenv}<2>
\begin{tabular}{c|c|l}
& objD & \\
\hline
0x00 & \ttfamily \grc 0x1c ($ptr_{B\rightarrow{}A}$)  & $\leftarrow B$ \\
\hline
0x04 & \ttfamily 0xcccccccc & \\
\hline
0x08 & \ttfamily 0xdddddddd &  \\
\hline
0x0c & \ttfamily \grc 0x1c ($ptr_{C\rightarrow{}A}$) & $\leftarrow C$ \\
\hline
0x10 & \ttfamily 0xeeeeeeee & \\
\hline
0x14 & \ttfamily 0xffffffff &  \\
\hline
0x18 & \ttfamily 0x99999999 & $\leftarrow D$ \\
\hline
0x1c & \ttfamily \bc 0xaaaaaaaa & $\leftarrow A$ \\
\hline
0x20 & \ttfamily \bc 0xbbbbbbbb & \\
\hline

\end{tabular}
\end{onlyenv}
\end{center}
\end{frame}



\begin{frame}[fragile]
\begin{bonusblock}{Dědičnost}
\begin{itemize}
\item vícenásobná dědičnost využívá automatické transformace počátku objektu
\begin{itemize}
\item kód metody neřeší s jakým typem objektu pracuje a prostě přistupuje k~atributům
\item pokud je objekt složitý je potřeba přepočítat jeho počátek před voláním metody předka
\item provádí kompilátor automaticky na pozadí
\end{itemize}

\item virtuální dědičnost využívá odkazu na předka na začátku každého objektu
\begin{itemize}
\item tak je možné vložit společného předka pouze jednou
\item metody využijí odkazu k dohledání předka a poté mohou pracovat s~jeho atributy
\end{itemize}

\end{itemize}
\end{bonusblock}
\end{frame}