% chybí - advance, distance, iter_swap

\hkapitola{Iterátory}


\begin{frame}[fragile]
\begin{block}{Iterátory}
\begin{itemize}
\item představují obecné \textit{rozhraní} pro procházení kontejnerů
\item vycházejí z logiky práce s ukazateli
\item umožňují číst i zapisovat hodnoty
\end{itemize}

\end{block}

\begin{block}{Kategorie iterátorů}
\begin{itemize}
\item vstupní (input) -- slouží k získání dat z kontejneru
\item výstupní (output) -- slouží k zapsání dat do kontejneru (standardně fungují tak, že přepisují stávající data!)
\end{itemize}

Detailní popis vlastností k nalezení na: 
\url{http://www.cplusplus.com/reference/iterator/InputIterator}
\url{http://en.cppreference.com/w/cpp/iterator}
\end{block}
\end{frame}


\begin{frame}[fragile]
\frametitle{Druhy iterátorů}
\begin{block}{}
\begin{itemize}
\item vstupní (input) -- jednou lze přistoupit k prvku (pak je nutné posun na další prvek)
\item výstupní (output) -- jednou lze zapsat prvek (pak je nutné posun na další prvek)
\item dopředný (forward) -- jeden prvek lze číst opakovaně, lze i zapisovat
\item obousměrný (bidirectional) -- v kontejneru jde pohyb dopředu i dozadu
\item s náhodným přístupem (random access) -- v kontejneru jde pohyb i~o~$X$ prvků na obě strany
\item spojitý (contiguous)\cpp{17} -- iterátor nad spojitým paměťovým blokem
\end{itemize}
\end{block}
\end{frame}



\begin{frame}[fragile]
\begin{block}{Procházení rozsahu}
Na projití rozsahu potřebujeme dva iterátory - začátek a konec. Proto všude najdeme dvojice metod: begin-end, cbegin-cend, rbegin-rend, crbegin-crend.
	
\begin{itemize}
\item \lstinline|begin-end| -- pohyb dopředu, prvky nekonstatní
\item \lstinline|cbegin-cend| -- pohyb dopředu, prvky konstatní
\item \lstinline|rbegin-rend| -- pohyb odzadu, prvky nekonstatní
\item \lstinline|crbegin-crend| -- pohyb odzadu, prvky konstatní

\end{itemize}
\end{block}

\begin{exampleblock}{{\YES} Použití iterátoru}
\begin{itemize}
\item \lstinline|*it| -- přístup k hodnotě prvku
\item \lstinline|it->...| -- přístup k hodnotě prvku (objekty)
\item \lstinline|++it, it++| -- posun na další prvky
\item \lstinline|it != it2, it != kontejner.end()| -- kontrola konce
\end{itemize}
\end{exampleblock}
\end{frame}




\begin{frame}[fragile]
\frametitle{Použití vstupních iterátorů}
\begin{yesblock}
\begin{lstlisting}
vector<int> vect;

for(vector<int>::iterator it = vect.begin(); it != vector.end(); it++) {
	int data = *it;
}
\end{lstlisting}
\end{yesblock}
\end{frame}



\begin{frame}[fragile]
\frametitle{Použití vstupních iterátorů}
\begin{yesblock}
\begin{lstlisting}
struct osoba { string jmeno; string prijmeni; };

vector<osoba> vect;

for (auto it = vect.begin(); it != vect.end(); it++) {
	osoba& os = *it;
	string jmeno = it->jmeno;
	string prijmeni = (*it).prijmeni;
}
\end{lstlisting}
\end{yesblock}
\end{frame}


\begin{frame}[fragile]
\frametitle{Použití vstupních iterátorů -- foreach\cpp{11}}
\begin{yesblock}
\begin{lstlisting}
struct osoba { string jmeno; string prijmeni; };

vector<osoba> vect;

for (osoba& it : vect) {
// také lze
// for (auto& it : vect) {
	osoba& os = it;
	string jmeno = it.jmeno;
	string prijmeni = it.prijmeni;
}
\end{lstlisting}
\end{yesblock}
\end{frame}


\begin{frame}[fragile]
\begin{alertblock}{{\NO} Pozor na rušení platnosti iterátorů}
\begin{lstlisting}
vector<int> vect;

for (auto it = vect.begin(); it != vect.end(); it++) {
	if (*it == 123456) 
		vect.erase(it); // smazani prvku -> zrusi platnost iteratoru -> crash
}		
\end{lstlisting}
\end{alertblock}

\begin{yesblock}
\begin{lstlisting}
for (auto it = vect.begin(); it != vect.end(); ) {
	if (*it == 123456) 
		it = vect.erase(it); // ok -> erase vraci platny iterator na dalsi prvek
	else
		it++;
}		
\end{lstlisting}
\end{yesblock}	
\end{frame}




\begin{frame}[fragile]
\begin{alertblock}{{\NO} Použití výstupních iterátorů}
\begin{lstlisting}
vector<int> vect;
auto it = vect.begin();
*it = 123;
// crash - vektor byl prazdny
\end{lstlisting}
\end{alertblock}

\begin{exampleblock}{{\YES} Použití výstupních iterátorů}
\begin{lstlisting}
vector<int> vect;
vect.resize(2);
auto it = vect.begin();
*it = 123;
it++;
*it = 456;
// vect nyni obsahuje [123, 456]
\end{lstlisting}
\end{exampleblock}	
\end{frame}




\begin{frame}[fragile]
\frametitle{Výstupní iterátorové adaptéry}
\begin{block}{}
Aby výstupní iterátory \textbf{nepřepisovaly}, ale \textbf{vkládaly} data - používají se adaptéry (hl. soubor \lstinline|<iterator>|) \vskip.3cm		
\begin{itemize}
\item \lstinline|insert_iterator| - volá metodu insert
\begin{itemize}
\item adaptér lze snadno vytvořit využitím pomocné funkce \lstinline|inserter(kontejner, pozice)|
\end{itemize}
\item \lstinline|back_insert_iterator| - volá metodu \lstinline|push_back|
\begin{itemize}

\item \lstinline|back_inserter(kontejner)|
\end{itemize}
\item \lstinline|front_insert_iterator| - volá metodu \lstinline|push_front|
\begin{itemize}
\item \lstinline|front_inserter(kontejner)|
\end{itemize}

\end{itemize}
\end{block}
\end{frame}




\begin{frame}[fragile]
\begin{exampleblock}{{\YES} Použití výstupních iterátorů -- adaptéru back\_insert\_iterator}
\begin{lstlisting}
vector<int> vect;
auto it = back_inserter(vect);
*it = 123;
it++;
*it = 456;
it++;
*it = 789;
// vect nyni obsahuje [123, 456, 789]
\end{lstlisting}
\end{exampleblock}	

\begin{yesblock}
\begin{lstlisting}
vector<int> vect;
auto it = back_inserter(vect);
for(int i = 0; i < 10; i++, it++) {
	*it = i;
}
// vect = [0, 1, 2, 3, 4, 5, 6, 7, 8, 9]
\end{lstlisting}
\end{yesblock}	
\end{frame}



\begin{frame}[fragile]
\begin{exampleblock}{{\YES} Kopírování obsahu pomocí dvou iterátorů}
\begin{lstlisting}
vector<int> read = {1, 10, 15, 20};
vector<int> write;

auto wit = back_inserter(write);

for (auto rit = read.begin(); rit != read.end(); rit++, wit++){
	*wit = *rit;
}
// write = [1, 10, 15, 20]
\end{lstlisting}
\end{exampleblock}	
\end{frame}


\begin{frame}[fragile]
\frametitle{Proudové iterátorové adaptéry}
\begin{block}{}
Pro přístup k proudům (*stream) se standardně používají operátory \lstinline|<<| a \lstinline|>>|. Existují dva adaptéry, aby se daly zpracovávat jako iterátory:

\begin{itemize}
\item \lstinline|ostream_iterator|
\begin{itemize}
\item  \lstinline|ostream_iterator(ostream)|
\item  \lstinline|ostream_iterator(ostream, oddelovac)|		
\end{itemize}
\item \lstinline|istream_iterator|
\begin{itemize}
\item \lstinline|istream_iterator(istream)|
\end{itemize}

\end{itemize}

\end{block}
\end{frame}


\begin{frame}[fragile]
\begin{exampleblock}{{\YES} Proudový iterátorový adaptér}
\begin{lstlisting}
ostream_iterator<int> osi{ cout, "--" };

*osi = 10;
osi++;
*osi = 20;
osi++;
*osi = 30;
osi++;

// na výstupu bude: 10--20--30--
\end{lstlisting}
\end{exampleblock}	
\end{frame}



\begin{frame}[fragile]
\begin{block}{Pomocné funkce v <iterator>}
\begin{itemize}
\item \lstinline|advance(iterator, posun)| -- posune iterátor o "posun" prvků
\item \lstinline|distance(it1, it2)| -- vrací počet prvků v rozsahu it1 -- it2
\item \lstinline|next(it, posun = 1)|\cpp{11} -- posune iterátor na následující prvek
\item \lstinline|prev(it, posun = 1)|\cpp{11} -- posune iterátor na předcházející prvek
\item \lstinline|iter_swap(it1, it2)| -- prohodí obsah iterátorů it1 a it2
\end{itemize}
\end{block}
\end{frame}



\kapitola{Tvorba vlastního vstupního iterátoru}


\begin{frame}[fragile]
\begin{exampleblock}{{\YES} Šablona kontejneru MyArray}
\begin{lstlisting}
// šablona staticky alokovaného pole obsahující prvku typu T (počet prvků je Size)
template<typename T, int Size>
struct MyArray {

	// pole prvků
	T _data[Size];
	// velikost pole
	const int _size = Size;
};
\end{lstlisting}
\end{exampleblock}	
\end{frame}

\begin{frame}[fragile]
\begin{exampleblock}{{\YES} Základ iterátoru \ldots}
\begin{lstlisting}
template<typename T, int Size>
struct MyArray {

	// implementace iterátoru
	struct iterator {
	};

	// begin - vrací iterátor ukazující na první prvek pole
	iterator begin() {
	}

	// end - vrací iterátor ukazující za poslední prvek pole
	iterator end() {
	}
	
	// ...
\end{lstlisting}
\end{exampleblock}	
\end{frame}



\begin{frame}[fragile]
\begin{deprecatedblock}{{\WARNING} Definice veřejných typů}
\begin{lstlisting}
// do C++17 lze dědit z std::iterator
struct iterator : std::iterator<std::forward_iterator_tag, T>
\end{lstlisting}
\end{deprecatedblock}	


\begin{exampleblock}{{\YES} Definice veřejných typů}
\begin{lstlisting}
typedef forward_iterator_tag iterator_category;

typedef T value_type;
typedef ptrdiff_t difference_type; // int / int64

typedef T* pointer;
typedef T& reference;
\end{lstlisting}
\end{exampleblock}	
\end{frame}

\begin{frame}[fragile]
\begin{exampleblock}{{\YES} Definice iterátoru\ldots}
\begin{lstlisting}
struct iterator {
	public:
		// konstruktor iterátoru - předáváme ukazatel s prvekem kam bude iterátor ukazovat
		iterator(T* ptr) : _ptr(ptr) { } 

	private:
		// pro realizaci iterátoru stačí uchovat ukazatel na aktuální prvek
		T* _ptr;
};		
\end{lstlisting}
\end{exampleblock}	
\end{frame}


\begin{frame}[fragile]
\begin{exampleblock}{{\YES} Metody begin() a end() v MyArray}
\begin{lstlisting}
// implementace begin() a end(), aby vracely korektní iterátor
iterator begin() {
	return {_data};
}

iterator end() {
	return {_data + _size};
}
\end{lstlisting}
\end{exampleblock}	
\end{frame}


\begin{frame}[fragile]
\begin{exampleblock}{{\YES} Posun iterátoru na další prvky}
\begin{lstlisting}
// posun iterátoru na další prvek
iterator& operator++() {
	_ptr++;
	return *this;
}

// postinkrementální varianta
iterator operator++(int) {
	iterator copy{*this};
	_ptr++;
	return copy;
}
\end{lstlisting}
\end{exampleblock}	
\end{frame}


\begin{frame}[fragile]
\begin{exampleblock}{{\YES} Přístup k prvkům 1.}
\begin{lstlisting}
// vrácení prvku z iterátoru
reference operator*() const {
	// stačí dereferencovat ukazatel na aktuální prvek
	return *_ptr;
}
\end{lstlisting}
\end{exampleblock}	
\end{frame}

\begin{frame}[fragile]
\begin{exampleblock}{{\YES} Přístup k prvkům 2.}
\begin{lstlisting}
// přístup k prvkům pomocí operátoru šipky
pointer operator->() const{
	return _ptr;
}    	
\end{lstlisting}
\end{exampleblock}	
\end{frame}


\begin{frame}[fragile]
\begin{exampleblock}{{\YES} Porovnání iterátorů}
\begin{lstlisting}
// porovnání iterátorů - aby bylo možné zjistit konec iterování
bool operator==(const iterator& it) const {
	return _ptr == it._ptr;
}

bool operator!=(const iterator& it) const {
	return !(*this == it);
}	
\end{lstlisting}
\end{exampleblock}	
\end{frame}


\begin{frame}[fragile]
\begin{exampleblock}{{\YES} Konstruktor MyArray pro initializer\_list}
\begin{lstlisting}
#include <initializer_list>		
MyArray(std::initializer_list<T> il) {
	int i = 0;
	for (auto it = il.begin(); it != il.end(); it++) {
		_data[i] = *it;
	}
}
\end{lstlisting}
\end{exampleblock}		

\begin{exampleblock}{{\YES} Test iterátoru}
\begin{lstlisting}
// a je hotovo... test kontejneru a iterátoru
// vytvoření pole s prvky [1, 2, 3]
MyArray<int, 3> ary { 1, 2, 3 };

for (auto it = ary.begin(); it != ary.end(); ++it) {
	std::cout << *it << std::endl;
}
\end{lstlisting}
\end{exampleblock}	
\end{frame}

