\hkapitola{Dědičnost}

\begin{frame}[fragile]
\frametitle{Dědičnost v C++}

\begin{bitemize}
\item podporována vícenásobná dědičnost
\begin{itemize}
\item třída může dědit z několika tříd
\end{itemize}
\item neexistují rozhraní
\begin{itemize}
\item ale existují abstraktní třídy
\item uzavřené (sealed/final) třídy
\end{itemize}
\item u zděděných složek (atributů a metod) lze ovlivnit viditelnost
\begin{itemize}
\item lze hromadně omezit přístup k těmto složkám
\item lze individuálně obnovit nebo omezit přístup
\end{itemize}
\item metody lze přetížit/přepsat v potomcích
\begin{itemize}
\item C++ rozlišuje časnou a pozdní vazbu
\end{itemize}
\end{bitemize}
\end{frame}


\begin{frame}[fragile]
\frametitle{Dědičnost}
\begin{noteblock}{}
\begin{lstlisting}[basicstyle=\scriptsize]
struct|class nazevDatovehoTypu [final] [dědičnost] {
  [složky - atributy, metody, vnořené typy]
} [objekty];
\end{lstlisting}
\end{noteblock}

\begin{bitemize}
\item{} [\lstinline|final|]\cpp{11} -- zakazuje dědit z této třídy
\item{} [dědičnost] -- specifikace předků
\begin{itemize}
\item \lstinline|: [viditelnost] [virtual] předek1, ...|
\end{itemize}
\end{bitemize}

\begin{yesblock}
\begin{lstlisting}
struct Pes : public virtual Zvire, public IIdentifikovatelny, IKopirovatelny {
  ...
};
\end{lstlisting}
\end{yesblock}
\end{frame}

\begin{frame}[fragile]
\frametitle{Dědičnost\ldots}
\begin{bitemize}
\item pokud viditelnost není uvedena využije se výchozí dle typu
\begin{itemize}
\item \lstinline|class| -- \lstinline|private|
\item \lstinline|struct| -- \lstinline|public|
\end{itemize}
\item viditelnost udává s jakou viditelností jsou zděděny složky z daného předka, pokud uvedeme
\begin{itemize}
\item \lstinline|public| -- nic se nemění
\item \lstinline|protected| -- \lstinline|public| složky z předka jsou převedeny na \lstinline|protected|
\item \lstinline|private| -- \lstinline|public| a \lstinline|protected| složky z předka jsou převedeny na \lstinline|private|
\end{itemize}
\end{bitemize}

\begin{noteblock}{}
Aby se to chovalo á la Java:
\begin{itemize}
\item používejte \lstinline|struct| a nic tam nepište
\item používejte \lstinline|struct| i \lstinline|class| a vždy uvádějte viditelnost \lstinline|public|
\end{itemize}
\end{noteblock}
\end{frame}






\begin{frame}[fragile]
\frametitle{Dědičnost\ldots}
\begin{bonusblock}{Obnovení viditelnosti}
\begin{itemize}
\item viditelnost je možné obnovit uvedením složky předka ve tvaru \lstinline|TřídaPředka::složka| v bloku s danou viditelností (rovněž lze využít i \lstinline|using TřídaPředka::složka|)
\end{itemize}
\end{bonusblock}
\vskip -2ex
\begin{bonusblock}{}
\begin{twocols}
\begin{lstlisting}
class TA {

  protected:
    int x, y;

  public:
    int GetX() const;
    // ...
};

\end{lstlisting}
\twocolssep
\begin{lstlisting}
class TB : private TA {

  protected:
    TA::x;

  public:
    TA::GetX;
    // ...
};
\end{lstlisting}
\end{twocols}
\end{bonusblock}
\end{frame}









\begin{frame}[fragile]
\begin{bitemize}
\item potomek může zastoupit předka
\begin{itemize}
\item je korektní přiřadit ukazatel na potomka do ukazatele na předka
\item je korektní přiřadit referenci na potomka do reference na předka
\item NENÍ korektní přiřadit objekt potomka do objektu předka
\begin{itemize}
\item dojde k oříznutí objektu!
\end{itemize}
\end{itemize}
\end{bitemize}

\begin{yesblock}
\begin{lstlisting}
Potomek potomek{};

Predek& ref = potomek;
Predek* ptr = &potomek;
...
\end{lstlisting}
\end{yesblock}

\begin{noblock}
\begin{lstlisting}
Predek predek = potomek; // oříznutí objektu!!!
\end{lstlisting}
\end{noblock}
\end{frame}







\begin{frame}[fragile]
\frametitle{Volání konstruktoru předka}

\begin{yesblock}
\begin{lstlisting}[basicstyle=\scriptsize]
struct Predek {
  Predek() : _a(0) { }
  Predek(int a) : _a(a) { }

private:
  int _a;
}

struct Potomek : Predek {
  Potomek() : _b(0) {
    // kompilátor implicitně volá Predek::Predek();
  }

  Potomek(int a, int b) : Predek(a), _b(0) {
    // kompilátor explicitně volá Predek::Predek(int);
  }

private:
  int _b;
}
\end{lstlisting}
\end{yesblock}
\end{frame}


\begin{frame}[fragile]
%\frametitle{Volání konstruktoru předka\ldots}
%\vskip -.99ex
\begin{yesblock}
\begin{lstlisting}[basicstyle=\scriptsize]
struct Predek {
  //Predek() : _a(0) { }
  Predek(int a) : _a(a) { }

private:
  int _a;
}
\end{lstlisting}
\end{yesblock}
%\vskip -2.0ex
\begin{noblock}
\begin{lstlisting}[basicstyle=\scriptsize]
struct Potomek : Predek {
  Potomek() : _b(0) {
    // kompilátor implicitně volá Predek::Predek();
    // konstruktor neexistuje !!! chyba kompilace
  }

  Potomek(int a, int b) : Predek(a), _b(0) {
    // kompilátor explicitně volá Predek::Predek(int);
    // ok!
  }

private:
  int _b;
}
\end{lstlisting}
\end{noblock}
\end{frame}




% virtualni nevirtualni dedeni
% using na konstruktory
% volani konstruktoru predka
% proces tvorby objektu?
