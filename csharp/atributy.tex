\nezkouskove

\hkapitola{Atributy}

\begin{frame}[fragile]
\vfill
\begin{bitemize}{Atributy}
\item obdoba anotací z Javy
\item slouží k doplnění metadat k jednotlivým typům a jejich složkám
\item atributy jsou realizovány jako potomci třídy \lstinline|System.Attribute|
\item atribut může definovat sadu parametrů, které je u něj možné nastavit
\item informace o atributech lze získat za běhu pomocí reflexe
\end{bitemize}
\vfill
\begin{yesblock}
\begin{lstlisting}
[System.Serializable]  
public class SampleClass  
{  
    // Objects of this type can be serialized.  
}  
\end{lstlisting}
\end{yesblock}
\vfill
\end{frame}


\begin{frame}[fragile]
\vfill
\begin{yesblock}
\begin{lstlisting}
// použití nativní funkce z DLL knihovny
[System.Runtime.InteropServices.DllImport("user32.dll")]  
extern static void SampleMethod();  
\end{lstlisting}
\end{yesblock}
\vfill
\begin{yesblock}
\begin{lstlisting}
void MethodA([In][Out] ref double x) { }  
void MethodB([Out][In] ref double x) { }  
void MethodC([In, Out] ref double x) { }  
\end{lstlisting}
\end{yesblock}
\vfill
\begin{yesblock}
\begin{lstlisting}
[Conditional("DEBUG"), Conditional("TEST1")]  
void TraceMethod()  
{  
    // ...  
}  
\end{lstlisting}
\end{yesblock}
\vfill
\end{frame}


\begin{frame}[fragile]
\vfill
\begin{bitemize}{Definice nového atributu}
\item je třída dědící z \lstinline|System.Attribute|
\end{bitemize}
\vfill
\begin{yesblock}
\begin{lstlisting}
[AttributeUsage(AttributeTargets.Class |  
                  AttributeTargets.Struct)  
]  
public class AuthorAttribute : Attribute  
{  
    private string name;  
    public double version;  

    public Author(string name)  
    {  
        this.name = name;  
        version = 1.0;  
    }  
}  
\end{lstlisting}
\end{yesblock}
\vfill
\end{frame}


\begin{frame}[fragile]
\vfill
\begin{yesblock}
\begin{lstlisting}
[Author("P. Ackerman", version = 1.1)]  
class SampleClass  
{  
    // P. Ackerman's code goes here...  
}  
\end{lstlisting}
\end{yesblock}
\vfill
\begin{yesblock}
\begin{lstlisting}
Type t = typeof(SampleClass);
Attribute[] attrs = Attribute.GetCustomAttributes(t);

foreach (Attribute attr in attrs)  
{  
    if (attr is Author)  
    {  
        AuthorAttribute a = (AuthorAttribute)attr;  
        Console.WriteLine("   {0}, version {1:f}", a.GetName(), a.version);  
    }  
}  
\end{lstlisting}
\end{yesblock}
\vfill
\end{frame}



\begin{frame}[fragile]
\vfill
\begin{bitemize}{}
\item \lstinline|AttributeUsage| specifikuje nad jakými typy/komponentami lze atribut použít, je možné povolit vícenásobné použití nad jednou komponentou
\item u atributu je možné uvést pro co konkrétné platí
\end{bitemize}
\vfill
\begin{yesblock}
\begin{lstlisting}
// default: applies to method  
[SomeAttr]  
int Method1() { return 0; }  

// applies to method  
[method: SomeAttr]  
int Method2() { return 0; }  

// applies to return value  
[return: SomeAttr]  
int Method3() { return 0; }  
\end{lstlisting}
\end{yesblock}
\vfill
\end{frame}

% attribute targetting
% custom attribute


\zkouskove
