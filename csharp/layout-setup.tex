\usepackage[czech]{babel} 
%\usepackage[IL2,T1,T2A]{fontenc} 
\usepackage[T1]{fontenc} 

%\usetheme{JuanLesPins}
\usetheme{Boadilla}

%%%%%%%%%%%%%%%%%%%%%%%%%%%%%%%%%%%%%%%%%%%%%%%%%%%
\setbeamertemplate{footline}[frame number]{}
\setbeamertemplate{navigation symbols}{}
\setbeamertemplate{blocks}[rounded][shadow=false]
%\setbeamertemplate{blocks}[default]
%\setbeamertemplate{background canvas}[vertical shading][bottom=white,top=structure.fg!5]

% zrušení mezer mezi bloky
\setlength{\medskipamount}{0pt}
\setlength{\smallskipamount}{0pt}

% pro vkládání obrázků
\usepackage{graphicx} 
% pro použití víceřádkových komentářů
\usepackage{verbatim} 
% H modifikátor figure
\usepackage{float}
% barvičky
\usepackage{xcolor,colortbl}
\usepackage{subfig}


\newcommand{\rc}{\cellcolor{red!25}}
\newcommand{\bc}{\cellcolor{blue!25}}
\newcommand{\gc}{\cellcolor{gray!25}}
\newcommand{\grc}{\cellcolor{green!25}}

\newcommand{\cpp}[1]{{\footnotesize$^{C++#1}$}}

\newcommand{\thisse}{\color{blue}\textbf{{\tiny this}}$\searrow$}
\newcommand{\thise}{\color{blue}\textbf{{\tiny this}}$\rightarrow$}

\usepackage{tikz}
\usetikzlibrary{arrows,shapes}
\newcommand{\tikzmark}[1]{\tikz[remember picture] \node[coordinate] (#1) {#1};}

% listingy
\usepackage{listings}

\newcommand{\commentcolor}{\color[rgb]{0.133,0.545,0.133}}
\lstset{
  tabsize=2,
  language=matlab,
  basicstyle=\listingssize,% \ttfamily,
  %upquote=true,
  mathescape=false,
  aboveskip=0pt, %{1.5\baselineskip},
  % zrušení nadbytečné mezery za posledním řádkem
  belowskip=0pt,
  columns=fullflexible,
  showstringspaces=false,,
  breaklines=true,
  prebreak = \raisebox{0ex}[0ex][0ex]{\ensuremath{\hookleftarrow}},
  frame=none,
  showtabs=false,
  showspaces=false,
  showstringspaces=false,
  identifierstyle=\ttfamily,
  keywordstyle=\color[rgb]{0,0,1}\bfseries,
  commentstyle=\color[rgb]{0.133,0.545,0.133},
  stringstyle=\color[rgb]{0.627,0.126,0.941},
  language=Java,
  inputencoding=utf8,
  extendedchars=true,
  literate={á}{{\'a}}1 {ã}{{\~a}}1 {é}{{\'e}}1 {ž}{{\v{z}}}1 {ý}{{\'y}}1 {ě}{{\v{e}}}1 {ř}{{\v{r}}}1 {í}{{\'i}}1 {ů}{{\r{u}}}1 {č}{{\v{c}}}1 {ú}{{\'u}}1 {š}{{\v{s}}}1 {ť}{{\v{t}}}1 {Č}{{\v{C}}}1 {Š}{{\v{S}}}1 {ň}{{\v{n}}}1 {Ř}{{\v{R}}}1 {ó}{{\'o}}1 %{\$}{{\$}}1
}

	
%\lstloadlanguages{[11]C++}
%\lstset{language=[11]C++}
\lstset{language=[Sharp]C}
\lstset{morekeywords={var,when}}
\lstset{escapeinside={<@}{@>}}

%\usepackage{pxfonts}
% odkazy v pdf
\usepackage{hyperref}

\author{\autor}
\title{\titulek}
\institute{UPCE/FEI/KST}
\date{}

\hypersetup{
pdftitle={\titulek},
pdfsubject={\predmet},
pdfauthor={\autor}
}









\mode<handout>
{
	\usepackage{pgf}
	\usepackage{pgfpages}

	\pgfpagesdeclarelayout{4 on 1 boxed}
	{
	  \edef\pgfpageoptionheight{\the\paperheight} 
	  \edef\pgfpageoptionwidth{\the\paperwidth}
	  \edef\pgfpageoptionborder{0pt}
	}
	{
	  \pgfpagesphysicalpageoptions
	  {%
	    logical pages=4,%
	    physical height=\pgfpageoptionheight,%
	    physical width=\pgfpageoptionwidth%
	  }
	  \pgfpageslogicalpageoptions{1}
	  {%
	    border code=\pgfsetlinewidth{2pt}\pgfstroke,%
	    border shrink=\pgfpageoptionborder,%
	    resized width=.5\pgfphysicalwidth,%
	    resized height=.5\pgfphysicalheight,%
	    center=\pgfpoint{.25\pgfphysicalwidth}{.75\pgfphysicalheight}%
	  }%
	  \pgfpageslogicalpageoptions{2}
	  {%
	    border code=\pgfsetlinewidth{2pt}\pgfstroke,%
	    border shrink=\pgfpageoptionborder,%
	    resized width=.5\pgfphysicalwidth,%
	    resized height=.5\pgfphysicalheight,%
	    center=\pgfpoint{.75\pgfphysicalwidth}{.75\pgfphysicalheight}%
	  }%
	  \pgfpageslogicalpageoptions{3}
	  {%
	    border code=\pgfsetlinewidth{2pt}\pgfstroke,%
	    border shrink=\pgfpageoptionborder,%
	    resized width=.5\pgfphysicalwidth,%
	    resized height=.5\pgfphysicalheight,%
	    center=\pgfpoint{.25\pgfphysicalwidth}{.25\pgfphysicalheight}%
	  }%
	  \pgfpageslogicalpageoptions{4}
	  {%
	    border code=\pgfsetlinewidth{2pt}\pgfstroke,%
	    border shrink=\pgfpageoptionborder,%
	    resized width=.5\pgfphysicalwidth,%
	    resized height=.5\pgfphysicalheight,%
	    center=\pgfpoint{.75\pgfphysicalwidth}{.25\pgfphysicalheight}%
	  }%
	}
	
	
	\pgfpagesdeclarelayout{4 on 1 b}
	{
	  \edef\pgfpageoptionheight{\the\paperheight} 
	  \edef\pgfpageoptionwidth{\the\paperwidth}
	  \edef\pgfpageoptionborder{0pt}
	}
	{
	  \pgfpagesphysicalpageoptions
	  {%
	    logical pages=4,%
	    physical height=\pgfpageoptionheight,%
	    physical width=\pgfpageoptionwidth%
	  }
	  \pgfpageslogicalpageoptions{1}
	  {%
	%    border code=\pgfsetlinewidth{2pt}\pgfstroke,%
	%    border shrink=\pgfpageoptionborder,%
	    resized width=.5\pgfphysicalwidth,%
	    resized height=.5\pgfphysicalheight,%
	    center=\pgfpoint{.25\pgfphysicalwidth}{.75\pgfphysicalheight}%
	  }%
	  \pgfpageslogicalpageoptions{2}
	  {%
	%   border code=\pgfsetlinewidth{2pt}\pgfstroke,%
	%    border shrink=\pgfpageoptionborder,%
	    resized width=.5\pgfphysicalwidth,%
	    resized height=.5\pgfphysicalheight,%
	    center=\pgfpoint{.75\pgfphysicalwidth}{.75\pgfphysicalheight}%
	  }%
	  \pgfpageslogicalpageoptions{3}
	  {%
	%    border code=\pgfsetlinewidth{2pt}\pgfstroke,%
	%    border shrink=\pgfpageoptionborder,%
	    resized width=.5\pgfphysicalwidth,%
	    resized height=.5\pgfphysicalheight,%
	    center=\pgfpoint{.25\pgfphysicalwidth}{.25\pgfphysicalheight}%
	  }%
	  \pgfpageslogicalpageoptions{4}
	  {%
	%    border code=\pgfsetlinewidth{2pt}\pgfstroke,%
	%    border shrink=\pgfpageoptionborder,%
	    resized width=.5\pgfphysicalwidth,%
	    resized height=.5\pgfphysicalheight,%
	    center=\pgfpoint{.75\pgfphysicalwidth}{.25\pgfphysicalheight}%
	  }%
	}


  \pgfpagesuselayout{4 on 1 b}[a4paper, border shrink=5mm, landscape]
  \nofiles
}





\newcommand{\hkapitola}[1]{
\section{#1}
\begin{frame}
\begin{block}{}
\Huge
\centering
#1
\end{block}
\end{frame}
}

\newcommand{\kapitola}[1]{
\subsection{#1}
\begin{frame}
\begin{block}{}
\Large
\centering
#1
\end{block}
\end{frame}
}

\newcommand{\pkapitola}[1]{
\subsubsection{#1}
\begin{frame}
\begin{block}{}
\large
\centering
#1
\end{block}
\end{frame}
}


\newcommand{\pulsirkycol}{.45\textwidth}
\newcommand{\tretinasirkycol}{.275\textwidth}

\newcommand{\raisesym}[1]{\raisebox{0.5\depth}{#1}}

\newcommand{\no}{\raisesym{$\times$}}
\newcommand{\yes}{\raisesym{\checkmark}}
\newcommand{\warning}{\raisesym{\fontencoding{U}\fontfamily{futs}\selectfont\char 66\relax}}


\newcommand{\NO}{\scriptsize\raisesym{$\times$}}
\newcommand{\YES}{\scriptsize\raisesym{\checkmark}}
\newcommand{\WARNING}{\scriptsize\raisesym{\fontencoding{U}\fontfamily{futs}\selectfont\char 66\relax}}

\newcommand{\dcc}{\color[rgb]{0.35,0.35,0.35}}

\newcommand{\nezkouskove}{%\setbeamertemplate{background canvas}[vertical shading][bottom=violet!5,top=violet!5]
\setbeamertemplate{background canvas}{%
\begin{tikzpicture}
    \clip (0,0) rectangle (\paperwidth,\paperheight);
    \fill[color=violet] (0,\paperheight) rectangle (\paperwidth,\paperheight-5pt);
    \fill[color=violet] (0,0) rectangle (\paperwidth,5pt);

	\fill[color=white] (.90\paperwidth,0) rectangle (\paperwidth,10pt);
\end{tikzpicture}
}
}
%\newcommand{\zkouskove}{\setbeamertemplate{background canvas}[vertical shading][bottom=white,top=structure.fg!5]}
\newcommand{\zkouskove}{\setbeamertemplate{background canvas}[vertical shading][bottom=white,top=white]}

\newenvironment<>{deprecatedblock}[1]{
  \begin{actionenv}#2
    \def\insertblocktitle{#1}
    \par
    \mode<presentation>{
      \setbeamercolor{block title}{fg=white,bg=gray!90!white}
      \setbeamercolor{block body}{fg=black,bg=gray!30}
      \setbeamercolor{itemize item}{fg=gray!20!white}
      \setbeamertemplate{itemize item}[triangle]
    }
    \usebeamertemplate{block begin}}
    {\par\usebeamertemplate{block end}\end{actionenv}}

\newenvironment<>{noteblock}[1]{
  \begin{actionenv}#2
    \def\insertblocktitle{#1}
    \par
    \mode<presentation>{
      \setbeamercolor{block title}{fg=orange!60!black,bg=yellow!60!white}
      \setbeamercolor{block body}{fg=black,bg=yellow!10}
      \setbeamercolor{itemize item}{fg=gray!20!black}
      \setbeamertemplate{itemize item}[triangle]
    }
    \usebeamertemplate{block begin}}
    {\par\usebeamertemplate{block end}\end{actionenv}}

\newenvironment<>{bonusblock}[1]{
  \begin{actionenv}#2
    \def\insertblocktitle{#1}
    \par
    \mode<presentation>{
      \setbeamercolor{block title}{fg=yellow,bg=violet!80!black}
      \setbeamercolor{block body}{fg=black,bg=lime!10!white}
      \setbeamercolor{itemize item}{fg=gray!20!black}
      \setbeamertemplate{itemize item}[triangle]
    }
    \usebeamertemplate{block begin}}
    {\par\usebeamertemplate{block end}\end{actionenv}}



\newenvironment{yesblock}{\begin{exampleblock}{\YES}}{\end{exampleblock}}
\newenvironment{noblock}{\begin{alertblock}{\NO}}{\end{alertblock}}
\newenvironment{oldblock}{\begin{deprecatedblock}{\WARNING}}{\end{deprecatedblock}}

%[totalwidth=\textwidth]
\newenvironment{twocols}{\begin{columns}\begin{column}{\pulsirkycol}}{\end{column}\end{columns}}
\newcommand{\twocolssep}{\end{column}\begin{column}{\pulsirkycol}}

\newenvironment{threecols}{\begin{columns}\begin{column}{\tretinasirkycol}}{\end{column}\end{columns}}
\newcommand{\threecolssep}{\end{column}\begin{column}{\tretinasirkycol}}

\newenvironment{bitemize}[1]{\begin{block}{#1}\begin{itemize}}{\end{itemize}\end{block}}



%%\setbeamercolor{background canvas}{bg=violet}


% Blocks:
% block
% exampleblock <- yesblock
% alertblock <- noblock
% noteblock
% deprecatedblock <- oldblock
% bonusblock
% twocols + cmd twocolssep
% threecols + cmd threecolssep
% bitemize - block + itemize
% Commands:
% \NO \YES \WARNING \pulsirkycol
