\hkapitola{Preprocesor}


\begin{frame}[fragile]
\begin{bitemize}{Preprocesor}
\item podobně jako v C++ nabízí C\# preprocesor a několik příkazů, které se provedou před vlastní kompilací
\item možnosti jsou více omezené než v C++
\begin{itemize}
\item možnost podmíněné kompilace (dle definovaných symbolů preprocesoru)
\item možnost označovat část kódu (\lstinline|#region|) pro větší přehlednost v~editoru
\item možnost vyvolat chybu nebo varování při kompilaci části kódu 
\item možnost změnit číslo řádku nebo název souboru (\lstinline|#line|)
\item další možnosti dle kompilátoru (\lstinline|#pragma|)
\end{itemize}
\end{bitemize}

\end{frame}

\begin{frame}[fragile]
\vfill
\begin{bitemize}{\#define, \#undef}
\item definuje a ruší symboly (konstanty) preprocesoru
\item definice musí být na prvním místě v souboru
\item lze na nich založit podmíněnou kompilaci
\end{bitemize}
\vfill
\begin{yesblock}
\begin{lstlisting}
#define DEBUG
#undef TRACE
\end{lstlisting}
\end{yesblock}
\vfill
\end{frame}




\begin{frame}[fragile]
\vfill
\begin{bitemize}{\#if, \#elif, \#else, \#endif}
\item podmínkový blok v preprocesoru
\end{bitemize}
\vfill
\begin{yesblock}
\begin{lstlisting}
#if (DEBUG && !MYTEST)
    Console.WriteLine("DEBUG is defined");
#elif (!DEBUG && MYTEST)
    Console.WriteLine("MYTEST is defined");
#elif (DEBUG && MYTEST)
    Console.WriteLine("DEBUG and MYTEST are defined");  
#else
    Console.WriteLine("DEBUG and MYTEST are not defined");
#endif
\end{lstlisting}
\end{yesblock}
\vfill
\end{frame}



\begin{frame}[fragile]
\vfill
\begin{bitemize}{ConditionalAttribute -- podmíněná kompilace pomocí atributu}
\item lze aplikovat na metody a třídy (pouze atributy)
\item pokud není podmínka splněna, daný objekt se nezkompiluje
\end{bitemize}
\vfill
\begin{yesblock}
\begin{lstlisting}
[Conditional("CONDITION1")]
public static void Method1(int x)
{
    Console.WriteLine("CONDITION1 is defined");
}

[Conditional("CONDITION1"), Conditional("CONDITION2")]  
public static void Method2()
{
    Console.WriteLine("CONDITION1 or CONDITION2 is defined");
}
\end{lstlisting}
\end{yesblock}
\vfill
\end{frame}



\begin{frame}[fragile]
\vfill
\begin{bitemize}{\#warning, \#error}
\item vyvolá varování nebo error v době kompilace na dané řádce
\end{bitemize}
\vfill
\begin{yesblock}
\begin{lstlisting}
#define DEBUG  
class MainClass   
{  
    static void Main()   
    {  
#if DEBUG  
#warning DEBUG is defined  
#endif  
    }  
}  
\end{lstlisting}
\end{yesblock}
\vfill
\end{frame}


\begin{frame}[fragile]
\vfill
\begin{bitemize}{\#region, \#endregion}
\item označuje blok kódu, který spolu logicky souvisí
\item ve VS lze takový blok naráz zobrazit/skrýt pro větší přehlednost
\end{bitemize}
\vfill
\begin{yesblock}
\begin{lstlisting}
#region MyClass definition  
public class MyClass   
{  
    static void Main()   
    {  
    }  
}  
#endregion  
\end{lstlisting}
\end{yesblock}
\vfill
\end{frame}

