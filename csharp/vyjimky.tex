
\hkapitola{Výjimky}

\begin{frame}[fragile]

\begin{bitemize}{Výjimky}
\item slouží k ošetření chybových stavů v programu
\item systém je velmi podobný Javě
\item[]
\item vyvolání výjimky \lstinline|throw ...|
\item zachycení a zpracování výjimky \lstinline|try - catch - finally|
\item výjimky odvozeny od základní třídy \lstinline|System.Exception|
\item[]
\item nezachycená výjimka se šíří z metody dále
\begin{itemize}
\item na nejvyšší úrovni způsobí pád programu s chybovým hlášením
\end{itemize}

\end{bitemize}

\end{frame}




\begin{frame}[fragile]
\vfill
\begin{bitemize}{Vyvolání výjimky}
\item \lstinline|throw objektVýjimky|
\end{bitemize}
\vfill
\begin{yesblock}
\begin{lstlisting}
static void ComplexCalculation()
{
    throw new InvalidOperationException("broken toe");
}
\end{lstlisting}
\end{yesblock}
\vfill
\end{frame}



\begin{frame}[fragile]
\begin{bitemize}{Zachycení výjimky}
\item \lstinline|catch (typVýjimky názevProměnné)|
\end{bitemize}
\vfill
\begin{yesblock}
\begin{lstlisting}
try
{
    Console.WriteLine("pre throw");
    ComplexCalculation();
    Console.WriteLine("post throw");
}
catch (Exception ex)
{
    Console.Error.WriteLine(ex.Message);
}
\end{lstlisting}
\end{yesblock}
\end{frame}



\begin{frame}[fragile]
\begin{bitemize}{Zachycení výjimky -- univerzální catch}
\item \lstinline|catch|
\end{bitemize}
\vfill
\begin{yesblock}
\begin{lstlisting}
try
{
    Console.WriteLine("pre throw");
    ComplexCalculation();
    Console.WriteLine("post throw");
}
catch 
{
    Console.Error.WriteLine("catched something");
}
\end{lstlisting}
\end{yesblock}
\end{frame}


\begin{frame}[fragile]
\begin{bitemize}{Blok finally}
\item \lstinline|finally| -- nepovinné, blok je proveden vždy
\end{bitemize}
\vfill
\begin{yesblock}
\begin{lstlisting}
try
{
    Console.WriteLine("pre throw");
    ComplexCalculation();
    Console.WriteLine("post throw");
}
catch (Exception ex)
{
    Console.Error.WriteLine(ex.Message);
}
finally
{
    Console.WriteLine("finally");
}
\end{lstlisting}
\end{yesblock}
\end{frame}





\begin{frame}[fragile]
\frametitle{Třída System.Exception}
\begin{noteblock}{}
\begin{lstlisting}
public class Exception : ISerializable, _Exception
{
    public Exception();
    public Exception(string message);
    public Exception(string message, Exception innerException);

    public virtual string Source { get; set; }
    public virtual string HelpLink { get; set; }
    public virtual string StackTrace { get; }
    public MethodBase TargetSite { get; }
    public Exception InnerException { get; }
    public virtual string Message { get; }
    public int HResult { get; protected set; }
    public virtual IDictionary Data { get; }
    
    // ...
}
\end{lstlisting}
\end{noteblock}

\end{frame}



\begin{frame}[fragile]
\frametitle{Potomci třídy System.Exception}
\begin{noteblock}{}
\begin{itemize}
\item \lstinline|SystemException|
\item \lstinline|ApplicationException|
\item[]
\item \lstinline|IndexOutOfRangeException|
\item \lstinline|NullReferenceException|
\item \lstinline|AccessViolationException|
\item \lstinline|InvalidOperationException|
\item \lstinline|ArgumentException|
\item \lstinline|ArgumentNullException|
\item \lstinline|ArgumentOutOfRangeException|
\item \lstinline|ArithmeticException|
\item \lstinline|IndexOutOfRangeException|
\item \ldots
\end{itemize}
\end{noteblock}

\end{frame}




\begin{frame}[fragile]
\begin{bitemize}{Řízené použití prostředků pomocí try -- finally}
\item konstrukci \lstinline|try -- finally| lze využít pro řízený přístup k prostředkům
\item blok \lstinline|finally| je proveden i v případě výskytu libovolné výjimky uvnitř \lstinline|try| bloku a je tak možné uklidit prostředky
\item rozhraní \lstinline|IDisposable| slouží k uvolnění těchto prostředků
\end{bitemize}
\vfill
\begin{yesblock}
\begin{lstlisting}
{
	Font font1 = new Font("Arial", 10.0f);
	try
	{
		byte charset = font1.GdiCharSet;
	}
	finally
	{
		if (font1 != null)
			((IDisposable)font1).Dispose();
	}
}
\end{lstlisting}
\end{yesblock}
\end{frame}


\begin{frame}[fragile]
\vfill
\begin{bitemize}{Řízené použití prostředků pomocí try -- finally}
\item pro zjednodušení C\# nabízí ekvivalentní konstrukci \lstinline|using(...) { ... }|
\end{bitemize}
\vfill
\begin{yesblock} 
\begin{lstlisting}
using (Font font1 = new Font("Arial", 10.0f))
{
    byte charset = font1.GdiCharSet;
}
\end{lstlisting}
\end{yesblock}
\vfill
\begin{bitemize}{}
\item \lstinline|using| lze vnořovat nebo lze deklarovat více řízených proměnných najednou
\end{bitemize}
\vfill
\begin{yesblock}
\begin{lstlisting}
using (Font font3 = new Font("Arial", 10.0f),
            font4 = new Font("Arial", 10.0f))
{
    // Use font3 and font4.
}
\end{lstlisting}
\end{yesblock}
\end{frame}


\begin{frame}[fragile]
\begin{bitemize}{Kdy používat using}
\item nad objekty realizující rozhraní \lstinline|IDisposable|
\item \lstinline|System.IO, System.Net| -- třídy realizující operace se soubory/síťovými proudy
\item \lstinline|System.Data| -- třídy realizující databázové připojení a operace
\item \lstinline|System.Drawing| -- třídy popisující grafické objekty
\item \lstinline|System.Windows.Forms.Control| -- WinForms prvky
\item \ldots
\end{bitemize}
\end{frame}


