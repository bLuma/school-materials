\hkapitola{Jmenné prostory}

\begin{frame}[fragile]
\frametitle{Jmenné prostory}
\vfill
\begin{bitemize}{namespace}
\item stejně jako balíčky slouží k logickému oddělení kódu -- rozdělení na spolu související komponenty a \textbf{zamezení konfliktů jmen}
\item jmenné prostory na rozdíl od Java balíčků \textbf{nevyžadují přesnou organizaci na disku}
\item jmenné prostory je možné \textbf{libovolně vnořovat}
\end{bitemize}
\vfill
\begin{bitemize}{}
\item uvnitř jmenného prostoru jsou vidět složky definované v tomto jmenném prostoru
\item složky definované v jiném jmenném prostoru je nutné zpřístupnit užitím plné cesty a názvu komponenty nebo zpřístupnit jmenný prostor pomocí \lstinline|using|
\end{bitemize}
\vfill
\end{frame}



\begin{frame}[fragile]
\vfill
\begin{bitemize}{Definování jmenných prostorů}
\item jmenný prostor je možné definovat na globální úrovni nebo v jiném jmenném prostoru
\item [] \lstinline|namespace NázevJmennéhoProstoru { ... }|
\end{bitemize}
\vfill
\begin{yesblock}
\begin{lstlisting}
namespace Game
{
    // Game ...

    namespace Graphics
    {
        // Game.Graphics ...
    }
}
\end{lstlisting}
\end{yesblock}
\vfill
\end{frame}





\begin{frame}[fragile]
\vfill
\begin{bitemize}{Vnořené jmenné prostory}
\item u vnořených jmenných prostorů není nutné vytvářet N vnořených bloků \lstinline|namespace|, ale postačí uvést plnou cestu k jmennému prostoru 
\end{bitemize}
\vfill
\begin{yesblock}
\begin{lstlisting}
namespace Game.Graphics
{
    // Game.Graphics ...
}
\end{lstlisting}
\end{yesblock}
\end{frame}




\begin{frame}[fragile]
\frametitle{Použití jmenných prostorů -- definice}

\begin{yesblock}
\begin{lstlisting}
namespace First
{
    class FirstClass
    {
        public const int FirstConst = 1;
    }
}
\end{lstlisting}
\end{yesblock}
\end{frame}




\begin{frame}[fragile]
\frametitle{Použití jmenných prostorů -- zpřístupnění jiného jmenného prostoru}
\vfill
\begin{bitemize}{}
\item standardně je nutné uvést plnou \uv{cestu} k dané složce
\end{bitemize}
\vfill
\begin{yesblock}
\begin{lstlisting}
namespace Second
{
    class SecondClass
    {
        static void Method()
        {
            // FirstConst // error CS0103
            // FirstClass.FirstConst // error CS0103
            var v = First.FirstClass.FirstConst;
        }
    }
}
\end{lstlisting}
\end{yesblock}
\vfill
\end{frame}




\begin{frame}[fragile]
\frametitle{Použití jmenných prostorů -- using}
\vfill
\begin{bitemize}{}
\item pomocí \lstinline|using| lze zpřístupnit obsah jmenného prostoru do aktuálního kontextu
\end{bitemize}
\vfill
\begin{yesblock}
\begin{lstlisting}
using First;

namespace Second
{
    class SecondClass
    {
        static void Method()
        {
            var v = FirstClass.FirstConst;
        }
    }
}
\end{lstlisting}
\end{yesblock}
\vfill
\end{frame}






\begin{frame}[fragile]
\frametitle{Zpřístupnění statických položek -- using static}
\vfill
\begin{bitemize}{}
\item od C\# 6 existuje klauzule \lstinline|using static|, která umožňuje přímo pracovat se statickými složkami vybraného typu
\end{bitemize}
\vfill
\begin{yesblock}
\begin{lstlisting}
using static System.Math;

class ExampleUsingStaticClass
{
    static double CalculateCircleArea(double radius)
    {
        // bez using static:
        // return Math.PI * Math.Pow(radius, 2);
        
        return PI * Pow(radius, 2);
    }
}
\end{lstlisting}
\end{yesblock}
\vfill
\end{frame}