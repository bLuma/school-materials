%BinaryReader - stream
%writer - stream
%bufferedstream - stream
%
%filestream
%memorystream
%
%streamreader - stream
%streamwriter - stream
%
%stringreader/writer - string
%
%Stream - abstract
%textreader/writer - abtstract class
%
%File,FileInfo
%Directory,DirectoryInfo
%Path

\hkapitola{Input/output}


\kapitola{Souborový systém}


\begin{frame}[fragile]
\begin{bitemize}{Abstrakce souborového systému}
\item Jmenný prostor \lstinline|System.IO|
\item Základní třídy:
\begin{itemize}
\item \lstinline|File| -- statická třída pro práci se soubory
\item \lstinline|Path| -- statická třída pro práci s cestami
\item \lstinline|Directory| -- statická třída pro práci s adresáři
\item[]
\item \lstinline|FileInfo| -- objekt představuje konkrétní soubor
\item \lstinline|DirectoryInfo| -- objekt představuje konkrétní adresář
\item \lstinline|DriveInfo| -- objekt představuje konkrétní jednotku

\end{itemize}

\end{bitemize}
\end{frame}




\begin{frame}[fragile]
\begin{yesblock}
\begin{lstlisting}
if (File.Exists(@".\file.txt"))
{
    // ...
}

FileInfo file = new FileInfo(@".\file.txt");
if (file.Exists)
{
    // ...
}
\end{lstlisting}
\end{yesblock}
\vfill
\begin{yesblock}
\begin{lstlisting}
FileStream fs = File.Create(@".\created.txt");
fs.Close();

FileStream fsInstance = new FileInfo(@".\created.txt").Create();
fsInstance.Close();
\end{lstlisting}
\end{yesblock}
\end{frame}




\begin{frame}[fragile]
\begin{yesblock}
\begin{lstlisting}
DirectoryInfo di = new DirectoryInfo(@".\");
foreach (FileInfo file in di.EnumerateFiles())
{
    Console.WriteLine($"{file.Name} - {file.Extension} - {file.LastAccessTime}");
}

/*
ConsoleApp1.exe - .exe - 05.02.2018 22:07:58
ConsoleApp1.exe.config - .config - 05.02.2018 22:07:58
ConsoleApp1.pdb - .pdb - 05.02.2018 22:07:58
created.txt - .txt - 11.03.2018 10:23:01
 */
\end{lstlisting}
\end{yesblock}
\vfill
\begin{yesblock}
\begin{lstlisting}
DirectoryInfo di = new DirectoryInfo(@".\nonExistentDirectory");
di.Create();

Console.WriteLine(di.FullName);
// C:\Users\<...>\ConsoleApp1\bin\Debug\nonExistentDirectory
\end{lstlisting}
\end{yesblock}
\end{frame}



\begin{frame}[fragile]
\begin{yesblock}
\begin{lstlisting}
DriveInfo di = new DriveInfo("C");

var sizeInGB = di.TotalSize / 1024 / 1024 / 1024;
var freeInGB = di.TotalFreeSpace / 1024 / 1024 / 1024;
var fs = di.DriveFormat;

Console.WriteLine($"{fs} - free {freeInGB} / {sizeInGB} GB");
// NTFS - free 125 / 432 GB
\end{lstlisting}
\end{yesblock}
\vfill
\begin{yesblock}
\begin{lstlisting}
DriveInfo[] drives = DriveInfo.GetDrives();
foreach (DriveInfo drive in drives)
{
    Console.WriteLine($"{drive.Name} - {drive.DriveFormat} - {drive.DriveType}");
}

// C:\ - NTFS - Fixed
// D:\ - NTFS - Fixed
\end{lstlisting}
\end{yesblock}
\end{frame}



\kapitola{Proudy}




\begin{frame}[fragile]
\vfill
\begin{bitemize}{Základní proudy pro práci se soubory, pamětí, \ldots}
\item knihovna využívá stejných principů jako Java
\item základní abstraktní třída \lstinline|Stream| reprezentuje datový proud
\begin{itemize}
\item potomci představují konkrétní použití -- \lstinline|FileStream|, \lstinline|MemoryStream|, \lstinline|NetworkStream|, \ldots
\item existuje zde řada dekorátorů -- \lstinline|BufferedStream|, \lstinline|DeflateStream|, \lstinline|GZipStream|, \lstinline|CryptoStream|, \ldots
\end{itemize}
\item pro realizaci textových přenosů slouží třídy \lstinline|StreamReader|, \lstinline|StreamWriter|
\begin{itemize}
\item binární přenosy \lstinline|BinaryReader|, \lstinline|BinaryWriter|
\end{itemize}
\end{bitemize}
\vfill
\begin{bitemize}{}
\item proudy je třeba korektně uzavírat, jinak může dojít ke ztrátě dat a prostředky mohou být dlouho blokovány
\item proudy typicky realizují \lstinline|IDisposable| rozhraní (lze využít konstrukci \lstinline|using|)
\end{bitemize}
\vfill
\end{frame}



\begin{frame}[fragile]
\vfill
\begin{bitemize}{Abstraktní třída Stream}
\item \lstinline|Position| -- vlastnost, pozice kurzoru
\item \lstinline|CanRead, CanWrite, CanSeek| -- vlastnosti
\item \lstinline|Seek| -- přesune kurzor v souboru
\item \lstinline|Read| -- čte pole bajtů
\item \lstinline|Write| -- zapisuje pole bajtů
\item \lstinline|ReadByte| -- čte bajt
\item \lstinline|WriteByte| -- zapisuje bajt
\item \lstinline|Close| -- uzavře proud
\end{bitemize}
\vfill
\begin{bitemize}{Potomci třídy Stream}
\item \lstinline|BaseStream| -- vlastnost, není definováno rozhraním, nemusí být u~všech
\end{bitemize}
\vfill
\end{frame}


\pkapitola{Zápis a čtení textových dat}


\begin{frame}[fragile]
\vfill
\begin{bitemize}{Zápis a čtení textových dat}
\item Třídy \lstinline|StreamWriter|, \lstinline|StreamReader|
\item pracují nad obecným proudem (soubor, paměť, síť, \ldots)
\item kódování znaků je volitelné
\begin{itemize}
\item výchozí kódování je UTF-8
\end{itemize}
\end{bitemize}
\vfill
\begin{bitemize}{Základní metody}
\item \lstinline|Write()| -- zápis znaku/znaků/hodnoty základních dat. typů
\item \lstinline|WriteLine()| -- zápis řádku textu
\item \lstinline|Read()| -- načtení znaku nebo pole znaků
\item \lstinline|ReadLine()| -- načtení řádku
\end{bitemize}
\vfill
\end{frame}




\begin{frame}[fragile]
\vfill
\begin{bitemize}{StreamWriter -- konstrukce}
\item \lstinline|StreamWriter(Stream stream)|
\item \lstinline|StreamWriter(string path)|
\item \lstinline|StreamWriter(Stream stream, Encoding encoding)|
\item \lstinline|StreamWriter(string path, bool append)|
\item \ldots
\end{bitemize}
\vfill
\begin{bitemize}{StreamReader -- konstrukce}
\item \lstinline|StreamReader(Stream stream)|
\item \lstinline|StreamReader(Stream stream, bool detectEncodingFromByteOrderMarks)|
\item \lstinline|StreamReader(Stream stream, Encoding encoding)|
\item \lstinline|StreamReader(Stream stream, Encoding encoding, bool detectEncodingFromByteOrderMarks)|
\item dtto se \lstinline|string path| \ldots
\end{bitemize}
\vfill
\end{frame}



\begin{frame}[fragile]
\frametitle{Zápis a čtení z textového souboru}
\begin{yesblock}
\begin{lstlisting}
using (StreamWriter writer = new StreamWriter(new FileStream("output.txt", FileMode.OpenOrCreate), Encoding.UTF8))
{
    writer.WriteLine("Hello World");
    writer.WriteLine("Příliš žluťoučký kůň úpěl ďábelské ódy");
}

using (var reader = new StreamReader("output.txt"))
{
    Console.WriteLine(reader.ReadLine());
    Console.WriteLine(reader.ReadLine());
}
\end{lstlisting}
\end{yesblock}
\end{frame}


\begin{frame}[fragile]
\frametitle{Základní metody StreamReader, StreamWriter}
\vfill
\begin{yesblock}
\begin{lstlisting}
writer.Write(123);
writer.Write(123.456);
writer.Write(true);
writer.Write(obj);
writer.WriteLine(dtto);
\end{lstlisting}
\end{yesblock}
\vfill
\begin{yesblock}
\begin{lstlisting}
int char = reader.Read();

char[] buffer = new char[100];
reader.Read(buffer, 0, buffer.Length);

string line = reader.ReadLine();

string tillTheEnd = reader.ReadToEnd();
\end{lstlisting}
\end{yesblock}
\vfill
\end{frame}






\begin{frame}[fragile]
\begin{bitemize}{Kódování textu}
\item Třída \lstinline|System.Text.Encoding|
\item Vlastnosti:
\begin{itemize}
\item \lstinline|ASCII| - ASCII kódování, bez podpory rozšířených znaků
\item \lstinline|BigEndianUnicode| -- UTF-16 Big Endian
\item \lstinline|Default| -- výchozí pro aktuální platformu (Windows - cp1250/windows-1250/iso-8859-2)
\item \lstinline|Unicode| -- UTF-16 Little Endian
\item \lstinline|UTF32| -- UTF-32 Little Endian
\item \lstinline|UTF7| -- UTF-7
\item \lstinline|UTF8| -- UTF-8
\end{itemize}
\item Statické metody
\begin{itemize}
\item \lstinline|Convert(...)| -- převede pole bajtů ze zadaného kódování do cílového
\item \lstinline|GetEncoding(...)| -- vrátí kódování podle označení nebo čísla kódové stránky
\item \lstinline|GetEncodings()| -- vrátí pole všech použitelných kódování
\end{itemize}

\end{bitemize}
\end{frame}



\begin{frame}[fragile]
\frametitle{Kódování textu}
\begin{yesblock}
\begin{lstlisting}
// zápis v UTF-16 Little Endian
using (StreamWriter sw = new StreamWriter(File.OpenWrite("unicode.txt"), Encoding.Unicode)) {
    sw.WriteLine("Příliš žluťoučký kůň úpěl dábělské ódy");
}

// false - nedetekuj kódování z BOM hlavičky souboru
// provede čtení v UTF-8
using (StreamReader sr = new StreamReader(File.OpenRead("unicode.txt"), false))
{
    Console.WriteLine(sr.ReadLine());
    // ??P Y?? l i a?  ~?l u e?o u
}
\end{lstlisting}
\end{yesblock}
\end{frame}



\pkapitola{Zápis a čtení binárních dat}



\begin{frame}[fragile]
\begin{bitemize}{Binární soubory}
\item pro práci s binárními soubory slouží třídy \lstinline|BinaryWriter|, \lstinline|BinaryReader|
\item nabízejí metody pro čtení a zápis všech primitivních datových typů
\item[]
\item C\# dále nabízí možnosti serializace nebo i možnost práce s unmanaged memory

\end{bitemize}
\end{frame}





\begin{frame}[fragile]
\frametitle{Zápis a čtení z binárních souborů}
\begin{yesblock}
\begin{lstlisting}
using (var writer = new BinaryWriter(new FileStream("binary.dat", FileMode.Create)))
{
    // write je přetížena pro všechny základní typy
    writer.Write(true);
    writer.Write(0x11223344);
    writer.Write(123.456);
    writer.Write('z');
    writer.Write("string");

    // a podporován je i blokový přenos pomocí pole bajtů
    byte[] buffer = new byte[16];
    writer.Write(buffer, 0, buffer.Length);
}
\end{lstlisting}
\end{yesblock}
\end{frame}


\begin{frame}[fragile]
\frametitle{Zápis a čtení z binárních souborů}
\begin{yesblock}
\begin{lstlisting}
using (var reader = new BinaryReader(new FileStream("binary.dat", FileMode.Open)))
{
    bool boolVar = reader.ReadBoolean();
    int intVar = reader.ReadInt32();
    double doubleVar = reader.ReadDouble();
    char charVar = reader.ReadChar();
    string strVar = reader.ReadString();

    byte[] buffer = new byte[16];
    reader.Read(buffer, 0, buffer.Length);
}
\end{lstlisting}
\end{yesblock}
\end{frame}



\nezkouskove

\begin{frame}[fragile]
\vfill
\begin{bonusblock}{~}
\begin{itemize}
\item ukázka použití unmanaged memory a marshallingu
\end{itemize}
\end{bonusblock}
\vfill
\begin{yesblock}
\begin{lstlisting}
// třída nemá ve výchozím chování pevnou strukturu, proto je třeba ji označit atributem
[StructLayout(LayoutKind.Sequential, CharSet = CharSet.Unicode)]
class Student
{
    // rovněž je možné specifikovat konkrétní způsob marshallingu jednotlivých datových složek
    [MarshalAs(UnmanagedType.ByValTStr, SizeConst = 50)]
    private string name;
    [MarshalAs(UnmanagedType.ByValTStr, SizeConst = 50)]
    private string lastName;
    [MarshalAs(UnmanagedType.I4)]
    private int id;

    // ...
}
\end{lstlisting}
\end{yesblock}
\vfill
\end{frame}


\begin{frame}[fragile]
\begin{yesblock}
\begin{lstlisting}
using (FileStream fs = File.OpenWrite("binary.dat"))
{
    Student student = new Student()
    {
        Name = "Petr",
        LastName = "Rychly",
        Id = 1234
    };

    var size = Marshal.SizeOf(student); // velikost struktury

    var ptr = Marshal.AllocHGlobal(size); // alokace unmanaged pole
    Marshal.StructureToPtr(student, ptr, false); 

    var bytes = new byte[size]; // managed pole pro uložení struktury
    Marshal.Copy(ptr, bytes, 0, size); // copy unmanaged -> managed
    Marshal.FreeHGlobal(ptr); // uvolnění pole

    fs.Write(bytes, 0, size);
}
\end{lstlisting}
\end{yesblock}
\end{frame}


\begin{frame}[fragile]
\begin{yesblock}
\begin{lstlisting}
using (FileStream fs = File.OpenRead("binary.dat"))
{
    Student student = new Student();

    var size = Marshal.SizeOf<Student>();

    var bytes = new byte[size];
    fs.Read(bytes, 0, size);

    var ptr = Marshal.AllocHGlobal(size);
    Marshal.Copy(bytes, 0, ptr, size);
    Marshal.PtrToStructure(ptr, student);
    Marshal.FreeHGlobal(ptr);

    Console.WriteLine($"{student.Name} {student.LastName} {student.Id}");
}
\end{lstlisting}
\end{yesblock}
\end{frame}


\zkouskove


\pkapitola{Paměťové proudy}


\begin{frame}[fragile]
\begin{bitemize}{Paměťové proudy}
\item realizovány třídou \lstinline|MemoryStream|
\item na pozadí používají pole bajtů (\lstinline|byte[]|), může být fixní délky nebo rozšířitelné
\item podporuje základní \lstinline|Write()| a \lstinline|Read()| operace
\item po ukončení zápisu lze pole získat pomocí met. \lstinline|ToArray()| nebo zapsat obsah do jiného proudu pomocí \lstinline|WriteTo(stream)|
\end{bitemize}
\vfill
\begin{yesblock}
\begin{lstlisting}
using (MemoryStream ms = new MemoryStream())
{
    for (int i = 0; i < 256; i++)
        ms.WriteByte((byte)i);

    using (FileStream fs = File.OpenWrite("bytes.dat"))
    {
        ms.WriteTo(fs);
    }
}
\end{lstlisting}
\end{yesblock}
\end{frame}