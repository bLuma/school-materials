\documentclass{beamer}
\usepackage[utf8]{inputenc}

\newcommand{\listingssize}{\normalsize}

%\newcommand{\titulek}{C\# - Úvod, datové typy, základní konstrukce}
\newcommand{\titulek}{C\# - Objektové typy}
\newcommand{\autor}{Ing. Roman Diviš}

\newcommand{\predmet}{\titulek}

\usepackage[czech]{babel} 
%\usepackage[IL2,T1,T2A]{fontenc} 
\usepackage[T1]{fontenc} 

%\usetheme{JuanLesPins}
\usetheme{Boadilla}

%%%%%%%%%%%%%%%%%%%%%%%%%%%%%%%%%%%%%%%%%%%%%%%%%%%
\setbeamertemplate{footline}[frame number]{}
\setbeamertemplate{navigation symbols}{}
\setbeamertemplate{blocks}[rounded][shadow=false]
%\setbeamertemplate{blocks}[default]
%\setbeamertemplate{background canvas}[vertical shading][bottom=white,top=structure.fg!5]

% zrušení mezer mezi bloky
\setlength{\medskipamount}{0pt}
\setlength{\smallskipamount}{0pt}

% pro vkládání obrázků
\usepackage{graphicx} 
% pro použití víceřádkových komentářů
\usepackage{verbatim} 
% H modifikátor figure
\usepackage{float}
% barvičky
\usepackage{xcolor,colortbl}
\usepackage{subfig}


\newcommand{\rc}{\cellcolor{red!25}}
\newcommand{\bc}{\cellcolor{blue!25}}
\newcommand{\gc}{\cellcolor{gray!25}}
\newcommand{\grc}{\cellcolor{green!25}}

\newcommand{\cpp}[1]{{\footnotesize$^{C++#1}$}}

\newcommand{\thisse}{\color{blue}\textbf{{\tiny this}}$\searrow$}
\newcommand{\thise}{\color{blue}\textbf{{\tiny this}}$\rightarrow$}

\usepackage{tikz}
\usetikzlibrary{arrows,shapes}
\newcommand{\tikzmark}[1]{\tikz[remember picture] \node[coordinate] (#1) {#1};}

% listingy
\usepackage{listings}

\newcommand{\commentcolor}{\color[rgb]{0.133,0.545,0.133}}
\lstset{
  tabsize=2,
  language=matlab,
  basicstyle=\listingssize,% \ttfamily,
  %upquote=true,
  mathescape=false,
  aboveskip=0pt, %{1.5\baselineskip},
  % zrušení nadbytečné mezery za posledním řádkem
  belowskip=0pt,
  columns=fullflexible,
  showstringspaces=false,,
  breaklines=true,
  prebreak = \raisebox{0ex}[0ex][0ex]{\ensuremath{\hookleftarrow}},
  frame=none,
  showtabs=false,
  showspaces=false,
  showstringspaces=false,
  identifierstyle=\ttfamily,
  keywordstyle=\color[rgb]{0,0,1}\bfseries,
  commentstyle=\color[rgb]{0.133,0.545,0.133},
  stringstyle=\color[rgb]{0.627,0.126,0.941},
  language=Java,
  inputencoding=utf8,
  extendedchars=true,
  literate={á}{{\'a}}1 {ã}{{\~a}}1 {é}{{\'e}}1 {ž}{{\v{z}}}1 {ý}{{\'y}}1 {ě}{{\v{e}}}1 {ř}{{\v{r}}}1 {í}{{\'i}}1 {ů}{{\r{u}}}1 {č}{{\v{c}}}1 {ú}{{\'u}}1 {š}{{\v{s}}}1 {ť}{{\v{t}}}1 {Č}{{\v{C}}}1 {Š}{{\v{S}}}1 {ň}{{\v{n}}}1 {Ř}{{\v{R}}}1 {ó}{{\'o}}1 %{\$}{{\$}}1
}

	
%\lstloadlanguages{[11]C++}
%\lstset{language=[11]C++}
\lstset{language=[Sharp]C}
\lstset{morekeywords={var,when}}
\lstset{escapeinside={<@}{@>}}

%\usepackage{pxfonts}
% odkazy v pdf
\usepackage{hyperref}

\author{\autor}
\title{\titulek}
\institute{UPCE/FEI/KST}
\date{}

\hypersetup{
pdftitle={\titulek},
pdfsubject={\predmet},
pdfauthor={\autor}
}









\mode<handout>
{
	\usepackage{pgf}
	\usepackage{pgfpages}

	\pgfpagesdeclarelayout{4 on 1 boxed}
	{
	  \edef\pgfpageoptionheight{\the\paperheight} 
	  \edef\pgfpageoptionwidth{\the\paperwidth}
	  \edef\pgfpageoptionborder{0pt}
	}
	{
	  \pgfpagesphysicalpageoptions
	  {%
	    logical pages=4,%
	    physical height=\pgfpageoptionheight,%
	    physical width=\pgfpageoptionwidth%
	  }
	  \pgfpageslogicalpageoptions{1}
	  {%
	    border code=\pgfsetlinewidth{2pt}\pgfstroke,%
	    border shrink=\pgfpageoptionborder,%
	    resized width=.5\pgfphysicalwidth,%
	    resized height=.5\pgfphysicalheight,%
	    center=\pgfpoint{.25\pgfphysicalwidth}{.75\pgfphysicalheight}%
	  }%
	  \pgfpageslogicalpageoptions{2}
	  {%
	    border code=\pgfsetlinewidth{2pt}\pgfstroke,%
	    border shrink=\pgfpageoptionborder,%
	    resized width=.5\pgfphysicalwidth,%
	    resized height=.5\pgfphysicalheight,%
	    center=\pgfpoint{.75\pgfphysicalwidth}{.75\pgfphysicalheight}%
	  }%
	  \pgfpageslogicalpageoptions{3}
	  {%
	    border code=\pgfsetlinewidth{2pt}\pgfstroke,%
	    border shrink=\pgfpageoptionborder,%
	    resized width=.5\pgfphysicalwidth,%
	    resized height=.5\pgfphysicalheight,%
	    center=\pgfpoint{.25\pgfphysicalwidth}{.25\pgfphysicalheight}%
	  }%
	  \pgfpageslogicalpageoptions{4}
	  {%
	    border code=\pgfsetlinewidth{2pt}\pgfstroke,%
	    border shrink=\pgfpageoptionborder,%
	    resized width=.5\pgfphysicalwidth,%
	    resized height=.5\pgfphysicalheight,%
	    center=\pgfpoint{.75\pgfphysicalwidth}{.25\pgfphysicalheight}%
	  }%
	}
	
	
	\pgfpagesdeclarelayout{4 on 1 b}
	{
	  \edef\pgfpageoptionheight{\the\paperheight} 
	  \edef\pgfpageoptionwidth{\the\paperwidth}
	  \edef\pgfpageoptionborder{0pt}
	}
	{
	  \pgfpagesphysicalpageoptions
	  {%
	    logical pages=4,%
	    physical height=\pgfpageoptionheight,%
	    physical width=\pgfpageoptionwidth%
	  }
	  \pgfpageslogicalpageoptions{1}
	  {%
	%    border code=\pgfsetlinewidth{2pt}\pgfstroke,%
	%    border shrink=\pgfpageoptionborder,%
	    resized width=.5\pgfphysicalwidth,%
	    resized height=.5\pgfphysicalheight,%
	    center=\pgfpoint{.25\pgfphysicalwidth}{.75\pgfphysicalheight}%
	  }%
	  \pgfpageslogicalpageoptions{2}
	  {%
	%   border code=\pgfsetlinewidth{2pt}\pgfstroke,%
	%    border shrink=\pgfpageoptionborder,%
	    resized width=.5\pgfphysicalwidth,%
	    resized height=.5\pgfphysicalheight,%
	    center=\pgfpoint{.75\pgfphysicalwidth}{.75\pgfphysicalheight}%
	  }%
	  \pgfpageslogicalpageoptions{3}
	  {%
	%    border code=\pgfsetlinewidth{2pt}\pgfstroke,%
	%    border shrink=\pgfpageoptionborder,%
	    resized width=.5\pgfphysicalwidth,%
	    resized height=.5\pgfphysicalheight,%
	    center=\pgfpoint{.25\pgfphysicalwidth}{.25\pgfphysicalheight}%
	  }%
	  \pgfpageslogicalpageoptions{4}
	  {%
	%    border code=\pgfsetlinewidth{2pt}\pgfstroke,%
	%    border shrink=\pgfpageoptionborder,%
	    resized width=.5\pgfphysicalwidth,%
	    resized height=.5\pgfphysicalheight,%
	    center=\pgfpoint{.75\pgfphysicalwidth}{.25\pgfphysicalheight}%
	  }%
	}


  \pgfpagesuselayout{4 on 1 b}[a4paper, border shrink=5mm, landscape]
  \nofiles
}





\newcommand{\hkapitola}[1]{
\section{#1}
\begin{frame}
\begin{block}{}
\Huge
\centering
#1
\end{block}
\end{frame}
}

\newcommand{\kapitola}[1]{
\subsection{#1}
\begin{frame}
\begin{block}{}
\Large
\centering
#1
\end{block}
\end{frame}
}

\newcommand{\pkapitola}[1]{
\subsubsection{#1}
\begin{frame}
\begin{block}{}
\large
\centering
#1
\end{block}
\end{frame}
}


\newcommand{\pulsirkycol}{.45\textwidth}
\newcommand{\tretinasirkycol}{.275\textwidth}

\newcommand{\raisesym}[1]{\raisebox{0.5\depth}{#1}}

\newcommand{\no}{\raisesym{$\times$}}
\newcommand{\yes}{\raisesym{\checkmark}}
\newcommand{\warning}{\raisesym{\fontencoding{U}\fontfamily{futs}\selectfont\char 66\relax}}


\newcommand{\NO}{\scriptsize\raisesym{$\times$}}
\newcommand{\YES}{\scriptsize\raisesym{\checkmark}}
\newcommand{\WARNING}{\scriptsize\raisesym{\fontencoding{U}\fontfamily{futs}\selectfont\char 66\relax}}

\newcommand{\dcc}{\color[rgb]{0.35,0.35,0.35}}

\newcommand{\nezkouskove}{%\setbeamertemplate{background canvas}[vertical shading][bottom=violet!5,top=violet!5]
\setbeamertemplate{background canvas}{%
\begin{tikzpicture}
    \clip (0,0) rectangle (\paperwidth,\paperheight);
    \fill[color=violet] (0,\paperheight) rectangle (\paperwidth,\paperheight-5pt);
    \fill[color=violet] (0,0) rectangle (\paperwidth,5pt);

	\fill[color=white] (.90\paperwidth,0) rectangle (\paperwidth,10pt);
\end{tikzpicture}
}
}
%\newcommand{\zkouskove}{\setbeamertemplate{background canvas}[vertical shading][bottom=white,top=structure.fg!5]}
\newcommand{\zkouskove}{\setbeamertemplate{background canvas}[vertical shading][bottom=white,top=white]}

\newenvironment<>{deprecatedblock}[1]{
  \begin{actionenv}#2
    \def\insertblocktitle{#1}
    \par
    \mode<presentation>{
      \setbeamercolor{block title}{fg=white,bg=gray!90!white}
      \setbeamercolor{block body}{fg=black,bg=gray!30}
      \setbeamercolor{itemize item}{fg=gray!20!white}
      \setbeamertemplate{itemize item}[triangle]
    }
    \usebeamertemplate{block begin}}
    {\par\usebeamertemplate{block end}\end{actionenv}}

\newenvironment<>{noteblock}[1]{
  \begin{actionenv}#2
    \def\insertblocktitle{#1}
    \par
    \mode<presentation>{
      \setbeamercolor{block title}{fg=orange!60!black,bg=yellow!60!white}
      \setbeamercolor{block body}{fg=black,bg=yellow!10}
      \setbeamercolor{itemize item}{fg=gray!20!black}
      \setbeamertemplate{itemize item}[triangle]
    }
    \usebeamertemplate{block begin}}
    {\par\usebeamertemplate{block end}\end{actionenv}}

\newenvironment<>{bonusblock}[1]{
  \begin{actionenv}#2
    \def\insertblocktitle{#1}
    \par
    \mode<presentation>{
      \setbeamercolor{block title}{fg=yellow,bg=violet!80!black}
      \setbeamercolor{block body}{fg=black,bg=lime!10!white}
      \setbeamercolor{itemize item}{fg=gray!20!black}
      \setbeamertemplate{itemize item}[triangle]
    }
    \usebeamertemplate{block begin}}
    {\par\usebeamertemplate{block end}\end{actionenv}}



\newenvironment{yesblock}{\begin{exampleblock}{\YES}}{\end{exampleblock}}
\newenvironment{noblock}{\begin{alertblock}{\NO}}{\end{alertblock}}
\newenvironment{oldblock}{\begin{deprecatedblock}{\WARNING}}{\end{deprecatedblock}}

%[totalwidth=\textwidth]
\newenvironment{twocols}{\begin{columns}\begin{column}{\pulsirkycol}}{\end{column}\end{columns}}
\newcommand{\twocolssep}{\end{column}\begin{column}{\pulsirkycol}}

\newenvironment{threecols}{\begin{columns}\begin{column}{\tretinasirkycol}}{\end{column}\end{columns}}
\newcommand{\threecolssep}{\end{column}\begin{column}{\tretinasirkycol}}

\newenvironment{bitemize}[1]{\begin{block}{#1}\begin{itemize}}{\end{itemize}\end{block}}



%%\setbeamercolor{background canvas}{bg=violet}


% Blocks:
% block
% exampleblock <- yesblock
% alertblock <- noblock
% noteblock
% deprecatedblock <- oldblock
% bonusblock
% twocols + cmd twocolssep
% threecols + cmd threecolssep
% bitemize - block + itemize
% Commands:
% \NO \YES \WARNING \pulsirkycol


%%%%%%%%%%%%%%%%%%%%%%%%%%%%%%%%%%%%%%%%%%%%%%%%%%%%%%%%%%%%%%%%%%%%%%%%
\begin{document}

%\section{\titulek}

\begin{frame}
  \titlepage
\end{frame}
\input{layout-obsah}

%\hkapitola{Použitý styl v přednáškách}


\begin{frame}[fragile]
\vfill
\begin{block}{Základní informace} 
Lorem ipsum dolor sit amet, consectetur adipiscing elit.
\end{block}
\vfill
\begin{exampleblock}{Příklad}
Nullam mattis efficitur aliquam.
\end{exampleblock}
\vfill
\begin{alertblock}{Chyba / příklad s chybou / nekorektní použití} 
Sed aliquam iaculis massa, vel tincidunt lacus tincidunt eget.
\end{alertblock}
\vfill
\begin{noteblock}{Poznámka (teorie, vhodné k zapamatování)}
Proin porta urna ut ipsum ornare, a ultricies elit dictum.
\end{noteblock}
\vfill
\begin{deprecatedblock}{Deprecated (staré, už nepoužívané)} 
Pellentesque habitant morbi tristique senectus et netus et malesuada fames. 
\end{deprecatedblock}
\vfill
\begin{bonusblock}{Bonus (nepotřebujete ke zkoušce, ale souvisí s tématem)}
Sed imperdiet pharetra est, sed ullamcorper neque. 
\end{bonusblock}
\vfill
\end{frame}


\begin{frame}[fragile]
\frametitle{Ukázka}
\vfill
\begin{block}{Ukazatele a dynamická paměť}
Ukázky korektní a nekorektní práce s ukazateli a dynamicky alokovanou pamětí:
\end{block}
\vfill
\begin{twocols}

\begin{exampleblock}{\YES\,Good}
\begin{lstlisting}
int* pointer = nullptr;
pointer = new int;
*pointer = 123;
\end{lstlisting}
\end{exampleblock}
\twocolssep
\begin{alertblock}{\NO\,Bad}
\begin{lstlisting}
int* pointer = 0xdeadbeef;
*pointer = 123;
\end{lstlisting}
\end{alertblock}

\end{twocols}
\vfill
\begin{deprecatedblock}{\WARNING\,deprecated}
\begin{lstlisting}
int* pointer = (int*)malloc(sizeof(int));
*pointer = 123;
\end{lstlisting}
\end{deprecatedblock}
\vfill
\end{frame}



\begin{frame}[fragile]
\frametitle{Definice kódu}
\vfill
\begin{noteblock}{}
\begin{lstlisting}[basicstyle=\small]
struct|class nazevDatovehoTypu [final] [dědičnost] {
  [složky - atributy, metody, vnořené typy]...
} [objekty];

[dědičnost]:
  : [viditelnost] [virtual] předek1, ...
\end{lstlisting}
\end{noteblock}
\vfill
\begin{bitemize}{Definuje:}
\item Začínáme klíčovým slovem \lstinline|struct| nebo \lstinline|class|
\item dále uvedeme název datového typu (\lstinline|Pes|, \lstinline|Kocka|, \lstinline|Student|, \ldots)
\item dále může být (ale nemusí) klíčové slovo \lstinline|final|
\item dále může být definováno dědění z předků
\item[]
\item Uvnitř třídy je možné definovat větší množství složek (\ldots)
\end{bitemize}
\vfill
\end{frame}








\nezkouskove
\begin{frame}[fragile]
\begin{center}
\large Slidy označené fialovými pruhy obsahují téma, které není vyžadováno u zkoušky a zápočtu.
\end{center}
\end{frame}
\zkouskove

%
\hkapitola{Datové typy, výrazy, základní konstrukce}

\begin{frame}[fragile]
%\footnotesize
\begin{tabular}{ccc}
\textbf{Java} & \textbf{C/C++} & \textbf{C\#} \\
\hline
\multicolumn{3}{c}{\textbf{Výstup}} \\
bytecode & nativní kód & MSIL (bytecode) \\
\hline
\multicolumn{3}{c}{\textbf{Správa paměti}} \\
garbage collector & ručně & garbage collector \\
reference & ukazatele, reference & ukazatele, reference \\
\hline
\multicolumn{3}{c}{\textbf{Objektový model}} \\
jednoduchá dědičnost & vícenásobná dědičnost & jednoduchá dědičnost \\
& & vlastnosti, indexery, \\
& & delegáty, události \\
\hline
\multicolumn{3}{c}{\textbf{Přetěžování operátorů}} \\
\no & \yes & \yes \\
\hline
\multicolumn{3}{c}{\textbf{Organizace modulů}} \\
package & namespace & assembly,namespace \\
Maven, Gradle & \ldots & NuGet \\

\hline
\multicolumn{3}{c}{\textbf{Generické programování}} \\
genericita (type erasure) & šablony & genericita (reification) \\

\hline
\multicolumn{3}{c}{\textbf{Knihovna jazyka}} \\
java.lang, java.util, \ldots & C/C++, STL & java like \\
\end{tabular}
\end{frame}

\begin{frame}[fragile]
\begin{center}
\includegraphics[height=\textheight]{img/dotnet_framework_stack.png}
\end{center}
\end{frame}

\begin{frame}[fragile]
\begin{center}
\includegraphics[width=\textwidth]{img/msil_code.jpg}
\end{center}
\end{frame}


\begin{frame}[fragile]
\begin{center}
\includegraphics[width=\textwidth]{img/net_standard.png}
\end{center}
\end{frame}

\begin{frame}[fragile]
\begin{bitemize}{C\#}
\item kompilovaný jazyk (MSIL - mezikód)
\begin{itemize}
\item do nativního kódu je převeden runtime pomocí JIT
\end{itemize}
\item statické (i dynamické) typování
\item správa dynamické paměti pomocí garbage collectoru
\item základní knihovna podobná Javovské
\item inspirován Javou, C/C++, \ldots
\item aktuální verze C\# 7.2
\item standard a některé komponenty uvolněny pod open source licencemi
\item podpora pro non-Windows platformy pomocí Mono, nově .NET Core

\end{bitemize}
\end{frame}


\begin{frame}[fragile]
\begin{exampleblock}{Hello world!}
\begin{lstlisting}[basicstyle=\small]
using System;
using System.Collections.Generic;
using System.Linq;
using System.Text;
using System.Threading.Tasks;

namespace ConsoleApplication1
{
    class Program
    {
        static void Main(string[] args)
        {
            Console.WriteLine("Hello World!");
        }
    }
}
\end{lstlisting}
\end{exampleblock}
\end{frame}


\begin{frame}[fragile]
\begin{exampleblock}{Hello world! a dokumentační komentáře}
\begin{lstlisting}[basicstyle=\small]
namespace ConsoleApplication1
{
    /// <summary>
    /// Hlavni program.
    /// </summary>
    class Program
    {
        /// <summary>
        /// Main metoda
        /// </summary>
        /// <param name="args">argumenty prikazove radky</param>
        static void Main(string[] args)
        {
            Console.WriteLine("Hello World!");
        }
    }
}
\end{lstlisting}
\end{exampleblock}
\end{frame}

%\pkapitola{Coding conventions}

\begin{frame}[fragile]
\vfill
\begin{bitemize}{Coding conventions}
\item C\# definuje řadu doporučení a pravidel, jakým stylem by měl být psán a formátován kód
\item Základní doporučení:
\begin{itemize}
\item \color{green!50!black}VeřejnéTypy, Třídy, Struktury, Metody, Vlastnosti, \ldots začínají velkým písmenem, používá se PascalCase
\item \color{blue}privátníAtributy, parametryMetod, lokálníProměnné, \ldots začínají malým písmenem, používá se camelCase
\item \color{red}nepoužívá se hungarian notation, prefix \lstinline|m_|, \ldots
\end{itemize}

\end{bitemize}
\vfill
\begin{yesblock}
\begin{lstlisting}
class CarFactory { }
struct StudentRecord { }

public void GetElementAt(int elementIndex) { }

private string name;
\end{lstlisting}
\end{yesblock}
\vfill
\end{frame}



\kapitola{Proměnné, datové typy}


\begin{frame}[fragile]
\vfill
\begin{block}{Primitivní datové typy, proměnné}
\begin{itemize}
\item existují hodnotové a referenční datové typy
\begin{itemize}
\item hodnotové typy jsou při předávání do metod nebo ve výrazech typu "\lstinline|promenna = puvodniHodnota|" předány/vytvořeny jako kopie původní hodnoty -- vzniká nová instance (objekt) daného typu se stejnou hodnotou (stavem)
\item referenční typy jsou předány jako reference, tj. existuje jediná instance daného objektu a manipulace s jejich stavem z nového i původního umístění mění jeden a ten samý objekt
\end{itemize}

\item lze používat reference (na hodnotové typy, \lstinline|ref|)
\item lze používat ukazatele (\lstinline|unsafe|)
\item lokální proměnné se pojmenovávají dle camelCase (pocetZivotu, maximalniVykonMotoru)
\end{itemize}
\end{block}
\vfill
\begin{noteblock}{}
\begin{lstlisting}
nazevDatovehoTypu nazevPromenne [= inicializacniHodnota];
\end{lstlisting}
\end{noteblock}
\vfill
\end{frame}




\begin{frame}[fragile]
\begin{exampleblock}{Primitivní datové typy}
\begin{lstlisting}
// celá čísla
int x = 123;
// znakový typ
char c = 'c';
// desetinné typy
double d = 3.141592;
float f = 3.141592f;
// řetězec (není primitivní typ)
string s = "hello world";
// logický typ
bool b = true;
// desitkový desetinný typ
decimal dec = 3.333M;
           
Console.WriteLine("{0} {1} {2} {3} {4} {5} {6} {7} {8}", x, c, d, f, dec, va, nullablex, s, b);
Console.WriteLine($"{x}, {c}, {d}, {f}, {s}, {b}, {dec}");
\end{lstlisting}
\end{exampleblock}
\end{frame}

%// var - odvozeno kompilatorem z hodnoty na prave strane
%var va = 12356;
%// nullovatelny typ (standardni int + hodnota null navic)
%int? nullablex = null;


\begin{frame}[fragile]
\vfill
\begin{exampleblock}{Primitivní datové typy}
\begin{lstlisting}
// uint je 32 bitový bezznaménkový typ
uint bezznamenkovyInteger;
\end{lstlisting}
\end{exampleblock}
\vfill
\begin{bonusblock}{~}
\begin{lstlisting}
System.SByte int8; // sbyte
System.Int16 int16; // short
System.Int32 int32; // int
System.Int64 int64; // long

System.Byte uint8; // byte
System.UInt16 uint16; // ushort
System.UInt32 uint32; // uint
System.UInt64 uint64; // ulong

System.Single fSingle; // float
System.Double fDouble; // double
System.Decimal fDecimal; // decimal
\end{lstlisting}
\end{bonusblock}
\vfill
\end{frame}


\begin{frame}[fragile]
\begin{block}{Synonyma}
\begin{itemize}
\item Pozor na synonyma, je možné používat obojí, ale preferována je použití klíčového slova před názvem typu\ldots
\item \lstinline|string == String|
\item \lstinline|object == Object|
\item \lstinline|int == System.Int32|
\item Tedy používejte \lstinline|string, object, int|
\end{itemize}
\end{block}
\end{frame}

\begin{frame}[fragile]
\vfill
\begin{block}{Nulovatelné datové typy}
\begin{itemize}
\item \lstinline|datovyTyp?|
\item slouží jako \uv{rozšiřující modifikátor} pro uložení hodnoty \lstinline|null| do libovolného typu
\item objekt pak má vlastnosti \lstinline|HasValue, Value| a metodu \lstinline|GetValueOrDefault|
\end{itemize}
\end{block}
\vfill
\begin{yesblock}
\begin{lstlisting}
// int může nabývat hodnot -2147483648 až 2147483647
// co když chceme říct, že hodnota není?
int? nulovatelnyInt = null;

nulovatelnyInt = 100;
nulovatelnyInt = null;

\end{lstlisting}
\end{yesblock}
\vfill
\end{frame}




\begin{frame}[fragile]
\vfill
\begin{block}{var -- automatické odvození datového typu}
\begin{itemize}
\item \lstinline|var| umožňuje nechat kompilátor odvodit datový typ
\item nejedná se o dynamický typ, \lstinline|var| nabývá pouze konkrétního datového typu
\item umožňuje použití anonymních (objektových) datových typů 
\end{itemize}
\end{block}
\vfill
\begin{yesblock}
\begin{lstlisting}
var varPromenna = 123;
// varPromenna je int
varPromenna++;
// nelze: varPromenna = "abcd";
\end{lstlisting}
\end{yesblock}
\vfill
\end{frame}


\begin{frame}[fragile]
\frametitle{Výčtové typy}
\vfill
\begin{yesblock}
\begin{lstlisting}
enum VyctovyTyp
{
    Prvni, // = 0
    Druhy, // = 1
    Treti // = 2
}

Console.WriteLine($"{VyctovyTyp.Prvni}");
\end{lstlisting}
\end{yesblock}
\vfill
\begin{yesblock}
\begin{lstlisting}
enum VyctovyTyp
{
    Prvni = 10,
    Druhy, // = 11
    Treti // = 12
}
\end{lstlisting}
\end{yesblock}
\vfill
\end{frame}


\begin{frame}[fragile]
\frametitle{Výčtové typy -- bitové masky}
\begin{bonusblock}{~}
\begin{lstlisting}
[Flags] // atribut System.FlagsAttribute
enum Permissions
{
    Read = 0x01,
    Write = 0x02,
    Execute = 0x04
}

static void Main(string[] args)
{
    Permissions p = Permissions.Read | Permissions.Write;
    if (p.HasFlag(Permissions.Read))
    {
        Console.WriteLine($"{p}");
    }
}
\end{lstlisting}
\end{bonusblock}
\end{frame}


\kapitola{Výrazy}


\begin{frame}[fragile]
\begin{exampleblock}{Proměnné a výrazy/operátory}
\begin{lstlisting}
int x = 123;

// přiřazení, složené výrazy
// =, +, -, *, /, %, +=, -=, *=, /=, %=
x = 456;
x += 99;
x = x - 99;
x = x * x;
// inkrementace/dekrementace
x++;
--x;
// ternární operátor
x = (x % 2 == 0) ? x : 0;

Console.WriteLine($"{x}");
\end{lstlisting}
\end{exampleblock}
\end{frame}


\begin{frame}[fragile]
\begin{exampleblock}{Proměnné a výrazy/operátory}
\begin{lstlisting}
int x = 123;
int y = 456;

// porovnání
// >, <, >=, <=, ==, !=
bool vysledekVetsi = x > y;
bool vysledekShoda = x == y;

// bitové operátory
// &, |, ^, >>, <<
int z = x & (y << 5);

\end{lstlisting}
\end{exampleblock}
\end{frame}



\begin{frame}[fragile]
\vfill
\begin{exampleblock}{Konverze datového typu -- kulaté závorky}
\begin{lstlisting}[basicstyle=\small]
double dbl = 123.45;
int ine = (int)dbl;
\end{lstlisting}
\end{exampleblock}
\vfill
\begin{exampleblock}{~}
\begin{lstlisting}[basicstyle=\small]
Parent parent = new Child();
Child child;

// vyvolá InvalidCastException pokud objekt nebude typu Child
child = (Child)parent; 
\end{lstlisting}
\end{exampleblock}
\vfill
\end{frame}





\begin{frame}[fragile]
\begin{exampleblock}{Konverze datového typu -- as, is}
\begin{lstlisting}[basicstyle=\small]
Parent parent = new Child();
Child child;

// null pokud nebude daného typu
child = parent as Child; 

// is - C# obdoba operátoru instanceof z Javy
if (parent is Child)
{
    Child cld = (Child)parent;
    cld.Action();
}

// C# 7 - umožňuje rovnou deklarovat proměnnou
if (parent is Child chd)
{
    chd.Action();
}
\end{lstlisting}
\end{exampleblock}
\end{frame}




\begin{frame}[fragile]
\begin{bonusblock}{Konverze datového typu -- reflexe + typeof}
\begin{lstlisting}[basicstyle=\small]
Parent parent = new Child();
Child child;

if (parent.GetType() == typeof(Child)) 
{
    child = (Child)parent;
}
\end{lstlisting}
\end{bonusblock}
\end{frame}



\begin{frame}[fragile]
\begin{exampleblock}{Null operátory}
\begin{lstlisting}
int? a = null;

// ??
int b = a ?? -1;
// a == null? potom b = -1
// a != null? potom b = a

// ?.
object o = null;
o?.ToString();
// o == null? potom nic nedělej
// o != null? potom zavolej ToString()

// ?[]
int[] array = null;
int? x = array?[0];
\end{lstlisting}
\end{exampleblock}
\vfill
\begin{yesblock}
\begin{lstlisting}
var value = firstObject?.secondObject?.maybeArray?[0] ?? "default";
\end{lstlisting}
\end{yesblock}
\end{frame}



\begin{frame}[fragile]
\begin{bonusblock}{Přetečení/podtečení -- checked blok}
\begin{lstlisting}
int i = 10, j = 0;
i = i / j; // vyvolá DivideByZeroException

(i, j) = (int.MaxValue, int.MinValue); // Tuples/n-tice C# 7
i++;
j--;
Console.WriteLine($"{i} {j}");
// i == -2147483648 == int.MinValue
// j == 2147483647 == int.MaxValue

(i, j) = (int.MaxValue, int.MinValue);
checked
{
    i++; // vyvolá OverflowException
    j--; // vyvolá OverflowException
}
\end{lstlisting}
\end{bonusblock}
\end{frame}


\pkapitola{Základní řídící konstrukce}



\begin{frame}[fragile]
\vfill
\begin{block}{Rozhodování}
\begin{itemize}
\item syntax stejná jako Java/C/C++
\item podmínka musí být vyhodnocena na typ \lstinline|bool|
\end{itemize}
\end{block}
\vfill
\begin{yesblock}
\begin{lstlisting}
int x = 123;
if (x > 10)
{
    Console.WriteLine("x > 10");
}
\end{lstlisting}
\end{yesblock}
\vfill
\begin{noblock}
\begin{lstlisting}
int x = 123;
if (x)
{
    Console.WriteLine("error");
}
\end{lstlisting}
\end{noblock}
\vfill
\end{frame}

\begin{frame}[fragile]
\begin{yesblock}
\begin{lstlisting}
int x = 123;
if (x > 10)
{
    Console.WriteLine("x > 10");
}
else
{
    Console.WriteLine("x <= 10");
}
\end{lstlisting}
\end{yesblock}
\end{frame}





\begin{frame}[fragile]
\vfill
\begin{block}{Cykly}
\begin{itemize}
\item \lstinline|while, do-while, for, foreach|
\item syntax shodná
\item \lstinline|foreach| lze využívat na pole a kolekce
\item podporováno řízení cyklů \lstinline|break, continue|
\end{itemize}
\end{block}
\vfill
\begin{yesblock}
\begin{lstlisting}
int x = 123;
for (int i = 0; i < x; i++) 
{
    Console.WriteLine("{0}", i);
}
\end{lstlisting}
\end{yesblock}
\vfill
\end{frame}



\begin{frame}[fragile]
\vfill
\begin{yesblock}
\begin{lstlisting}
do
{
    x--;
} while (x > 0);
\end{lstlisting}
\end{yesblock}
\vfill
\begin{yesblock}
\begin{lstlisting}
while (x < 10)
{
    x++;
}
\end{lstlisting}
\end{yesblock}
\vfill
\end{frame}




\begin{frame}[fragile]
\begin{yesblock}
\begin{lstlisting}
int x = 123;
while (true)
{
    if (x == 200)
        continue;

    if (x == 300)
        break;
    
    x++;
}
\end{lstlisting}
\end{yesblock}
\end{frame}





\begin{frame}[fragile]
\vfill
\begin{block}{Switch}
\begin{itemize}
\item syntax shodná
\item nelze propadávat z jednoho \lstinline|case| do druhého, je nutné explicitně skočit \lstinline|goto case 123|
\item nepovinná větev \lstinline|default| pro ostatních hodnot
\end{itemize}
\end{block}
\vfill
\begin{bonusblock}{~}
\begin{itemize}
\item od C\# 7 podpora pattern matching
\begin{itemize}
\item umožňuje rozlišit datový typ objektu (lze dále definovat podmínku)
\item \lstinline|case string s when s.Contains("foobar"):|
\item \lstinline|case int n when n > 100:|
\item \lstinline|case int n:|
\end{itemize}
\end{itemize}
\end{bonusblock}
\vfill
\end{frame}





\begin{frame}[fragile]
\begin{yesblock}
\begin{lstlisting}
int x = 123;
switch (x)
{
    case 0: 
        Console.WriteLine("0");
        break;

    case 1:
        Console.WriteLine("1");
        //goto case 0; // preskok na 0
        //goto default; // preskok na default
        break;

    default:
        Console.WriteLine("...");
        break;
}
\end{lstlisting}
\end{yesblock}
\end{frame}




\kapitola{Pole}

\begin{frame}[fragile]
\begin{block}{Pole}
\begin{itemize}
\item pole je \textbf{referenční objekt}
\begin{itemize}
\item při předání do metody je možné měnit obsah pole (stejně jako v Javě)
\end{itemize}

\item C\# umožňuje vytvářet 3 typy polí
\begin{itemize}
\item jednorozměrná pole
\item vícerozměrná pole (pravidelná)
\item vícerozměrná pole (nepravidelná, jagged array)
\end{itemize}

\item syntax obdobný Javě, hranaté závorky, indexování prvků od nuly, přístup mimo rozsah pole vyvolá výjimku
\end{itemize}
\end{block}
\end{frame}

\begin{frame}[fragile]
\frametitle{Jednorozměrné pole}
\begin{yesblock}
\begin{lstlisting}
int[] pole;
pole = new int[10];

pole[0] = 10;
pole[1] = 20;
//pole[10] = 15; // IndexOutOfRangeExpection

int pocetPrvkuPole = pole.Length;

Console.WriteLine($"{pocetPrvkuPole} {pole.Length} {pole[0]}{pole[1]}");
\end{lstlisting}
\end{yesblock}
\end{frame}



\begin{frame}[fragile]
\frametitle{Vícerozměrné pole (pravidelné)}
\begin{yesblock}
\begin{lstlisting}
int[,] pole = new int[2,3];

pole[0, 0] = 1;
pole[0, 1] = 2;
pole[0, 2] = 3;
pole[1, 0] = 4;
pole[1, 1] = 5;
pole[1, 2] = 6;

int pocetPrvkuVPoli = pole.Length; // == 2*3 == 6
int pocetPrvkuVPrvniDimenzi = pole.GetLength(0); // == 2
int pocetPrvkuVDruheDimenzi = pole.GetLength(1); // == 3

Console.WriteLine($"{pocetPrvkuVPoli} {pole.Length} {pocetPrvkuVPrvniDimenzi} {pocetPrvkuVDruheDimenzi} {pole[0, 0]} {pole[0, 1]}");
\end{lstlisting}
\end{yesblock}
\end{frame}


\begin{frame}[fragile]
\frametitle{Vícerozměrné pole (pravidelné)}
\begin{yesblock}
\begin{lstlisting}
int[] pole1D = new int[2];
int[,] pole2D = new int[2,3];
int[,,] pole3D = new int[2,3,5];
int[,,,] pole4D = new int[2,3,5,6];
int[,,,,] pole5D = new int[2,3,5,6,7];
\end{lstlisting}
\end{yesblock}
\end{frame}



\begin{frame}[fragile]
\frametitle{Vícerozměrné pole (nepravidelné, jagged array)}
\begin{yesblock}
\begin{lstlisting}
int[][] jaggedArray = new int[3][];
jaggedArray[0] = new int[2];
jaggedArray[1] = new int[3];
jaggedArray[2] = new int[4];

jaggedArray[0][0] = 1;
jaggedArray[0][1] = 2;
jaggedArray[1][0] = 3;
jaggedArray[1][1] = 4;
jaggedArray[1][2] = 5;

Console.WriteLine($"{jaggedArray[0][0]} {jaggedArray[0][1]}");
\end{lstlisting}
\end{yesblock}
\end{frame}





\kapitola{Základní konzolové příkazy}


\begin{frame}[fragile]
\vfill
\begin{block}{Výstupní parametry metod}
Umožňují předat volající funkci další výsledky.
\end{block}
\vfill
\begin{yesblock}
\begin{lstlisting}
static void OutMethod(int a, out int b, out int c)
{
    b = 2 * a;
    c = 3 * a;
}

static void Main(string[] args)
{
    int a = 123;
    int b, c;
    OutMethod(a, out b, out c);
    Console.WriteLine("{0} {1} {2}", a, b, c);
}
\end{lstlisting}
\end{yesblock}
\vfill
\end{frame}


\begin{frame}[fragile]
\vfill
\begin{block}{Referenční parametry metod}
Parametry funkce předávané "odkazem".
\end{block}
\vfill
\begin{yesblock}
\begin{lstlisting}
static void RefMethod(int a, ref int b)
{
    b = b * a;
}

static void Main(string[] args)
{
    int a = 123;
    int b = 456;
    RefMethod(a, ref b);
    Console.WriteLine("{0} {1}", a, b);
}
\end{lstlisting}
\end{yesblock}
\vfill
\end{frame}



\begin{frame}[fragile]
\vfill
\begin{exampleblock}{Konverze string $\rightarrow$ int/double/\ldots}
\begin{lstlisting}
string str = "123";
// Parse - vyvolá FormatException, pokud parametr neobsahuje řetězec s číslem
int istr = int.Parse(str);
\end{lstlisting}
\end{exampleblock}
\vfill
\begin{exampleblock}{Konverze string $\rightarrow$ int/double/\ldots}
\begin{lstlisting}
// TryParse - vrací bool (úspěch/neúspěch konverze), zkonvertovaná hodnota je vrácena výstupním parametrem
int istr2;
bool uspech = int.TryParse(str, out istr2);
if (uspech) 
{
    Console.WriteLine(istr2);
}
\end{lstlisting}
\end{exampleblock}
\vfill
\end{frame}


\begin{frame}[fragile]
\vfill
\begin{exampleblock}{Také lze využít třídu System.Convert}
\begin{lstlisting}
int valueInt = Convert.ToInt32("123456");
double valueDouble = Convert.ToDouble("3,1415");
\end{lstlisting}
\end{exampleblock}
\vfill
\begin{exampleblock}{Konverze int/double/\ldots $\rightarrow$ string}
\begin{lstlisting}
int i = 123;

string s = i.ToString();
// lze také:
// string s = 123.ToString();
\end{lstlisting}
\end{exampleblock}
\vfill
\end{frame}


\begin{frame}[fragile]
\begin{exampleblock}{Třída System.Console}
\begin{lstlisting}
Console.Write(...); // vytiskne text
Console.WriteLine(...); // vytiskne text a odradkuje
Console.Read(); // nacte znak
Console.ReadKey(); // nacte stisk klavesy
Console.ReadLine(); // nacte radek textu
\end{lstlisting}
\end{exampleblock}
\end{frame}

%\hkapitola{Objektové typy}

\begin{frame}[fragile]
\frametitle{Objektové typy}
\begin{bitemize}
\item \lstinline|class| -- plnohodnotný objektový typ
\item \lstinline|struct| -- plnohodnotný objektový typ
\item \lstinline|union| -- omezený objektový typ (neumí dědičnost, polymorfizmus)
\end{bitemize}

\begin{block}{}
Základní vlastnosti:
\begin{itemize}
\item vícenásobná dědičnost
\begin{itemize}
\item neexistují rozhraní
\end{itemize}

\item polymorfizmus
\item H x CPP soubor
\begin{itemize}
\item H -- deklarace (celý typ, atributy, prototypy metod)
\item CPP -- definice (pouze těla metod, statické atributy)
\end{itemize}
\end{itemize}
\end{block}
\end{frame}


\begin{frame}[fragile]
\frametitle{struct x class}
\begin{bitemize}
\item \lstinline|struct| i \lstinline|class| jsou zaměnitelné
\item \lstinline|struct| může dědit z \lstinline|class| a naopak
\item liší se pouze ve výchozí úrovni viditelnosti složek (\lstinline|public| x \lstinline|private|)
\end{bitemize}
\end{frame}


\kapitola{Základní syntaxe objektových typů}

\begin{frame}[fragile]
\frametitle{Definice struct a class}
\begin{noteblock}{}
\begin{lstlisting}
struct|class nazevDatovehoTypu [final] [dědičnost] {
  [složky - atributy, metody, vnořené typy]
} [objekty];
\end{lstlisting}
\end{noteblock}


\begin{twocols}
\begin{yesblock}
\begin{lstlisting}
struct Pes {
  ...
};
\end{lstlisting}
\end{yesblock}

\twocolssep

\begin{yesblock}
\begin{lstlisting}
class Pes {
  ...
};
\end{lstlisting}
\end{yesblock}
\end{twocols}

\begin{block}{dopředná deklarace}
\begin{itemize}
\item pokud potřebujeme pracovat s typem, ale ne s jeho vnitřkem
\item \lstinline!struct|class nazevDatovehoTypu;!
\end{itemize}
\end{block}

\end{frame}


\begin{frame}[fragile]
\begin{noteblock}{}
\begin{lstlisting}[basicstyle=\scriptsize]
struct|class nazevDatovehoTypu [final] [dědičnost] {
  [složky - atributy, metody, vnořené typy]
} [objekty];
\end{lstlisting}
\end{noteblock}

\begin{bitemize}
\item{} [objekty] -- umožňuje vytvořit staticky alokované objekty od dané třídy v~globálním prostoru
\end{bitemize}

\begin{yesblock}
\begin{lstlisting}
struct Pes {
  ...
} punta, rafan;

void main() { 
  punta.stekej(); 
  rafan.stekej(); 
}
\end{lstlisting}
\end{yesblock}
\end{frame}


\begin{frame}[fragile]
\begin{noteblock}{}
\begin{lstlisting}[basicstyle=\scriptsize]
struct|class nazevDatovehoTypu [final] [dědičnost] {
  [složky - atributy, metody, vnořené typy]
} [objekty];
\end{lstlisting}
\end{noteblock}

\begin{bitemize}
\item{} [\lstinline|final|]\cpp{11} -- zakazuje dědit z této třídy
\item{} [dědičnost] -- specifikace předků
\begin{itemize}
\item \lstinline|: [viditelnost] [virtual] předek1, ...|
\end{itemize}
\end{bitemize}

\begin{yesblock}
\begin{lstlisting}
struct Pes : public virtual Zvire, public IIdentifikovatelny, IKopirovatelny {
  ...
};
\end{lstlisting}
\end{yesblock}
\end{frame}



\begin{frame}[fragile]
\begin{noteblock}{}
\begin{lstlisting}[basicstyle=\scriptsize]
struct|class nazevDatovehoTypu [final] [dědičnost] {
  [složky - atributy, metody, vnořené typy]
} [objekty];
\end{lstlisting}
\end{noteblock}

\begin{bitemize}
\item{} [složky] -- dále dle typu atribut, metoda, vnořený typ
\end{bitemize}

\begin{yesblock}
\begin{lstlisting}
struct Pes {

  void stekej() { ... }

  string jmeno;

};
\end{lstlisting}
\end{yesblock}
\end{frame}


%
\hkapitola{Objektové typy -- dědičnost, polymorfizmus}

\kapitola{Dědičnost}

\begin{frame}[fragile]
\begin{bitemize}{Dědičnost, polymorfizmus}
\item jednoduchá dědičnost -- je možné dědit z jednoho předka (Java)
\begin{itemize}
\item syntaxe připomíná C++
\item pokud není specifikován předek, třída automaticky dědí z~\lstinline|System.Object|
\end{itemize}
\item vícenásobná realizace rozhraní (Java)
\item explicitní rozlišování polymorfních metod -- časná/pozdní vazba (C++)
\end{bitemize}
\end{frame}



\begin{frame}[fragile]
\begin{bitemize}{Konstrukce objektu}
\item \lstinline|new System.Windows.Forms.Form()|
\item proces postupného volání konstruktorů
\begin{itemize}
\item \lstinline|System.Object|
\item \lstinline|System.MarshalByRefObject|
\item \lstinline|System.ComponentModel.Component|
\item \lstinline|System.Windows.Forms.Control|
\item \lstinline|System.Windows.Forms.ScrollableControl|
\item \lstinline|System.Windows.Forms.ContainerControl|
\item \lstinline|System.Windows.Forms.Form|
\end{itemize}
\item poté je objekt zkonstruován a lze jej používat
\end{bitemize}
\end{frame}



\begin{frame}[fragile]
\frametitle{Dědičnost}
\begin{noteblock}{}
\begin{lstlisting}[basicstyle=\small]
[viditelnost] [modifikátory] class NazevTridy [dědičnostARozhraní] { 
	[složkyTřídy]...
}

dědičnostARozhraní:
	: [názevPředkaNeboRozhraní], ...
\end{lstlisting}
\end{noteblock}
\vfill
\begin{yesblock}
\begin{lstlisting}[basicstyle=\small]
class Person
{
    public string Name { get; set; }
}

class Student : Person
{
    public string StudentID { get; set; }
}
\end{lstlisting}
\end{yesblock}
\end{frame}



\begin{frame}[fragile]
\frametitle{Dědičnost -- příklady}
\begin{yesblock}
\begin{lstlisting}[basicstyle=\small]
class Student : Person, IComparable, ICloneable { }
class Student : IComparable { }
class Student : ICloneable, IComparable { }
\end{lstlisting}
\end{yesblock}
\vfill
\begin{noblock}
\begin{lstlisting}[basicstyle=\small]
class Student : Person, Object { }
class Student : Person, LivingEntity { }
\end{lstlisting}
\end{noblock}
\end{frame}



\begin{frame}[fragile]
\begin{bitemize}{}
\item automaticky se volá konstruktor předka
\begin{itemize}
\item kompilátor umí volat pouze bezparametrický konstruktor
\end{itemize}
\end{bitemize}

\begin{yesblock}
\begin{lstlisting}[basicstyle=\small]
class Person
{
  public Person() => Console.WriteLine("Person()");
  public Person(string name) => Console.WriteLine("Person(string)");
}

class Student : Person
{
  public Student() => Console.WriteLine("Student()");
  public Student(string netid) => Console.WriteLine("Student(string)")
}
\end{lstlisting}
\end{yesblock}
\end{frame}


\begin{frame}[fragile]
\begin{yesblock}
\begin{lstlisting}[basicstyle=\small]
Student student = new Student();
// Person()
// Student()

Student studentWithParam = new Student("netid");
// Person()
// Student(string netid)
\end{lstlisting}
\end{yesblock}
\end{frame}


\begin{frame}[fragile]
\begin{noblock}
\begin{lstlisting}[basicstyle=\small]
class Person
{
  //public Person() => Console.WriteLine("Person()");
  public Person(string name) => Console.WriteLine("Person(string)");
}

// nelze zkompilovat - kompilátor neumí vytvořit objekt
class Student : Person
{
  public Student() => Console.WriteLine("Student()");
  public Student(string netid) => Console.WriteLine("Student(string)")
}
\end{lstlisting}
\end{noblock}
\end{frame}



\begin{frame}[fragile]
\begin{bitemize}{}
\item konstruktor předka lze zavolat pomocí \lstinline|: base([parametry])|
\end{bitemize}

\begin{yesblock}
\begin{lstlisting}[basicstyle=\small]
class Person
{
    //public Person() => Console.WriteLine("Person()");
    public Person(string name) => Console.WriteLine("Person(string name)");
}

class Student : Person
{
    public Student() : base("unknown") 
    {
        Console.WriteLine("Student()");
    }

    public Student(string name, string netid) : base(name) 
        => Console.WriteLine("Student(string, string)");
}
\end{lstlisting}
\end{yesblock}
\end{frame}




\begin{frame}[fragile]
\begin{bitemize}{}
\item lze zavolat i jiný konstruktor z aktuální třídy \lstinline|: this([parametry])|
\end{bitemize}

\begin{yesblock}
\begin{lstlisting}[basicstyle=\small]
class Student : Person
{
    public Student() : this("unknown", "unidentified") 
    { 
    }

    public Student(string name, string netid) : base(name) 
        => Console.WriteLine("Student(string, string)");
}
\end{lstlisting}
\end{yesblock}
\end{frame}
%\hkapitola{Výjimky}


\begin{frame}[fragile]
\frametitle{Ošetřování chybových stavů}

\begin{bitemize}
\item \lstinline|abort()|
\item návratová hodnota funkce
\item technika dlouhých skoků (long jumps) v jazyce C
\item \lstinline|goto|
\end{bitemize}
\end{frame}


\begin{frame}[fragile]
\frametitle{Výjimky}

\begin{bitemize}
\item výjimky slouží pro řešení chybových stavů v průběhu programu
\begin{itemize}
\item pokusný blok \lstinline|try|, kde může vzniknout výjimka
\item zachycující bloky \lstinline|catch|, které ošetřují výjimky
\item příkaz \lstinline|throw| vyvolávající výjimku
\end{itemize}
\item výjimkou v C++ může být prakticky cokoliv (int, char, string, objekt, dynamicky alokovaný objekt, \ldots)
\begin{itemize}
\item obecně se nedoporučuje používat dynamickou alokaci paměti pro objekt výjimky; je pak nutné řešit jeho dealokaci
\end{itemize}
\item pokud výjimka není zachycena na úrovni kde vznikla šíří se postupně dále
\begin{itemize}
\item neošetřená výjimka způsobí pád programu
\end{itemize}
\item v C++ není obdoba bloku {\ttfamily finally} z Javy/C\#
\item C++ podporuje označování funkcí, kde výjimka nevzniká (\lstinline|noexcept|\cpp{11})
\begin{itemize}
\item původní systém označoval konkrétní druhy výjimek, které mohly vznikat; jeho špatná definice a implementace způsobily jeho zrušení
\end{itemize}
\end{bitemize}
\end{frame}


\begin{frame}[fragile]
\frametitle{Základní použití výjimek}
\begin{noteblock}{}
\begin{lstlisting}[basicstyle=\small]
try {
  [kód, kde může vzniknout výjimka]
} catchHandler...

catchHandler:
  catch(datovýTypVýjimky názevProměnné) { [blok příkazů handleru] }
  catch(...) { [blok příkazů handleru] }
\end{lstlisting}
\end{noteblock}

\begin{yesblock}
\begin{lstlisting}[basicstyle=\small]
try {
  ...
  throw 100;
  ...

} catch (int i) {
  cout << "Zachyceno " << i;
}
\end{lstlisting}
\end{yesblock}
\end{frame}



\begin{frame}[fragile]
\frametitle{Základní použití výjimek\ldots}
\begin{bitemize}
\item \lstinline|catch| handlerů může následovat více za sebou
\begin{itemize}
\item záleží na pořadí!
\item výjimka je zachycena do prvního vyhovujícího handleru
\end{itemize}
\end{bitemize}
\begin{yesblock}
\begin{lstlisting}
try {
  ...
  throw 3.14;
  ...

} catch (int i) {
  cout << "Zachyceno int " << i;
} catch (double d) {
  cout << "Zachyceno double " << d;
}
\end{lstlisting}
\end{yesblock}
\end{frame}


\begin{frame}[fragile]
\frametitle{Základní použití výjimek\ldots}
\vskip -1ex
\begin{bitemize}
\item lze vytvořit univerzální \lstinline|catch| handler
\begin{itemize}
\item nelze pak pracovat s vlastní hodnotou výjimky
\end{itemize}
\end{bitemize}
\vskip -1ex
\begin{yesblock}
\begin{lstlisting}
try {
  ...
  throw 'E';
  ...

} catch (int i) {
  cout << "Zachyceno int " << i;
} catch (double d) {
  cout << "Zachyceno double " << d;
} catch (...) {
  cou << "Zachyceno vše ostatní";
}
\end{lstlisting}
\end{yesblock}
\end{frame}



\begin{frame}[fragile]
\frametitle{Základní použití výjimek\ldots}
\begin{bitemize}
\item neošetřená výjimka se šíří dále
\end{bitemize}
\begin{yesblock}
\begin{lstlisting}
void funkce() {
  throw "Vyjimka";
}

try {
  ...
  funkce();
  ...

} catch (const char* str) {
  cout << "Zachyceno " << str;
}
\end{lstlisting}
\end{yesblock}
\end{frame}



\begin{frame}[fragile]
\frametitle{Základní použití výjimek\ldots}
\begin{bitemize}
\item dynamicky alokované objekty výjimky jsou možné
\begin{itemize}
\item ale nevhodné
\item alokace je další místo, kde může vzniknout výjimka!
\end{itemize}
\end{bitemize}
\begin{yesblock}
\begin{lstlisting}[basicstyle=\small]
try {
  ...
  throw new int(-99);
  ...

} catch (int* i) {
  cout << "Zachyceno " << *i;
  delete i;
}
\end{lstlisting}
\end{yesblock}
\end{frame}




\begin{frame}[fragile]
\frametitle{Základní použití výjimek\ldots}
\begin{bitemize}
\item objekt výjimky je při vyvolání nakopírován někam do paměti
\item \lstinline|catch| handler přijímá kopii objektu výjimky
\begin{itemize}
\item pokud není použita reference!
\end{itemize}
\end{bitemize}
\begin{yesblock}
\begin{lstlisting}
try {
  ...
  throw error_object{"My error"};
  ...

} catch (error_object& errobj) {
  cout << "Zachyceno " << errobj.getMessage();
}
\end{lstlisting}
\end{yesblock}
\end{frame}




\begin{frame}[fragile]
\frametitle{throw()}
\begin{oldblock}
\begin{itemize}
\item \lstinline|throw| také sloužilo k definici, jaké výjimky může funkce vyvolat
\begin{itemize}
\item obdoba {\ttfamily throws} z Javy
\item nikdy to pořádně nefungovalo, kompilátory to ignorovaly
\item deprecated v C++11
\item zcela odstraněno v C++17
\end{itemize}
\end{itemize}
\end{oldblock}
\vskip -2ex
\begin{deprecatedblock}{}
\begin{lstlisting}
void func1() throw() { ... } // nevyvolává výjimky
void func2() throw(int) { ... } // vyvolává int
void func3() throw(int, char) { ... } // vyvolává int i char 
\end{lstlisting}
\end{deprecatedblock}
\end{frame}





\begin{frame}[fragile]
\frametitle{noexcept\cpp{11}}

\begin{bitemize}
\item \lstinline|noexcept|\cpp{11} -- nově zjednodušuje předchozí systém a slouží k označení funkcí a metod, které nevyvolávají výjimky
\begin{itemize}
\item vyvolání výjimky z označené funkce by vedlo k vyvolání \lstinline|terminate| a ukončení programu
\end{itemize}
\end{bitemize}

\begin{yesblock}
\begin{lstlisting}
void func() noexcept { ... }
\end{lstlisting}
\end{yesblock}
\end{frame}


\kapitola{Výjimky v C++ knihovně}

\begin{frame}[fragile]
\frametitle{std::exception}
\begin{bitemize}
\item \lstinline|std::exception| je třída ze které dědí všechny standardní výjimky definované v knihovně C++
\item objekt výjimky obsahuje pouze textovou informaci o druhu chyby (\lstinline|virtual const char* what() const|)
\item definováno v hl. s. \lstinline|<exception>|
\begin{itemize}
\item potomci definováni v hl. s. \lstinline|<stdexcept>|
\end{itemize}
\end{bitemize}
\end{frame}



\begin{frame}[fragile]
\frametitle{Potomci std::exception}

\begin{bitemize}
\item \lstinline|logic_error|
\begin{itemize}
\item \lstinline|invalid_argument|
\item \lstinline|domain_error|
\item \lstinline|length_error|
\item \lstinline|out_of_range|
\item \lstinline|future_error|\cpp{11}
\item \lstinline|bad_optional_access|\cpp{17}
\end{itemize}

\item \lstinline|runtime_error|
\begin{itemize}
\item \lstinline|range_error|
\item \lstinline|overflow_error|
\item \lstinline|underflow_error|
\item \lstinline|regex_error|\cpp{11}
\item \lstinline|system_error|\cpp{11}
\begin{itemize}
\item \lstinline|ios_base::failure|\cpp{11}
\item \lstinline|filesystem::filesystem_error|\cpp{17}
\end{itemize}
\end{itemize}
\end{bitemize}
\end{frame}



\begin{frame}[fragile]
\frametitle{Potomci std::exception\ldots}

\begin{bitemize}
\item \lstinline[keywordstyle={\ttfamily}]|bad_typeid|
\item \lstinline[keywordstyle={\ttfamily}]|bad_cast|
\begin{itemize}
\item \lstinline|bad_any_cast|\cpp{17}
\end{itemize}
\item \lstinline|bad_weak_ptr|\cpp{11}
\item \lstinline|bad_function_call|\cpp{11}
\item \lstinline|bad_alloc|
\begin{itemize}
\item \lstinline|bad_array_new_length|\cpp{11}
\end{itemize}
\item \lstinline|bad_exception|
\item \lstinline|bad_variant_access|\cpp{17}
\end{bitemize}
\end{frame}



\begin{frame}[fragile]
\begin{bonusblock}{Function try}
\begin{itemize}
\item celé tělo funkce nebo metody lze zabalit do bloku try
\begin{itemize}
\item funguje i na konstruktory a inicializační část
\end{itemize}
\end{itemize}
\end{bonusblock}
\vskip -2ex
\begin{bonusblock}{}
\begin{lstlisting}
struct Trida {

  Trida(const std::string& atr) try : _atr(atr) {
    // ...
  } catch (const std::exception& ex) {
    std::cerr << "oops";
  }

  std::string _atr;
}
\end{lstlisting}
\end{bonusblock}
\end{frame}

\hkapitola{Delegáty, události}

\begin{frame}[fragile]
\begin{bitemize}{Delegát}
\item \textbf{definuje datový typ} představující \textit{ukazatel na metodu}
\item typově bezpečný, bezpečné volání, \textbf{hodnotový typ}
\item instance delegátu (proměnná typu delegát) může obsahovat 0, 1 nebo více ukazatelů na metodu shodného předpisu
\end{bitemize}
\vfill
\begin{noteblock}{definice delegátu -- nového datového typu}
\begin{lstlisting}
delegate typNávratovéHodnoty názevTypuDelegátu([parametryMetody]);
\end{lstlisting}
\end{noteblock}
\vfill
\begin{yesblock}
\begin{lstlisting}
delegate void SimpleDelegate();
delegate int ReturnDelegate();
delegate int FunctionalDelegate(int a, int b);
\end{lstlisting}
\end{yesblock}
\vfill
\begin{bitemize}{}
\item název delegátu by se měl skládat z 
\begin{itemize}
\item vlastního popisu funkce
\item koncovky \lstinline|Callback| -- u obecného použití
\item koncovky \lstinline|EventHandler| -- u událostí
\end{itemize}

\end{bitemize}
\end{frame}





\begin{frame}[fragile]
\frametitle{Základní použití delegátu}
\begin{yesblock}
\begin{lstlisting}
delegate int CalculateCallback(); // definice delegáta

class Program
{
    static int CalculateStatic() => 1;  // statická metoda
    int CalculateInstance() => 2;       // instanční metoda

    static void Main(string[] args)
    {
        // vytvoření objektů delegátů
        CalculateCallback calculateStatic = CalculateStatic;
        CalculateCallback calculateInstance = (new Program()).CalculateInstance;

        var rs = calculateStatic();
        var ri = calculateInstance();
        Console.WriteLine($"{rs} {ri}");
    } }
\end{lstlisting}
\end{yesblock}
\end{frame}





\begin{frame}[fragile]
\frametitle{Vícenásobný delegát}
\vfill
\begin{bitemize}{}
\item do jedné instance (proměnné) delegátu je možné uložit několik odkazů na metody najednou
\begin{itemize}
\item při vyvolání delegátu dojde k vyvolání všech metod
\end{itemize}

\end{bitemize}
\vfill
\begin{yesblock}
\begin{lstlisting}
delegate void OutputCallback();

class Program
{
    void Out1() => Console.WriteLine("1");
    void Out2() => Console.WriteLine("2");
    void Out3() => Console.WriteLine("3");
    void Out4() => Console.WriteLine("4");

    static void Main(string[] args)
    {
        // ...
    }
}
\end{lstlisting}
\end{yesblock}
\vfill
\end{frame}




\begin{frame}[fragile]
\frametitle{Vícenásobný delegát\ldots}
\begin{yesblock}
\begin{lstlisting}
var program = new Program();
OutputCallback cc = null;

cc = program.Out1;
cc += program.Out2; cc += program.Out3; cc += program.Out4;
cc(); // 1 2 3 4

cc -= program.Out2;
cc(); // 1 3 4 

cc += program.Out4;
cc(); // 1 3 4 4

cc = program.Out2;
cc(); // 2

cc -= program.Out2;
cc(); // NullReferenceException
\end{lstlisting}
\end{yesblock}
\end{frame}




\begin{frame}[fragile]
\frametitle{Bezpečné vyvolání delegátu}
\vfill
\begin{bitemize}{}
\item pokus o vyvolání delegátu, ve kterém není uložen žádný odkaz na metodu vyvolá výjimku \lstinline|NullReferenceException|
\end{bitemize}
\vfill
\begin{yesblock}
\begin{lstlisting}
var program = new Program();
OutputCallback cc = null;

if (cc != null)
{
    cc();
    // nebo
    cc.Invoke();
}

// nebo

cc?.Invoke();
\end{lstlisting}
\end{yesblock}
\vfill
\end{frame}




\begin{frame}[fragile]
\frametitle{Delegát jako hodnotový typ}
\vfill
\begin{bitemize}{}
\item delegát je \textbf{hodnotový typ}, předávání do metody (v parametru) nebo přiřazení do jiné proměnné vytvoří kopii aktuálního stavu delegátu
\end{bitemize}
\vfill
\begin{yesblock}
\begin{lstlisting}
OutputCallback firstDelegate = program.Out1;

// hodnotový typ - zkopíruje ukazatele
OutputCallback secondDelegate = firstDelegate;
// změna neovlivní firstDelegate
secondDelegate += program.Out2;

firstDelegate(); // 1
secondDelegate(); // 1 2
\end{lstlisting}
\end{yesblock}
\vfill
\end{frame}





\begin{frame}[fragile]
\begin{bitemize}{Základní typy delegátů v C\# knihovně}
\item \lstinline|void Action()|
\item \lstinline|void Action<p1>(p1)| -- generický delegát, je možné dosadit libovolný typ parametrů
\item \lstinline|void Action<p1, p2>(p1, p2)|
\item \lstinline|ret Func<ret>()|
\item \lstinline|ret Func<p1, ret>(p1)|
\item \lstinline|ret Func<p1, p2, ret>(p1, p2)|
\item \lstinline|bool Predicate<p>(p)|
\end{bitemize}

\vfill

\begin{yesblock}
\begin{lstlisting}
delegate void Example1Callback(Person person, int value);
// je "ekvivalentním" typem
Action<Person, int>

delegate string Example2Callback(Student student);
Func<Student, string>
\end{lstlisting}
\end{yesblock}
\end{frame}





\begin{frame}[fragile]
\begin{bitemize}{Delegát může obsahovat}
\item ukazatel na statickou metodu
\item ukazatel na instanční metodu
\item ukazatel na anonymní metodu
\end{bitemize}
\end{frame}



\pkapitola{Anonymní metody}


\begin{frame}[fragile]
\begin{bitemize}{Anonymní metody}
\item metoda nemá jméno
\item metoda je obvykle definována někde uprostřed bloku jiné metody
\item C\# podporuje několik způsobů zápisu anonymních metod
\begin{itemize}
\item od C\# 2 -- s využitím \lstinline|delegate|
\item od C\# 3 -- lambda výrazy (preferovaný způsob)
\end{itemize}

\end{bitemize}
\end{frame}


\begin{frame}[fragile]
\begin{bitemize}{}
\item anonymní metoda -- původní syntaxe
\begin{itemize}
\item klíčové slovo \lstinline|delegate|
\item seznam parametrů metody
\item tělo metody
\end{itemize}
\end{bitemize}
\vfill
\begin{yesblock}
\begin{lstlisting}
delegate int TransformCallback(int value);

static void Main(string[] args)
{
    TransformCallback tc = delegate (int value)
    {
        return value + 1;
    };

    int result = tc(1);
    Console.WriteLine($"{result}"); // 2
}
\end{lstlisting}
\end{yesblock}
\end{frame}




\begin{frame}[fragile]
\begin{bitemize}{}
\item anonymní metoda -- lambda funkce / výraz
\begin{itemize}
\item \lstinline|([parametry]) => { příkazy }|
\end{itemize}
\end{bitemize}
\vfill
\begin{yesblock}
\begin{lstlisting}
TransformCallback tc = (int i) => {
    return i + 1;
};

// blok může být nahrazen pouze výrazem
tc = (int i) => i + 1;

// datové typy parametrů mohou být odvozeny
tc = (i) => i + 1;

// pokud je pouze jeden parametr není nutné uvádět závorky
tc = i => i + 1;

int result = tc(1);
Console.WriteLine($"{result}"); // 2
\end{lstlisting}
\end{yesblock}
\end{frame}







\begin{frame}[fragile]
\begin{bitemize}{Praktická ukázka použití delegátů a anonymních metod}
\item parametrizace chování obecných algoritmů
\item technologie LINQ
\end{bitemize}
\vfill
\begin{yesblock}
\begin{lstlisting}
List<int> values = new List<int>
{
    10, 2, 24, 32, 5, 674, 23, 9875, 12, 43, 43, 23
};

var result = values
    // vyber pouze hodnoty větší než 10
    .Where(v => v > 10)
    // seřaď hodnoty od nejmenší po největší
    .OrderBy(v => v)
    // každou hodnotu umocni na druhou
    .Select(v => v * v)
    // vrať výsledek jako List
    .ToList();
// result = 144 529 529 576 1024 1849 1849 454276 97515625
\end{lstlisting}
\end{yesblock}
\end{frame}



\nezkouskove

\begin{frame}[fragile]
\vfill
\begin{bitemize}{Kovariance, kontravariance parametrů a návratových hodnot}
\item C\# uplatňuje kovarianci návratových typů delegátů
\item C\# uplatňuje kontravarianci parametrů delegátů
\end{bitemize}
\vfill
\begin{bitemize}{}
\item kovariance znamená, že metoda může mít typ, který je více specifický než původně definovaný 
\begin{itemize}
\item metoda vrací potomka typu, který vrací delegát dle definice
\end{itemize}

\item kontravariance znamená, že metoda může mít typ, který je méně specifický než původně definovaný
\begin{itemize}
\item metoda má v parametru předka typu, který definuje delegát
\end{itemize}

\end{bitemize}
\vfill
\end{frame}


\begin{frame}[fragile]
\vfill
\vskip 1ex
\begin{exampleblock}{Kovariance}
\begin{lstlisting}
class Mammals{}  
class Dogs : Mammals{}  

class Program  
{  
    // Define the delegate.  
    public delegate Mammals HandlerMethod();  

    public static Mammals MammalsHandler() => null;
    public static Dogs DogsHandler() => null;

    static void Test()  
    {  
        HandlerMethod handlerMammals = MammalsHandler;  

        // Covariance enables this assignment.  
        HandlerMethod handlerDogs = DogsHandler;  
    }  
}  
\end{lstlisting}
\end{exampleblock}
\vfill
\end{frame}


\begin{frame}[fragile]
\vfill
\vskip 1ex
\begin{exampleblock}{Kontravariance}
\begin{lstlisting}
// Event handler that accepts a parameter of the EventArgs type.  
private void MultiHandler(object sender, System.EventArgs e)  
{  
    label1.Text = System.DateTime.Now.ToString();  
}  

public Form1()  
{  
    InitializeComponent();  

    // You can use a method that has an EventArgs parameter,  
    // although the event expects the KeyEventArgs parameter.  
    this.button1.KeyDown += this.MultiHandler;  

    // You can use the same method   
    // for an event that expects the MouseEventArgs parameter.  
    this.button1.MouseClick += this.MultiHandler;  

}  
\end{lstlisting}
\end{exampleblock}
\vfill
\end{frame}


\zkouskove

\pkapitola{Události}

\begin{frame}[fragile]
\begin{bitemize}{Události -- teoreticky}
\item realizuje \textit{návrhový vzor observer}
\item umožňuje objektům sledovat změny stavu konkrétního objektu
\begin{itemize}
\item příklad -- časovač nabídne událost Tik, která nastane vždy jednou za sekundu
\item konzole / grafické rozhraní se přihlásí (metodu PrintTime) k odběru Tiků
\item časovač odpočítá sekundu a vyvolá událost Tik, dojde k vyvolání metody PrintTime
\end{itemize}

\end{bitemize}
\end{frame}



\begin{frame}[fragile]
\begin{yesblock}
\begin{lstlisting}
static void Main(string[] args)
{
	// vytvoř časovač a spusť ho
    Timer t = new Timer();
    // nastav obslužnou metodu události
    t.Tick += TimerTickHandler;

    // ... program běží do stisku klávesy ...
    // Tick! Tick! Tick!

    Console.ReadKey();
}

private static void TimerTickHandler(object sender, EventArgs eventArgs)
{
    Console.WriteLine("Tick!");
}
\end{lstlisting}
\end{yesblock}
\end{frame}



\begin{frame}[fragile]
\begin{bitemize}{Událost -- event}
\item je složkou třídy
\item je \uv{proměnná} typu konkrétního delegáta
\item chová se velmi podobně jako normální proměnná typu delegát, ale
\begin{itemize}
\item vyvolat delegát může pouze definující třída
\item operátor = může použít pouze definující třída
\item ostatní třídy mohou používat +=, -= pro úpravu ukazatelů uložených v~delegátu
\end{itemize}
\end{bitemize}

\end{frame}






\begin{frame}[fragile]
\frametitle{Vytvoření vlastní události}
\vfill
\begin{bitemize}{1. delegát}
\item událost musí mít definovaný delegát, který definuje parametry události
\begin{itemize}
\item navratová hodnota \lstinline|void|
\item název delegátu končí \lstinline|EventHandler|
\item parametry metody
\begin{itemize}
\item \lstinline|object sender| -- kdo vyvolal událost
\item \lstinline|EventArgs eventArgs| -- parametry události (objekt \lstinline|EventArgs| nebo potomek této třídy)
\end{itemize}
\end{itemize}
\end{bitemize}
\vfill
\begin{yesblock}
\begin{lstlisting}
delegate void TickEventHandler(object sender, EventArgs eventArgs);
\end{lstlisting}
\end{yesblock}
\vfill
\end{frame}




\begin{frame}[fragile]
\vfill
\begin{bitemize}{2. událost}
\item ve třídě vytvoříme událost
\begin{itemize}
\item od běžné proměnné typu delegát se odlišuje klíčovým slovem \lstinline|event|
\end{itemize}
\end{bitemize}
\vfill
\begin{yesblock}
\begin{lstlisting}
class Timer
{
    public event TickEventHandler Tick;

    // ...
}
\end{lstlisting}
\end{yesblock}
\vfill
\end{frame}




\begin{frame}[fragile]
\begin{bitemize}{3. metoda pro vyvolání události}
\item ve třídě vytvoříme pomocnou metodu \lstinline|OnEvent...|
\begin{itemize}
\item bezpečně vyvolává událost
\item může být \lstinline|protected virtual| nebo i \lstinline|private| dle použití
\end{itemize}
\end{bitemize}
\vfill
\begin{yesblock}
\begin{lstlisting}
protected virtual void OnTick(EventArgs eventArgs)
{
    Tick?.Invoke(this, eventArgs);
}
\end{lstlisting}
\end{yesblock}
\vfill
\begin{oldblock}
\begin{lstlisting}
protected virtual void OnTick(EventArgs eventArgs)
{
    TickEventHandler handlers = Tick;
    if (handlers != null)
        handlers(this, eventArgs);
}
\end{lstlisting}
\end{oldblock}
\end{frame}







\begin{frame}[fragile]
\vfill
\begin{bitemize}{4. kód vyvolávající událost}
\item doplníme kód, který událost vyvolá
\end{bitemize}
\vfill
\begin{yesblock}
\begin{lstlisting}
public Timer()
{
    Thread t = new Thread(TimingThread);
    t.IsBackground = true;
    t.Start();
}

private void TimingThread()
{
    while (true)
    {
        Thread.Sleep(1000);

        OnTick(new EventArgs()); // <- vyvolej událost
    }
}
\end{lstlisting}
\end{yesblock}
\vfill
\end{frame}




\begin{frame}[fragile]
\vfill
\begin{bitemize}{5. použití události}
\item vně třídy nastavíme handler a událost zachytíme
\end{bitemize}
\vfill
\begin{yesblock}
\begin{lstlisting}
static void Main(string[] args)
{
	// vytvoř časovač a spusť ho
    Timer t = new Timer();
    // nastav obslužnou metodu události
    t.Tick += TimerTickHandler;

    // Tick! Tick! Tick!

    Console.ReadKey();
}

private static void TimerTickHandler(object sender, EventArgs eventArgs) =>
    Console.WriteLine("Tick!");
\end{lstlisting}
\end{yesblock}
\vfill
\end{frame}





\begin{frame}[fragile]
\vfill
\begin{bitemize}{Událost jako vlastnost}
\item událost může být definována jako vlastnost
\item vlastnost pak nabízí akce \lstinline|add| a \lstinline|remove| pro zpracování operátorů +=, -=
\end{bitemize}
\vfill
\begin{yesblock}
\begin{lstlisting}
private TickEventHandler tick;
public event TickEventHandler Tick
{
    add { tick += value; }
    remove { tick -= value; }
}
\end{lstlisting}
\end{yesblock}
\vfill
\end{frame}

%\hkapitola{Přetěžování operátorů}

\begin{frame}[fragile]
\vfill
\begin{bitemize}{Přetěžování operátorů}
\item podobně jako C++ lze v C\# přetěžovat operátory u svých objektových typů
\item množina přetížitelných operátorů je oproti C++ menší
\item syntax a pravidla pro přetěžování jsou striktnější a snaží se zamezit problémům
\begin{itemize}
\item např. není možné přetížit == bez přetížení !=
\end{itemize}

\end{bitemize}
\vfill
\begin{bitemize}{Přetížitelné operátory}
\item \lstinline|- ! ~ ++ -- true false|
\item \lstinline|- * / % & ^ | | \lstinline| << >> == != > < >= <=|
\item konverzní operátory
\item \lstinline|[ ]| (indexery)
\end{bitemize}
\vfill
\end{frame}


\nezkouskove

\begin{frame}[fragile]
\frametitle{Přetížení oprerátoru}
\vfill
\begin{noteblock}{}
\begin{lstlisting}
public static typ operator symbol ( parametry ) tělo
\end{lstlisting}
\end{noteblock}
\vfill
\begin{bitemize}{}
\item parametry
\begin{itemize}
\item unární operátor -- jeden parametr
\item binární operátor -- dva parametry
\end{itemize}

\item logické dvojice -- vždy je nutné přetížit oba operátory naráz
\begin{itemize}
\item \lstinline|== !=|
\item \lstinline|< >|
\item \lstinline|<= >=|
\item \lstinline|true false|
\end{itemize}

\item inkremetace/dekremetace se přetěžuje jednou metodou, kompilátor ji automaticky použije pro prefixovou i postfixovou variantu
\item operátory \lstinline|true,false| umožňují dát objekt přímo do výrazů očekávající logickou hodnotu (\lstinline|if, while, ...|)

\end{bitemize}
\vfill
\end{frame}


\begin{frame}[fragile]
\frametitle{Přetížení oprerátoru -- konverzní operátory}
\vfill
\begin{noteblock}{}
\begin{lstlisting}
public static implicit operator cílovýTyp ( zdrojovýTyp parametr ) tělo
public static explicit operator cílovýTyp ( zdrojovýTyp parametr ) tělo
\end{lstlisting}
\end{noteblock}
\vfill
\begin{bitemize}{}
\item není možné zároveň definovat pravidla pro implicitní a explicitní konverzi
\item kompilátor tyto operátory volá při výrazech:
\begin{itemize}
\item implicitní
\begin{itemize}
\item \lstinline|cílovýTyp proměnná = objektZrojovéhoTypu|
\end{itemize}

\item explicitní
\begin{itemize}
\item \lstinline|cílovýTyp proměnná = (cílovýTyp)objektZdrojovéhoTypu|
\end{itemize}
\end{itemize}

\end{bitemize}
\vfill
\end{frame}

\zkouskove

\begin{frame}[fragile]
\frametitle{Indexery}
\vfill
\begin{bitemize}{}
\item definují přetížení operátoru \lstinline|[ ]| pro přístup k prvkům/datům
\item definují se jako nestatické vlastnosti
\end{bitemize}
\vfill
\begin{noteblock}{}
\begin{lstlisting}
[modifikátor] návratovýTyp this [ seznamParametrů ] deklaracePřístupovýchMetod
\end{lstlisting}
\end{noteblock}
\vfill
\begin{yesblock}
\begin{lstlisting}
class IndexerExample
{
    readonly int[] values = new int[] { 1, 2, 3, 4, 5, 6, 7, 8, 9, 10 };

    public int this[int index]
    {
        get => values[index];
    }
}
\end{lstlisting}
\end{yesblock}
\vfill
\end{frame}


\begin{frame}[fragile]
\frametitle{Indexery}
\vfill
\begin{bitemize}{}
\item je možné definovat několik různých přetížení indexerů s různými parametry
\item rozhraní mohou definovat indexery
\item indexer může data číst/zapisovat nebo provádět obojí
\end{bitemize}
\vfill
\begin{yesblock}
\begin{lstlisting}
var indexer = new IndexerExample();
int numberThree = indexer[2];
\end{lstlisting}
\end{yesblock}
\vfill
\end{frame}

%\hkapitola{Preprocesor}


\begin{frame}[fragile]
\begin{bitemize}{Preprocesor}
\item podobně jako v C++ nabízí C\# preprocesor a několik příkazů, které se provedou před vlastní kompilací
\item možnosti jsou více omezené než v C++
\begin{itemize}
\item možnost podmíněné kompilace (dle definovaných symbolů preprocesoru)
\item možnost označovat část kódu (\lstinline|#region|) pro větší přehlednost v~editoru
\item možnost vyvolat chybu nebo varování při kompilaci části kódu 
\item možnost změnit číslo řádku nebo název souboru (\lstinline|#line|)
\item další možnosti dle kompilátoru (\lstinline|#pragma|)
\end{itemize}
\end{bitemize}

\end{frame}

\begin{frame}[fragile]
\vfill
\begin{bitemize}{\#define, \#undef}
\item definuje a ruší symboly (konstanty) preprocesoru
\item definice musí být na prvním místě v souboru
\item lze na nich založit podmíněnou kompilaci
\end{bitemize}
\vfill
\begin{yesblock}
\begin{lstlisting}
#define DEBUG
#undef TRACE
\end{lstlisting}
\end{yesblock}
\vfill
\end{frame}




\begin{frame}[fragile]
\vfill
\begin{bitemize}{\#if, \#elif, \#else, \#endif}
\item podmínkový blok v preprocesoru
\end{bitemize}
\vfill
\begin{yesblock}
\begin{lstlisting}
#if (DEBUG && !MYTEST)
    Console.WriteLine("DEBUG is defined");
#elif (!DEBUG && MYTEST)
    Console.WriteLine("MYTEST is defined");
#elif (DEBUG && MYTEST)
    Console.WriteLine("DEBUG and MYTEST are defined");  
#else
    Console.WriteLine("DEBUG and MYTEST are not defined");
#endif
\end{lstlisting}
\end{yesblock}
\vfill
\end{frame}



\begin{frame}[fragile]
\vfill
\begin{bitemize}{ConditionalAttribute -- podmíněná kompilace pomocí atributu}
\item lze aplikovat na metody a třídy (pouze atributy)
\item pokud není podmínka splněna, daný objekt se nezkompiluje
\end{bitemize}
\vfill
\begin{yesblock}
\begin{lstlisting}
[Conditional("CONDITION1")]
public static void Method1(int x)
{
    Console.WriteLine("CONDITION1 is defined");
}

[Conditional("CONDITION1"), Conditional("CONDITION2")]  
public static void Method2()
{
    Console.WriteLine("CONDITION1 or CONDITION2 is defined");
}
\end{lstlisting}
\end{yesblock}
\vfill
\end{frame}



\begin{frame}[fragile]
\vfill
\begin{bitemize}{\#warning, \#error}
\item vyvolá varování nebo error v době kompilace na dané řádce
\end{bitemize}
\vfill
\begin{yesblock}
\begin{lstlisting}
#define DEBUG  
class MainClass   
{  
    static void Main()   
    {  
#if DEBUG  
#warning DEBUG is defined  
#endif  
    }  
}  
\end{lstlisting}
\end{yesblock}
\vfill
\end{frame}


\begin{frame}[fragile]
\vfill
\begin{bitemize}{\#region, \#endregion}
\item označuje blok kódu, který spolu logicky souvisí
\item ve VS lze takový blok naráz zobrazit/skrýt pro větší přehlednost
\end{bitemize}
\vfill
\begin{yesblock}
\begin{lstlisting}
#region MyClass definition  
public class MyClass   
{  
    static void Main()   
    {  
    }  
}  
#endregion  
\end{lstlisting}
\end{yesblock}
\vfill
\end{frame}


%\hkapitola{Genericita}

\begin{frame}[fragile]
\begin{bitemize}{Genericita}
\item podobně jako v Javě je v C\# k dispozici genericita pro možnost generického programování
\item syntaxe velmi podobná Javě
\begin{itemize}
\item jinak se definují podmínky kovariance a kontravariance a omezující podmínky
\end{itemize}

\item Java genericitu realizuje technikou \uv{type erasure} (ve zkompilovaném kódu se genericita zahodí), C\# využívá \uv{reification} (\uv{zhmotnění typů} -- lze používat generické typy v reflexi i všude jinde)
\begin{itemize}
\item v důsledku existují třídy/rozhraní s genericitou a bez ní (\lstinline|IEnumerable| není to samé co \lstinline|IEnumerable<T>|)
\item metody lze přetěžovat s různými generickými parametry
\end{itemize}
\item parametrem generického typu může být libovolný hodnotový i referenční typ (tedy i primitivní datové typy)
\item vnořené typy automaticky dědí nařazené generické parametry
\end{bitemize}
\vfill


\end{frame}


\begin{frame}[fragile]
\begin{yesblock}
\begin{lstlisting}
class GenericHolder<T>
{
    public T Value { get; set; }
}
\end{lstlisting}
\end{yesblock}
\vfill
\begin{yesblock}
\begin{lstlisting}
List<int> list1 = new List<int>();

// No boxing, no casting:
list1.Add(3);

// Compile-time error:
// list1.Add("It is raining in Redmond.");
\end{lstlisting}
\end{yesblock}
\end{frame}
%\hkapitola{Jmenné prostory}

\begin{frame}[fragile]
\frametitle{Jmenné prostory}

\begin{bitemize}{namespace}
\item stejně jako balíčky slouží k logickému oddělení kódu -- rozdělení na spolu související komponenty a zamezení konfliktů jmen
\item jmenné prostory na rozdíl od Java balíčků nevyžadují přesnou organizaci na disku
\item jmenné prostory je možné libovolně vnořovat
\item uvnitř jmenného prostoru jsou vidět komponenty definované v tomto jmenném prostoru
\item []
\item komponenty definované v jiném jmenném prostoru je nutné zpřístupnit užitím plné cesty a názvu komponenty nebo zpřístupnit jmenný prostor pomocí \lstinline|using|
\end{bitemize}
\end{frame}


\begin{frame}[fragile]
\frametitle{Definovaní jmenných prostorů}

\begin{yesblock}
\begin{lstlisting}
namespace Game
{
    // Game ...

    namespace Graphics
    {
        // Game.Graphics ...
    }
}
\end{lstlisting}
\end{yesblock}
\vfill
\begin{yesblock}
\begin{lstlisting}
namespace Game.Graphics
{
    // Game.Graphics ...
}
\end{lstlisting}
\end{yesblock}
\end{frame}




\begin{frame}[fragile]
\frametitle{Použití jmenných prostorů -- definice}

\begin{yesblock}
\begin{lstlisting}
namespace First
{
    class FirstClass
    {
        public const int FirstConst = 1;
    }
}
\end{lstlisting}
\end{yesblock}
\end{frame}




\begin{frame}[fragile]
\frametitle{Použití jmenných prostorů -- zpřístupnění jiného jmenného prostoru}
\vfill
\begin{bitemize}{}
\item standardně je nutné uvést plnou \uv{cestu} k dané komponentě
\end{bitemize}
\vfill
\begin{yesblock}
\begin{lstlisting}
namespace Second
{
    class SecondClass
    {
        static void Method()
        {
            // FirstConst // error CS0103
            var v = First.FirstClass.FirstConst;
        }
    }
}
\end{lstlisting}
\end{yesblock}
\vfill
\end{frame}




\begin{frame}[fragile]
\frametitle{Použití jmenných prostorů -- using}
\vfill
\begin{bitemize}{}
\item pomocí \lstinline|using| lze zpřístupnit obsah jmenného prostoru do aktuálního kontextu
\end{bitemize}
\vfill
\begin{yesblock}
\begin{lstlisting}
using First;

namespace Second
{
    class SecondClass
    {
        static void Method()
        {
            var v = FirstClass.FirstConst;
        }
    }
}
\end{lstlisting}
\end{yesblock}
\vfill
\end{frame}






\begin{frame}[fragile]
\frametitle{Zpřístupnění statických položek -- using static}
\vfill
\begin{bitemize}{}
\item od C\# 6 existuje klauzule \lstinline|using static|, která umožňuje přímo pracovat se statickými složkami vybraného typu
\end{bitemize}
\vfill
\begin{yesblock}
\begin{lstlisting}
using static System.Math;

class ExampleUsingStaticClass
{
    static double CalculateCircleArea(double radius)
    {
        // bez using static:
        // return Math.PI * Math.Pow(radius, 2);
        
        return PI * Pow(radius, 2);
    }
}
\end{lstlisting}
\end{yesblock}
\vfill
\end{frame}

%%System.Collections
%ArrayList
%BitArray
%Hashtbake
%Queue
%SortedList
%Stack

%System.Collections.Generic
%Dictionary
%List

%SystemCollectionsConcurrent
%ConcurrentDictionary/Queue/Stack
%ConcurrentBag, BlockingCollection


\hkapitola{Kolekce}

\begin{frame}[fragile]
\vfill
\begin{bitemize}{Kolekce}
\item podobně jako v Javě se v základní knihovně nachází celá řada kolekcí pro ukládání dat
\item kolekce se liší způsobem organizace dat a jejich účelem (a tedy i složitostí a variabilitou základních operací)
\item od C\# 2 přibyla genericita a s ní také generické kolekce a rozhraní
\end{bitemize}
\vfill
\begin{bitemize}{Jmenné prostory kolekcí}
\item \lstinline|System.Collections| -- základní kolekce
\item \lstinline|System.Collections.Generic| -- generické kolekce
\item \lstinline|System.Collections.Concurrent| -- kolekce pro paralelní programování (synchronizované kolekce)
\end{bitemize}
\vfill
\end{frame}


\begin{frame}[fragile]
\begin{bitemize}{System.Collections.Generic -- obecná rozhraní}
\item \lstinline|ICollection<T>| -- základ kolekčního chování
\begin{itemize}
\item \footnotesize Add(T), Clear(), Contains(T), CopyTo(T[],int), Remove(T), Count, IsReadOnly
\end{itemize}

\item \lstinline|IComparer<T>| -- porovnání dvou objektů (větší menší)
\begin{itemize}
\item \footnotesize Compare(T,T)
\end{itemize}

\item \lstinline|IEqualityComparer<T>| -- porovnání dvou objektů (shoda neshoda)
\begin{itemize}
\item \footnotesize Equals(T,T)
\end{itemize}

\item \lstinline|IEnumerable<T>| -- enumerovatelná (iterovatelná) kolekce (\lstinline|foreach|)
\begin{itemize}
\item \footnotesize GetEnumerator()
\end{itemize}

\item \lstinline|IEnumerator<T>| -- enumerátor (iterátor)
\begin{itemize}
\item \footnotesize Current, MoveNext(), Reset()
\end{itemize}

\item \lstinline|IReadOnlyCollection<T>| -- kolekce pouze pro čtení
\begin{itemize}
\item \footnotesize rozh. IEnumerable, Count
\end{itemize}

\end{bitemize}
\end{frame}



\begin{frame}[fragile]
\begin{bitemize}{System.Collections.Generic -- rozhraní kolekcí}
\item \lstinline|IList<T>| -- seznam prvků, prvky identifikovány pořadím v kolekci
\begin{itemize}
\item \footnotesize IndexOf(T), Insert(int, T), RemoveAt(int), this[int]
\end{itemize}

\item \lstinline|IDictionary<TKey, TValue>| -- asociativní kolekce, přiřazuje ke klíčům konkrétní hodnoty
\begin{itemize}
\item \footnotesize Add(TKey,TValue), ContainsKey(TKey), Remove(TKey), TryGetValue(TKey,out TValue), Keys, Values, this[TKey]
\end{itemize}

\item \lstinline|ISet<T>| -- množina prvků
\begin{itemize}
\item \footnotesize množinové operace (průnik, sjednocení, \ldots), test shody množiny, test je podmnožina, \ldots
\end{itemize}

\item \lstinline|IReadOnlyList<T>|
\item \lstinline|IReadOnlyDictionary<TKey, TValue>|
\end{bitemize}
\end{frame}




\begin{frame}[fragile]
\begin{bitemize}{System.Collections.Generic -- kolekce}
\item \lstinline|List<T>| -- \uv{ArrayList}
\item \lstinline|LinkedList<T>| -- spojový seznam
\item \lstinline|SortedList<TKey, TValue>|
\item []
\item \lstinline|Dictionary<TKey, TValue>| -- tabulka 
\item \lstinline|SortedDictionary<TKey, TValue>|
\item \lstinline|SortedSet<T>| -- množina hodnot
\item \lstinline|HashSet<T>|
\item []
\item \lstinline|Queue<T>| -- fronta 
\item \lstinline|Stack<T>| -- zásobník
\end{bitemize}
\end{frame}




\begin{frame}[fragile]
\begin{yesblock}
\begin{lstlisting}
List<string> listOfStrings = new List<string>()
{
    "Hello",
    "World"
};

string hello = listOfStrings[0];

listOfStrings.Add("!");
listOfStrings.Remove("Hello");
listOfStrings.Insert(1, hello);

foreach (string str in listOfStrings)
{
    Console.WriteLine(str);
}

// World Hello !
\end{lstlisting}
\end{yesblock}
\end{frame}




\begin{frame}[fragile]
\begin{yesblock}
\begin{lstlisting}
LinkedList<string> linkedList = new LinkedList<string>();
linkedList.AddFirst("2");
linkedList.AddFirst("1");
linkedList.AddLast("3");

LinkedListNode<string> node = linkedList.Find("2");
node.Value = "-2";
linkedList.AddAfter(node, "2.5");
linkedList.AddBefore(node, "1.5");

foreach (var item in linkedList)
{
    Console.WriteLine(item);
}

// 1 1.5 -2 2.5 3 
\end{lstlisting}
\end{yesblock}
\end{frame}






\begin{frame}[fragile]
\begin{yesblock}
\begin{lstlisting}
Dictionary<string, double> prevodyJednotek = new Dictionary<string, double>()
{
    {"dm",  0.1},
    {"cm",  0.01}
};

prevodyJednotek["mm"] = 0.001;
double mmRatio = prevodyJednotek["mm"];

if (prevodyJednotek.TryGetValue("cm", out double cm)) {
    // ...
}

prevodyJednotek.Add("um", 0.000001);
bool hasUm = prevodyJednotek.ContainsKey("um");

double value = prevodyJednotek["pm"] * 25; // KeyNotFoundException
\end{lstlisting}
\end{yesblock}
\end{frame}



\begin{frame}[fragile]
\begin{yesblock}
\begin{lstlisting}[deletekeywords={set}]
HashSet<int> set = new HashSet<int>();
set.Add(1);
set.Add(1);
set.Add(2);
set.Add(1);
set.Add(3);
set.Add(5);
set.Add(100);
set.Add(1500);
set.Add(500);

bool hasFour = set.Contains(4);
bool isSubset = set.IsSubsetOf(new List<int>() { 2, 5 });

foreach (var item in set)
{
    Console.WriteLine(item);
}

// 1 2 3 5 100 1500 500
\end{lstlisting}
\end{yesblock}
\end{frame}


\begin{frame}[fragile]
\begin{yesblock}
\begin{lstlisting}
class MyInt { public int Integer { get; set; } }
class MyIntComparer : IComparer<MyInt>
{
    public int Compare(MyInt x, MyInt y) => x.Integer - y.Integer;
}
\end{lstlisting}
\end{yesblock}
\vfill
\begin{yesblock}
\begin{lstlisting}
SortedList<MyInt, string> sl = new SortedList<MyInt, string>(
    new MyIntComparer());
sl[new MyInt() { Integer = 10 }] = "V10";
sl[new MyInt() { Integer = 1 }] = "V1";
sl[new MyInt() { Integer = 30 }] = "V30";

foreach (KeyValuePair<MyInt, string> item in sl)
{
    Console.WriteLine(item.Value);
}

// V1 V10 V30
\end{lstlisting}
\end{yesblock}
\end{frame}


%%BinaryReader - stream
%writer - stream
%bufferedstream - stream
%
%filestream
%memorystream
%
%streamreader - stream
%streamwriter - stream
%
%stringreader/writer - string
%
%Stream - abstract
%textreader/writer - abtstract class
%
%File,FileInfo
%Directory,DirectoryInfo
%Path

\hkapitola{Input/output}

\begin{frame}[fragile]
\vfill
\begin{bitemize}{Základní proudy pro práci se soubory, pamětí, \ldots}
\item knihovna využívá stejných principů jako Java
\item základní abstraktní třída \lstinline|Stream| reprezentuje datový proud
\begin{itemize}
\item potomci představují konkrétní použití -- \lstinline|FileStream|, \lstinline|MemoryStream|, \lstinline|NetworkStream|, \ldots
\item existuje zde řada dekorátorů -- \lstinline|BufferedStream|, \lstinline|DeflateStream|, \lstinline|GZipStream|, \lstinline|CryptoStream|, \ldots
\end{itemize}
\item pro realizaci textových přenosů slouží třídy \lstinline|StreamReader|, \lstinline|StreamWriter|
\begin{itemize}
\item binární přenosy \lstinline|BinaryReader|, \lstinline|BinaryWriter|
\end{itemize}
\end{bitemize}
\vfill
\begin{bitemize}{}
\item proudy je třeba korektně uzavírat, jinak může dojít ke ztrátě dat a prostředky mohou být dlouho blokovány
\item proudy typicky realizují \lstinline|IDisposable| rozhraní (lze využít konstrukci \lstinline|using|)
\end{bitemize}
\vfill
\end{frame}



\begin{frame}[fragile]
\vfill
\begin{bitemize}{Abstraktní třída Stream}
\item \lstinline|Position| -- vlastnost, pozice kurzoru
\item \lstinline|CanRead, CanWrite, CanSeek| -- vlastnosti
\item \lstinline|Seek| -- přesune kurzor v souboru
\item \lstinline|Read| -- čte pole bajtů
\item \lstinline|Write| -- zapisuje pole bajtů
\item \lstinline|ReadByte| -- čte bajt
\item \lstinline|WriteByte| -- zapisuje bajt
\item \lstinline|Close| -- uzavře proud
\end{bitemize}
\vfill
\begin{bitemize}{Potomci třídy Stream}
\item \lstinline|BaseStream| -- vlastnost, není definováno rozhraním, nemusí být u~všech
\end{bitemize}
\vfill
\end{frame}






\begin{frame}[fragile]
\frametitle{Zápis a čtení z textového souboru}
\begin{yesblock}
\begin{lstlisting}
using (StreamWriter writer = new StreamWriter(new FileStream("output.txt", FileMode.OpenOrCreate), Encoding.UTF8))
{
    writer.WriteLine("Hello World");
    writer.WriteLine("Příliš žluťoučký kůň úpěl ďábelské ódy");
}

using (var reader = new StreamReader("output.txt"))
{
    Console.WriteLine(reader.ReadLine());
    Console.WriteLine(reader.ReadLine());
}
\end{lstlisting}
\end{yesblock}
\end{frame}


\begin{frame}[fragile]
\frametitle{Základní metody StreamReader, StreamWriter}
\vfill
\begin{yesblock}
\begin{lstlisting}
writer.Write(123);
writer.Write(123.456);
writer.Write(true);
writer.Write(obj);
writer.WriteLine(dtto);
\end{lstlisting}
\end{yesblock}
\vfill
\begin{yesblock}
\begin{lstlisting}
int char = reader.Read();

char[] buffer = new char[100];
reader.Read(buffer, 0, buffer.Length);

string line = reader.ReadLine();

string tillTheEnd = reader.ReadToEnd();
\end{lstlisting}
\end{yesblock}
\vfill
\end{frame}



\begin{frame}[fragile]
\vfill
\begin{bitemize}{StreamWriter -- konstrukce}
\item \lstinline|StreamWriter(Stream stream)|
\item \lstinline|StreamWriter(string path)|
\item \lstinline|StreamWriter(Stream stream, Encoding encoding)|
\item \lstinline|StreamWriter(string path, bool append)|
\item \ldots
\end{bitemize}
\vfill
\begin{bitemize}{StreamReader -- konstrukce}
\item \lstinline|StreamReader(Stream stream)|
\item \lstinline|StreamReader(Stream stream, bool detectEncodingFromByteOrderMarks)|
\item \lstinline|StreamReader(Stream stream, Encoding encoding)|
\item \lstinline|StreamReader(Stream stream, Encoding encoding, bool detectEncodingFromByteOrderMarks)|
\item dtto se \lstinline|string path| \ldots
\end{bitemize}
\vfill
\end{frame}




\begin{frame}[fragile]
\frametitle{Zápis a čtení z binárních souborů}
\begin{yesblock}
\begin{lstlisting}
using (var writer = new BinaryWriter(new FileStream("binary.dat", FileMode.Create)))
{
    // write je přetížena pro všechny základní typy
    writer.Write(true);
    writer.Write(0x11223344);
    writer.Write(123.456);
    writer.Write('z');
    writer.Write("string");

    // a podporován je i blokový přenos pomocí pole bajtů
    byte[] buffer = new byte[16];
    writer.Write(buffer, 0, buffer.Length);
}
\end{lstlisting}
\end{yesblock}
\end{frame}


\begin{frame}[fragile]
\frametitle{Zápis a čtení z binárních souborů}
\begin{yesblock}
\begin{lstlisting}
using (var reader = new BinaryReader(new FileStream("binary.dat", FileMode.Open)))
{
    bool boolVar = reader.ReadBoolean();
    int intVar = reader.ReadInt32();
    double doubleVar = reader.ReadDouble();
    char charVar = reader.ReadChar();
    string strVar = reader.ReadString();

    byte[] buffer = new byte[16];
    reader.Read(buffer, 0, buffer.Length);
}
\end{lstlisting}
\end{yesblock}
\end{frame}


%\begin{block}{Probereme později...}
%\begin{itemize}
%\item pole (jednoduché, vícerozměrné, jagged array)
%\item výjimky (try, catch, finally)
%\item delegáty, události (delegate, event)
%\item OOP (struct, class, interface, ...) 
%\item lambda výrazy  -- možná :)
%\item genericita, LINQ, extension methods -- probereme na navazujícím...
%\item lock, async/await, atributy -- v dalekém nedohlednu
%\end{itemize}
%\end{block}
%\end{frame}



\end{document}