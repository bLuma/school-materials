
\hkapitola{Delegáty, události}

\begin{frame}[fragile]
\begin{bitemize}{Delegát}
\item \textbf{definuje datový typ} představující \textit{ukazatel na metodu}
\item typově bezpečný, bezpečné volání, \textbf{hodnotový typ}
\item instance delegátu (proměnná typu delegát) může obsahovat 0, 1 nebo více ukazatelů na metodu shodného předpisu
\end{bitemize}
\vfill
\begin{noteblock}{definice delegátu -- nového datového typu}
\begin{lstlisting}
delegate typNávratovéHodnoty názevTypuDelegátu([parametryMetody]);
\end{lstlisting}
\end{noteblock}
\vfill
\begin{yesblock}
\begin{lstlisting}
delegate void SimpleDelegate();
delegate int ReturnDelegate();
delegate int FunctionalDelegate(int a, int b);
\end{lstlisting}
\end{yesblock}
\vfill
\begin{bitemize}{}
\item název delegátu by se měl skládat z 
\begin{itemize}
\item vlastního popisu funkce
\item koncovky \lstinline|Callback| -- u obecného použití
\item koncovky \lstinline|EventHandler| -- u událostí
\end{itemize}

\end{bitemize}
\end{frame}





\begin{frame}[fragile]
\frametitle{Základní použití delegátu}
\begin{yesblock}
\begin{lstlisting}
delegate int CalculateCallback(); // definice delegáta

class Program
{
    static int CalculateStatic() => 1;  // statická metoda
    int CalculateInstance() => 2;       // instanční metoda

    static void Main(string[] args)
    {
        // vytvoření objektů delegátů
        CalculateCallback calculateStatic = CalculateStatic;
        CalculateCallback calculateInstance = (new Program()).CalculateInstance;

        var rs = calculateStatic();
        var ri = calculateInstance();
        Console.WriteLine($"{rs} {ri}");
    } }
\end{lstlisting}
\end{yesblock}
\end{frame}





\begin{frame}[fragile]
\frametitle{Vícenásobný delegát}
\vfill
\begin{bitemize}{}
\item do jedné instance (proměnné) delegátu je možné uložit několik odkazů na metody najednou
\begin{itemize}
\item při vyvolání delegátu dojde k vyvolání všech metod
\end{itemize}

\end{bitemize}
\vfill
\begin{yesblock}
\begin{lstlisting}
delegate void OutputCallback();

class Program
{
    void Out1() => Console.WriteLine("1");
    void Out2() => Console.WriteLine("2");
    void Out3() => Console.WriteLine("3");
    void Out4() => Console.WriteLine("4");

    static void Main(string[] args)
    {
        // ...
    }
}
\end{lstlisting}
\end{yesblock}
\vfill
\end{frame}




\begin{frame}[fragile]
\frametitle{Vícenásobný delegát\ldots}
\begin{yesblock}
\begin{lstlisting}
var p = new Program();
OutputCallback cc = null;

cc = p.Out1;
cc += p.Out2; cc += p.Out3; cc += p.Out4;
cc(); // 1 2 3 4

cc -= p.Out2;
cc(); // 1 3 4 

cc += p.Out4;
cc(); // 1 3 4 4

cc = p.Out2;
cc(); // 2

cc -= p.Out2;
cc(); // NullReferenceException
\end{lstlisting}
\end{yesblock}
\end{frame}




\begin{frame}[fragile]
\frametitle{Bezpečné vyvolání delegátu}
\vfill
\begin{bitemize}{}
\item pokus o vyvolání delegátu, ve kterém není uložen žádný odkaz na metodu vyvolá výjimku \lstinline|NullReferenceException|
\end{bitemize}
\vfill
\begin{yesblock}
\begin{lstlisting}
var program = new Program();
OutputCallback cc = null;

if (cc != null)
{
    cc();
    // nebo
    cc.Invoke();
}

// nebo

cc?.Invoke();
\end{lstlisting}
\end{yesblock}
\vfill
\end{frame}




\begin{frame}[fragile]
\frametitle{Delegát jako hodnotový typ}
\vfill
\begin{bitemize}{}
\item delegát je \textbf{hodnotový typ}, předávání do metody (v parametru) nebo přiřazení do jiné proměnné vytvoří kopii aktuálního stavu delegátu
\end{bitemize}
\vfill
\begin{yesblock}
\begin{lstlisting}
OutputCallback firstDelegate = program.Out1;

// hodnotový typ - zkopíruje ukazatele
OutputCallback secondDelegate = firstDelegate;
// změna neovlivní firstDelegate
secondDelegate += program.Out2;

firstDelegate(); // 1
secondDelegate(); // 1 2
\end{lstlisting}
\end{yesblock}
\vfill
\end{frame}





\begin{frame}[fragile]
\begin{bitemize}{Základní typy delegátů v C\# knihovně}
\item \lstinline|void Action()|
\item \lstinline|void Action<p1>(p1)| -- generický delegát, je možné dosadit libovolný typ parametrů
\item \lstinline|void Action<p1, p2>(p1, p2)|
\item \lstinline|ret Func<ret>()|
\item \lstinline|ret Func<p1, ret>(p1)|
\item \lstinline|ret Func<p1, p2, ret>(p1, p2)|
\item \lstinline|bool Predicate<p>(p)|
\end{bitemize}

\vfill

\begin{yesblock}
\begin{lstlisting}
delegate void Example1Callback(Person person, int value);
// je "ekvivalentním" typem
Action<Person, int>

delegate string Example2Callback(Student student);
Func<Student, string>
\end{lstlisting}
\end{yesblock}
\end{frame}





\begin{frame}[fragile]
\vfill
\begin{bitemize}{Delegát může obsahovat}
\item ukazatel na statickou metodu
\item ukazatel na instanční metodu
\item ukazatel na anonymní metodu
\end{bitemize}
\vfill
\begin{bitemize}{}
\item delegát je datový typ, je možné vytvářet delegáty jako:
\begin{itemize}
\item lokální proměnné 
\item atributy tříd
\item parametry metod
\end{itemize}

\end{bitemize}
\vfill
\end{frame}



\pkapitola{Anonymní metody}


\begin{frame}[fragile]
\begin{bitemize}{Anonymní metody}
\item metoda nemá jméno
\item metoda je obvykle definována někde uprostřed bloku jiné metody
\item C\# podporuje několik způsobů zápisu anonymních metod
\begin{itemize}
\item od C\# 2 -- s využitím \lstinline|delegate|
\item od C\# 3 -- lambda výrazy (preferovaný způsob)
\end{itemize}

\end{bitemize}
\end{frame}


\begin{frame}[fragile]
\begin{bitemize}{}
\item anonymní metoda -- původní syntaxe
\begin{itemize}
\item klíčové slovo \lstinline|delegate|
\item seznam parametrů metody
\item tělo metody
\end{itemize}
\end{bitemize}
\vfill
\begin{yesblock}
\begin{lstlisting}
delegate int TransformCallback(int value);

static void Main(string[] args)
{
    TransformCallback tc = delegate (int value)
    {
        return value + 1;
    };

    int result = tc(1);
    Console.WriteLine($"{result}"); // 2
}
\end{lstlisting}
\end{yesblock}
\end{frame}




\begin{frame}[fragile]
\begin{bitemize}{}
\item anonymní metoda -- lambda funkce / výraz
\begin{itemize}
\item \lstinline|([parametry]) => { příkazy }|
\end{itemize}
\end{bitemize}
\vfill
\begin{yesblock}
\begin{lstlisting}
TransformCallback tc = (int i) => {
    return i + 1;
};

// blok může být nahrazen pouze výrazem
tc = (int i) => i + 1;

// datové typy parametrů mohou být odvozeny
tc = (i) => i + 1;

// pokud je pouze jeden parametr, není nutné uvádět závorky
tc = i => i + 1;

int result = tc(1);
Console.WriteLine($"{result}"); // 2
\end{lstlisting}
\end{yesblock}
\end{frame}







\begin{frame}[fragile]
\begin{bitemize}{Praktická ukázka použití delegátů a anonymních metod}
\item parametrizace chování obecných algoritmů
\item technologie LINQ
\end{bitemize}
\vfill
\begin{yesblock}
\begin{lstlisting}
List<int> values = new List<int>
{
    10, 2, 24, 32, 5, 674, 23, 9875, 12, 43, 43, 23
};

var result = values
    // vyber pouze hodnoty větší než 10
    .Where(v => v > 10)
    // seřaď hodnoty od nejmenší po největší
    .OrderBy(v => v)
    // každou hodnotu umocni na druhou
    .Select(v => v * v)
    // vrať výsledek jako List
    .ToList();
// result = 144 529 529 576 1024 1849 1849 454276 97515625
\end{lstlisting}
\end{yesblock}
\end{frame}



\nezkouskove

\begin{frame}[fragile]
\vfill
\begin{bitemize}{Kovariance, kontravariance parametrů a návratových hodnot}
\item C\# uplatňuje kovarianci návratových typů delegátů
\item C\# uplatňuje kontravarianci parametrů delegátů
\end{bitemize}
\vfill
\begin{bitemize}{}
\item kovariance znamená, že metoda může mít typ, který je více specifický než původně definovaný 
\begin{itemize}
\item metoda vrací potomka typu, který vrací delegát dle definice
\end{itemize}

\item kontravariance znamená, že metoda může mít typ, který je méně specifický než původně definovaný
\begin{itemize}
\item metoda má v parametru předka typu, který definuje delegát
\end{itemize}

\end{bitemize}
\vfill
\end{frame}


\begin{frame}[fragile]
\vfill
\vskip 1ex
\begin{exampleblock}{Kovariance}
\begin{lstlisting}
class Mammals{}  
class Dogs : Mammals{}  

class Program  
{  
    // Define the delegate.  
    public delegate Mammals HandlerMethod();  

    public static Mammals MammalsHandler() => null;
    public static Dogs DogsHandler() => null;

    static void Test()  
    {  
        HandlerMethod handlerMammals = MammalsHandler;  

        // Covariance enables this assignment.  
        HandlerMethod handlerDogs = DogsHandler;  
    }  
}  
\end{lstlisting}
\end{exampleblock}
\vfill
\end{frame}


\begin{frame}[fragile]
\vfill
\vskip 1ex
\begin{exampleblock}{Kontravariance}
\begin{lstlisting}
// Event handler that accepts a parameter of the EventArgs type.  
private void MultiHandler(object sender, System.EventArgs e)  
{  
    label1.Text = System.DateTime.Now.ToString();  
}  

public Form1()  
{  
    InitializeComponent();  

    // You can use a method that has an EventArgs parameter,  
    // although the event expects the KeyEventArgs parameter.  
    this.button1.KeyDown += this.MultiHandler;  

    // You can use the same method   
    // for an event that expects the MouseEventArgs parameter.  
    this.button1.MouseClick += this.MultiHandler;  

}  
\end{lstlisting}
\end{exampleblock}
\vfill
\end{frame}


\zkouskove

\pkapitola{Události}

\begin{frame}[fragile]
\begin{bitemize}{Události -- teoreticky}
\item realizuje \textit{návrhový vzor observer}
\item umožňuje objektům sledovat změny stavu konkrétního objektu
\begin{itemize}
\item příklad -- časovač nabídne událost Tick, která nastane vždy jednou za sekundu
\item konzole / grafické rozhraní se přihlásí (metodu PrintTime) k odběru Ticků
\item časovač odpočítá sekundu a vyvolá událost Tick, dojde k vyvolání metody PrintTime
\end{itemize}

\end{bitemize}
\end{frame}



\begin{frame}[fragile]
\begin{yesblock}
\begin{lstlisting}
static void Main(string[] args)
{
	// vytvoř časovač a spusť ho
    Timer t = new Timer();
    // nastav obslužnou metodu události
    t.Tick += TimerTickHandler;

    // ... program běží do stisku klávesy ...
    // Tick! Tick! Tick!

    Console.ReadKey();
}

private static void TimerTickHandler(object sender, EventArgs eventArgs)
{
    Console.WriteLine("Tick!");
}
\end{lstlisting}
\end{yesblock}
\end{frame}



\begin{frame}[fragile]
\begin{bitemize}{Událost -- event}
\item je složkou třídy
\item je \uv{proměnná} typu konkrétního delegáta
\item chová se velmi podobně jako normální proměnná typu delegát, ale
\begin{itemize}
\item vyvolat delegát může pouze definující třída
\item operátor = může použít pouze definující třída
\item ostatní třídy mohou používat +=, -= pro úpravu ukazatelů uložených v~delegátu
\end{itemize}
\end{bitemize}

\end{frame}






\begin{frame}[fragile]
\frametitle{Vytvoření vlastní události}
\vfill
\begin{bitemize}{1. delegát}
\item událost musí mít definovaný delegát, který definuje parametry události
\begin{itemize}
\item navratová hodnota \lstinline|void|
\item název delegátu končí \lstinline|EventHandler|
\item parametry metody
\begin{itemize}
\item \lstinline|object sender| -- kdo vyvolal událost
\item \lstinline|EventArgs eventArgs| -- parametry události (objekt \lstinline|EventArgs| nebo potomek této třídy)
\end{itemize}
\end{itemize}
\end{bitemize}
\vfill
\begin{yesblock}
\begin{lstlisting}
delegate void TickEventHandler(object sender, EventArgs eventArgs);
\end{lstlisting}
\end{yesblock}
\vfill
\end{frame}




\begin{frame}[fragile]
\vfill
\begin{bitemize}{2. událost}
\item ve třídě vytvoříme událost
\begin{itemize}
\item od běžné proměnné typu delegát se odlišuje klíčovým slovem \lstinline|event|
\end{itemize}
\end{bitemize}
\vfill
\begin{yesblock}
\begin{lstlisting}
class Timer
{
    public event TickEventHandler Tick;

    // ...
}
\end{lstlisting}
\end{yesblock}
\vfill
\end{frame}




\begin{frame}[fragile]
\begin{bitemize}{3. metoda pro vyvolání události}
\item ve třídě vytvoříme pomocnou metodu \lstinline|OnEvent...|
\begin{itemize}
\item bezpečně vyvolává událost
\item může být \lstinline|protected virtual| nebo i \lstinline|private| dle použití
\end{itemize}
\end{bitemize}
\vfill
\begin{yesblock}
\begin{lstlisting}
protected virtual void OnTick(EventArgs eventArgs)
{
    Tick?.Invoke(this, eventArgs);
}
\end{lstlisting}
\end{yesblock}
\vfill
\begin{oldblock}
\begin{lstlisting}
protected virtual void OnTick(EventArgs eventArgs)
{
    TickEventHandler handlers = Tick;
    if (handlers != null)
        handlers(this, eventArgs);
}
\end{lstlisting}
\end{oldblock}
\end{frame}







\begin{frame}[fragile]
\vfill
\begin{bitemize}{4. kód vyvolávající událost}
\item doplníme kód, který událost vyvolá
\end{bitemize}
\vfill
\begin{yesblock}
\begin{lstlisting}
public Timer()
{
    Thread t = new Thread(TimingThread);
    t.IsBackground = true;
    t.Start();
}

private void TimingThread()
{
    while (true)
    {
        Thread.Sleep(1000);

        OnTick(new EventArgs()); // <- vyvolej událost
    }
}
\end{lstlisting}
\end{yesblock}
\vfill
\end{frame}




\begin{frame}[fragile]
\vfill
\begin{bitemize}{5. použití události}
\item vně třídy nastavíme handler a událost zachytíme
\end{bitemize}
\vfill
\begin{yesblock}
\begin{lstlisting}
static void Main(string[] args)
{
	// vytvoř časovač a spusť ho
    Timer t = new Timer();
    // nastav obslužnou metodu události
    t.Tick += TimerTickHandler;

    // Tick! Tick! Tick!

    Console.ReadKey();
}

private static void TimerTickHandler(object sender, EventArgs eventArgs) => Console.WriteLine("Tick!");
\end{lstlisting}
\end{yesblock}
\vfill
\end{frame}





\begin{frame}[fragile]
\vfill
\begin{bitemize}{Událost jako vlastnost}
\item událost může být definována jako vlastnost
\item vlastnost pak nabízí akce \lstinline|add| a \lstinline|remove| pro zpracování operátorů +=, -=
\end{bitemize}
\vfill
\begin{yesblock}
\begin{lstlisting}[morekeywords={add,remove,value}]
private TickEventHandler tick;
public event TickEventHandler Tick
{
    add { tick += value; }
    remove { tick -= value; }
}
\end{lstlisting}
\end{yesblock}
\vfill
\end{frame}
