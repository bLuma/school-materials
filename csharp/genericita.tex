\hkapitola{Genericita}

\begin{frame}[fragile]
\begin{bitemize}{Genericita}
\item podobně jako v Javě je v C\# k dispozici genericita pro možnost \textbf{generického programování}
\item syntaxe velmi podobná Javě
\begin{itemize}
\item jinak se definují podmínky kovariance a kontravariance a omezující podmínky
\end{itemize}

\item Java genericitu realizuje technikou \uv{type erasure} (ve zkompilovaném kódu se genericita zahodí), C\# využívá \uv{reification} (\uv{zhmotnění typů} -- lze používat generické typy v reflexi i všude jinde)
\begin{itemize}
\item v důsledku existují třídy/rozhraní s genericitou a bez ní (\lstinline|IEnumerable| není to samé co \lstinline|IEnumerable<T>|)
\item metody lze přetěžovat s různými generickými parametry
\end{itemize}
\item \textbf{parametrem} generického typu může být \textbf{libovolný hodnotový i referenční typ} (tedy i primitivní datové typy)
\item vnořené typy automaticky dědí nařazené generické parametry
\end{bitemize}
\vfill


\end{frame}


\begin{frame}[fragile]
\begin{yesblock}
\begin{lstlisting}
class GenericHolder<T>
{
    public T Value { get; set; }
}
\end{lstlisting}
\end{yesblock}
\vfill
\begin{yesblock}
\begin{lstlisting}
GenericHolder<int> intHolder = new GenericHolder<int>();
intHolder.Value = 123;

Console.WriteLine(intHolder.Value);

// Chyba kompilace:
// intHolder.Value = "abcde";
\end{lstlisting}
\end{yesblock}
\end{frame}



\begin{frame}[fragile]
\vfill
\begin{bitemize}{Použití genericity}
\item generická třída/rozhraní -- za názvem třídy/rozhraní se uvede seznam typových parametrů
\begin{itemize}
\item \lstinline|class GenericClass<T> { }|
\end{itemize}

\item generická metoda -- obdobně za názvem metody
\begin{itemize}
\item \lstinline|public void GenericMethod<T>() { }|
\end{itemize}

\item generický delegát -- obdobně
\item pole a genericita -- jednorozměrná pole automaticky realizují rozhraní \lstinline|IList<T>|, použitelné pro čtení hodnot, modifikace vyvolají výjimku
\end{bitemize}
\vfill
\begin{bitemize}{Vytváření generických objektů/proměnných/\ldots}
\item při práci s generickými typy je nutné uvádět dosazené typy (\lstinline|Typ<KonkrétníTyp, JinýKonkrétníTyp>|)
\begin{itemize}
\item i při vytváření objektu, neexistuje tu obdoba Javovského diamond operátoru
\end{itemize}
\end{bitemize}
\vfill
\end{frame}


\begin{frame}[fragile]
\vfill
\begin{bitemize}{Pojmenování typových parametrů}
\item obecné \lstinline|T| v mnoha případech postačí
\item pokud je více typových parametrů, měly by být pojmenovány podle jejich účelu
\begin{itemize}
\item \lstinline|TÚčel, TJinýÚčel|
\item pokud je typový parametr omezen na určitý typ, mělo by jeho označení vycházet z omezujícího typu
\end{itemize}
\end{bitemize}
\vfill
\begin{yesblock}
\begin{lstlisting}
public interface ISessionChannel<TSession> { /*...*/ }
public delegate TOutput Converter<TInput, TOutput>(TInput from);
public class List<T> { /*...*/ }
\end{lstlisting}
\end{yesblock}
\vfill
\end{frame}



\begin{frame}[fragile]
\vfill
\begin{bitemize}{Omezení typových parametrů}
\item C\# umožňuje definovat omezení, které musí platit pro typový parametr
\item \lstinline[morekeywords={where}]|where T : omezeníTypovéhoParametru|
\end{bitemize}
\vfill
\begin{bitemize}{}
\item \lstinline|struct| -- musí být hodnotového typu (nezahrnuje nullable typy)
\item \lstinline|class| -- musí být referenčního typu (třída, rozhraní, delegát, pole)
\item \lstinline|new()| -- musí mít definovaný veřejný bezparametrický konstruktor
\item \lstinline|názevTřídy| -- musí být potomkem dané třídy
\item \lstinline|názevRozhraní| -- musí realizovat dané rozhraní
\item \lstinline|U| -- musí být stejného typu nebo potomkem jako obecný typ \lstinline|U|
\end{bitemize}
\vfill
\end{frame}


\begin{frame}[fragile]
\begin{yesblock}
\begin{lstlisting}[morekeywords={where}]
class OrderedList<T> where T : IComparable<T>
{
    // ...

    public void Add(T item)
    {
        if (item.CompareTo(...) > 0)
            // ...
    }
}
\end{lstlisting}
\end{yesblock}
\end{frame}


\begin{frame}[fragile]
\begin{bitemize}{}
\item je možné definovat více omezení na jeden typový parametr
\item je možné definovat omezení pro více typových parametrů
\end{bitemize}
\vfill
\begin{yesblock}
\begin{lstlisting}[morekeywords={where}]
class EmployeeList<T> where T : Employee, IEmployee, System.IComparable<T>, new()
{
	// ...
}
\end{lstlisting}
\end{yesblock}
\vfill
\begin{yesblock}
\begin{lstlisting}[morekeywords={where}]
class Base { }
class Test<T, U>
    where U : struct
    where T : Base, new() { }
\end{lstlisting}
\end{yesblock}
\end{frame}



\begin{frame}[fragile]
\begin{bitemize}{Inicializace proměnné typového parametru}
\item v obecném případě se může jednat o hodnotový i referenční typ
\begin{itemize}
\item hodnota \lstinline|null| je platná jen pro refereční typy
\item získat výchozí hodnotu pro daný typ je možné pomocí \lstinline|default(Typ)|
\end{itemize}
\end{bitemize}
\vfill
\begin{yesblock}
\begin{lstlisting}
class PtrCounter<T>
{
    public T Ptr { get; set; }
    public int References { get; set; }

    public PtrCounter()
    {
        Ptr = default(T);
        References = 0;
    }

    // ...
}
\end{lstlisting}
\end{yesblock}
\end{frame}
