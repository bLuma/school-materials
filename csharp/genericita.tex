\hkapitola{Genericita}

\begin{frame}[fragile]
\begin{bitemize}{Genericita}
\item podobně jako v Javě je v C\# k dispozici genericita pro možnost generického programování
\item syntaxe velmi podobná Javě
\begin{itemize}
\item jinak se definují podmínky kovariance a kontravariance a omezující podmínky
\end{itemize}

\item Java genericitu realizuje technikou \uv{type erasure} (ve zkompilovaném kódu se genericita zahodí), C\# využívá \uv{reification} (\uv{zhmotnění typů} -- lze používat generické typy v reflexi i všude jinde)
\begin{itemize}
\item v důsledku existují třídy/rozhraní s genericitou a bez ní (\lstinline|IEnumerable| není to samé co \lstinline|IEnumerable<T>|)
\item metody lze přetěžovat s různými generickými parametry
\end{itemize}
\item parametrem generického typu může být libovolný hodnotový i referenční typ (tedy i primitivní datové typy)
\item vnořené typy automaticky dědí nařazené generické parametry
\end{bitemize}
\vfill


\end{frame}


\begin{frame}[fragile]
\begin{yesblock}
\begin{lstlisting}
class GenericHolder<T>
{
    public T Value { get; set; }
}
\end{lstlisting}
\end{yesblock}
\vfill
\begin{yesblock}
\begin{lstlisting}
List<int> list1 = new List<int>();

// No boxing, no casting:
list1.Add(3);

// Compile-time error:
// list1.Add("It is raining in Redmond.");
\end{lstlisting}
\end{yesblock}
\end{frame}