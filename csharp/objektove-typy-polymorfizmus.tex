
\hkapitola{Objektové typy -- dědičnost, polymorfizmus}

\kapitola{Dědičnost}

\begin{frame}[fragile]
\begin{bitemize}{Dědičnost, polymorfizmus}
\item jednoduchá dědičnost -- je možné dědit z jednoho předka (Java)
\begin{itemize}
\item syntaxe připomíná C++
\item pokud není specifikován předek, třída automaticky dědí z~\lstinline|System.Object|
\end{itemize}
\item vícenásobná realizace rozhraní (Java)
\item explicitní rozlišování polymorfních metod -- časná/pozdní vazba (C++)
\end{bitemize}
\end{frame}



\begin{frame}[fragile]
\begin{bitemize}{Konstrukce objektu}
\item \lstinline|new System.Windows.Forms.Form()|
\item proces postupného volání konstruktorů
\begin{itemize}
\item \lstinline|System.Object|
\item \lstinline|System.MarshalByRefObject|
\item \lstinline|System.ComponentModel.Component|
\item \lstinline|System.Windows.Forms.Control|
\item \lstinline|System.Windows.Forms.ScrollableControl|
\item \lstinline|System.Windows.Forms.ContainerControl|
\item \lstinline|System.Windows.Forms.Form|
\end{itemize}
\item poté je objekt zkonstruován a lze jej používat
\end{bitemize}
\end{frame}



\begin{frame}[fragile]
\frametitle{Dědičnost}
\begin{noteblock}{}
\begin{lstlisting}[basicstyle=\small]
[viditelnost] [modifikátory] class NazevTridy [dědičnostARozhraní] { 
	[složkyTřídy]...
}

dědičnostARozhraní:
	: názevPředkaNeboRozhraní, ...
\end{lstlisting}
\end{noteblock}
\vfill
\begin{yesblock}
\begin{lstlisting}[basicstyle=\small]
class Person
{
    public string Name { get; set; }
}

class Student : Person
{
    public string StudentID { get; set; }
}
\end{lstlisting}
\end{yesblock}
\end{frame}



\begin{frame}[fragile]
\frametitle{Dědičnost -- příklady}
\begin{noteblock}{}
\begin{lstlisting}[basicstyle=\small]
class Object { }
class LivingEntity { }
class Person : Object { }

interface IComparable { }
interface ICloneable { }
\end{lstlisting}
\end{noteblock}
\vfill
\begin{yesblock}
\begin{lstlisting}[basicstyle=\small]
class Student : Person { }
class Student : Person, IComparable, ICloneable { }
class Student : IComparable { }
class Student : ICloneable, IComparable { }

struct Student : IComparable { }
\end{lstlisting}
\end{yesblock}
\vfill
\begin{noblock}
\begin{lstlisting}[basicstyle=\small]
class Student : Person, Object { }
class Student : Person, LivingEntity { }

struct Student : Person { }
\end{lstlisting}
\end{noblock}
\end{frame}



\begin{frame}[fragile]
\vfill
\begin{bitemize}{}
\item automaticky se volá konstruktor předka
\begin{itemize}
\item kompilátor umí volat pouze bezparametrický konstruktor
\end{itemize}
\end{bitemize}
\vfill
\begin{yesblock}
\begin{lstlisting}[basicstyle=\small]
class Person
{
  public Person() => Console.WriteLine("Person()");
  public Person(string name) => Console.WriteLine("Person(string)");
}

class Student : Person
{
  public Student() => Console.WriteLine("Student()");
  public Student(string netid) => Console.WriteLine("Student(string)")
}
\end{lstlisting}
\end{yesblock}
\vfill
\end{frame}


\begin{frame}[fragile]
\begin{yesblock}
\begin{lstlisting}[basicstyle=\small]
Student student = new Student();
// Person()
// Student()

Student studentWithParam = new Student("netid");
// Person()
// Student(string netid)
\end{lstlisting}
\end{yesblock}
\end{frame}


\begin{frame}[fragile]
\vfill
\begin{bitemize}{}
\item pokud kompilátor nemá k dispozici konstruktor k zavolání a není možné korektně inicializovat objekt, dojde k chybě kompilace
\end{bitemize}
\vfill
\begin{noblock}
\begin{lstlisting}[basicstyle=\small]
class Person
{
  //public Person() => Console.WriteLine("Person()");
  public Person(string name) => Console.WriteLine("Person(string)");
}

// nelze zkompilovat - kompilátor neumí vytvořit objekt
class Student : Person
{
  public Student() => Console.WriteLine("Student()");
  public Student(string netid) => Console.WriteLine("Student(string)")
}
\end{lstlisting}
\end{noblock}
\vfill
\end{frame}



\begin{frame}[fragile]
\begin{bitemize}{}
\item konstruktor předka lze zavolat pomocí \lstinline|: base([parametry])|
\end{bitemize}
\vfill
\begin{yesblock}
\begin{lstlisting}[basicstyle=\small]
class Person
{
    //public Person() => Console.WriteLine("Person()");
    public Person(string name) => Console.WriteLine("Person(string name)");
}

class Student : Person
{
    public Student() : base("unknown") 
    {
        Console.WriteLine("Student()");
    }

    public Student(string name, string netid) : base(name) 
        => Console.WriteLine("Student(string, string)");
}
\end{lstlisting}
\end{yesblock}
\end{frame}




\begin{frame}[fragile]
\begin{bitemize}{}
\item lze zavolat i jiný konstruktor z aktuální třídy \lstinline|: this([parametry])|
\end{bitemize}
\vfill
\begin{yesblock}
\begin{lstlisting}[basicstyle=\small]
class Student : Person
{
    public Student() : this("unknown", "unidentified") 
    { 
    }

    public Student(string name, string netid) : base(name) 
        => Console.WriteLine("Student(string, string)");
}
\end{lstlisting}
\end{yesblock}
\end{frame}



\kapitola{Polymorfizmus}



\begin{frame}[fragile]
\frametitle{Polymorfizmus}

\begin{bitemize}{Základní vlastnosti}
\item potomek může zastoupit předka
\begin{itemize}
\item všude kde je očekáván předek je možné dosadit potomka
\end{itemize}

\item potomek může metody předka \textbf{zakrýt} nebo \textbf{přepsat}
\begin{itemize}
\item v Javě je automaticky uplatňováno přepsání (tzv. pozdní vazba)
\item v C++ je časná/pozdní vazba rozlišena modifikátorem u metody
\item v C\# je uplatněn systém podobný C++
\end{itemize}

\end{bitemize}
\end{frame}



\begin{frame}[fragile]
\begin{noblock}
\begin{lstlisting}[basicstyle=\small]
class Person
{
  public void DoWork() => Console.WriteLine("Person goes to job!");
}

class Student : Person
{
  // warning CS0108 - úmyslné zakrývání má být označeno new
  public void DoWork() => Console.WriteLine("Student goes to school!");
}
\end{lstlisting}
\end{noblock}
\vfill
\begin{yesblock}
\begin{lstlisting}[basicstyle=\small]
Person person = new Person();
person.DoWork(); // -> person goes to job

Student student = new Student();
student.DoWork(); // -> student goes to school

person = student;
@@person.DoWork();@@ // -> person goes to job 
\end{lstlisting}
\end{yesblock}
\end{frame}



\begin{frame}[fragile]
\begin{bitemize}{Zakrývání}
\item zakrývání znamená vytvoření metody se stejným názvem (resp. signaturou) jako v~předkovi
\begin{itemize}
\item tato metoda, ale nemá polymorfní chování
\item metodu je nutné volat přímo nad instancí potomka, jinak se její kód nepoužije
\item kompilátor vyžaduje označování těchto metod klíčovým slovem \lstinline|new|
\end{itemize}
\end{bitemize}
\vfill
\begin{yesblock}
\begin{lstlisting}[basicstyle=\small]
class Person
{
  public void DoWork() => Console.WriteLine("Person goes to job!");
}

class Student : Person
{
  public new void DoWork() => Console.WriteLine("Student goes to school!");
}
\end{lstlisting}
\end{yesblock}
\end{frame}



\begin{frame}[fragile]
\begin{bitemize}{Polymorfizmus}
\item polymorfizmus umožňuje přetížit metodu v potomkovi
\begin{itemize}
\item taková metoda je použita, pokud pracujeme s typem potomka i předka
\item uplatňuje se tzv. pozdní vazba
\item v předkovi je nutné metodu označit \lstinline|virtual|
\end{itemize}
\end{bitemize}
\vfill
\begin{noblock}
\begin{lstlisting}[basicstyle=\small]
class Person
{
  public virtual void DoWork() => Console.WriteLine("Person");
}

class Student : Person
{
  // warning CS0114 - zakrývání virtual metody
  public void DoWork() => Console.WriteLine("Student");
}

// uvedený kód bude mít stejné chování jako předchozí dva slidy
// uplatněno je zakrývání / časná vazba!
\end{lstlisting}
\end{noblock}
\end{frame}



\begin{frame}[fragile]
\begin{bitemize}{Polymorfizmus}
\item polymorfizmus umožňuje přetížit metodu v potomkovi
\begin{itemize}
\item v předkovi je nutné metodu označit \lstinline|virtual|
\item v potomkovi je nutné metodu označit \lstinline|override|
\end{itemize}
\end{bitemize}
\vfill
\begin{yesblock}
\begin{lstlisting}[basicstyle=\small]
class Person
{
  public virtual void DoWork() => Console.WriteLine("Person");
}

class Student : Person
{
  public override void DoWork() => Console.WriteLine("Student");
}
\end{lstlisting}
\end{yesblock}
\vfill
\begin{yesblock}
\begin{lstlisting}[basicstyle=\small]
Person person = new Student();
person.DoWork(); // -> Student
\end{lstlisting}
\end{yesblock}
\end{frame}


\begin{frame}[fragile]
\begin{bitemize}{Rozpoznání typu objektu}
\item operátory \lstinline|is|, \lstinline|as|, \lstinline|(přetypování)|
\item metoda \lstinline|object.GetType()| a operátor \lstinline|typeof|
\item pattern matching (\lstinline|switch|)
\end{bitemize}
\vfill
\begin{bonusblock}{Polymorfizmus}
\begin{itemize}
\item C\# umožňuje hierarchii virtuálních metod přerušit a začít novou, je tak možné vytvořit metody
\begin{itemize}
\item prarodič -- \lstinline|virtual|
\item rodič -- \lstinline|override|
\item potomek -- \lstinline|new virtual|
\item \ldots
\end{itemize}

\end{itemize}
\end{bonusblock}
\end{frame}





\begin{frame}[fragile]
\begin{bitemize}{Abstraktní metody a třídy}
\item abstraktní metoda (\lstinline|abstract|) = virtuální metoda + nemá definici (tělo)
\begin{itemize}
\item musí být definována v abstraktní třídě
\end{itemize}
\item z abstraktní třídy nejde vytvořit objekt
\begin{itemize}
\item je nutné v potomkovi doplnit definici metody a vytvářet objekt potomka
\end{itemize}
\end{bitemize}
\vfill
\begin{yesblock}
\begin{lstlisting}[basicstyle=\small]
abstract class Person
{
    public abstract void DoWork();
}
\end{lstlisting}
\end{yesblock}
\vfill
\begin{noblock}
\begin{lstlisting}[basicstyle=\small]
Person person = new Person();
\end{lstlisting}
\end{noblock}
\end{frame}



\pkapitola{Rozhraní -- interface}



\begin{frame}[fragile]
\begin{bitemize}{Rozhraní}
\item značí předpis, který musí třída splnit
\begin{itemize}
\item třída může realizovat i více rozhraní
\item třída může rozhraní realizovat implicitně nebo explicitně
\end{itemize}
\item neobsahuje datové složky (atributy)
\item rozhraní nemůže definovat viditelnost složek
\item může obsahovat metody, vlastnosti, indexery, události
\end{bitemize}
\vfill
\begin{yesblock}
\begin{lstlisting}[basicstyle=\small]
interface IExample
{
    // Property
    int Property { get; set; }
    // Method
    void Method(int paramAlfa, int paramBeta);
    // Indexer
    int this[int param] { get; set; }
    // Event
    event ExampleEventHandler Event;
}
\end{lstlisting}
\end{yesblock}
\end{frame}




\begin{frame}[fragile]
\vfill
\begin{block}{}
Implicitní realizace rozhraní
\end{block}
\vfill
\begin{yesblock}
\begin{lstlisting}[basicstyle=\small]
interface IStringizable
{
    string Stringize();
}

class Student : IStringizable
{
    public string Name { get; set; }

    public string Stringize()
    {
        return Name;
    }
}
\end{lstlisting}
\end{yesblock}
\vfill
\end{frame}


\begin{frame}[fragile]
\vfill
\begin{bitemize}{Implicitní realizace rozhraní}
\item obdobné chování jako v Javě
\item metody rozhraní lze volat přímo nad objektem realizující třídy nebo po přetypování
\end{bitemize}
\vfill
\begin{yesblock}
\begin{lstlisting}[basicstyle=\small]
Student student = new Student();
var a = student.Stringize();

IStringizable istringizable = student;
var b = istringizable.Stringize();
\end{lstlisting}
\end{yesblock}
\vfill
\end{frame}







\begin{frame}[fragile]
\vfill
\begin{block}{}
Explicitní realizace rozhraní
\end{block}
\vfill
\begin{yesblock}
\begin{lstlisting}[basicstyle=\small]
interface IStringizable
{
    string Stringize();
}

class Student : IStringizable
{
    public string Name { get; set; }

    string IStringizable.Stringize() // <- uvedení názvu rozhraní před název metody
    {
        return Name;
    }
}
\end{lstlisting}
\end{yesblock}
\vfill
\end{frame}


\begin{frame}[fragile]
\vfill
\begin{bitemize}{Explicitní realizace rozhraní}
\item nemá obdobu v Javě nebo C++
\item metody rozhraní lze volat pouze nad typem rozhraní
\item tento způsob realizace lze využít k rozlišení rozhraní, které mají složky se stejným identifikátorem (signaturou)
\end{bitemize}
\vfill
\begin{yesblock}
\begin{lstlisting}[basicstyle=\small]
Student student = new Student();
// error CS1061
// var a = student.Stringize();

IStringizable istringizable = student;
var b = istringizable.Stringize();
\end{lstlisting}
\end{yesblock}
\vfill
\end{frame}







\begin{frame}[fragile]
\begin{yesblock}
\begin{lstlisting}[basicstyle=\small]
interface IDrawable
{
    void Draw();
}

interface ISurface
{
    void Draw();
}

class Rectangle : IDrawable, ISurface
{
    public void Draw()
    {
        // ...
    }
}
\end{lstlisting}
\end{yesblock}
\vfill
\begin{bitemize}{}
\item implicitní realizace
\item obě rozhraní volají stejnou metodu \lstinline|Draw()|
\end{bitemize}
\end{frame}


\begin{frame}[fragile]
\begin{yesblock}
\begin{lstlisting}
interface IDrawable { void Draw(); }
interface ISurface { void Draw(); }

class Rectangle : IDrawable, ISurface
{
    public void Draw()
    {
        // ISurface
    }

    void IDrawable.Draw()
    {
        // IDrawable
    }
}
\end{lstlisting}
\end{yesblock}
\vfill
\begin{bitemize}{}
\item kombinace implicitní a explicitní realizace
\end{bitemize}
\end{frame}




\begin{frame}[fragile]
\begin{yesblock}
\begin{lstlisting}
interface IDrawable { void Draw(); }
interface ISurface { void Draw(); }

class Rectangle : IDrawable, ISurface
{
    void ISurface.Draw()
    {
        // ISurface
    }

    void IDrawable.Draw()
    {
        // IDrawable
    }
}
\end{lstlisting}
\end{yesblock}
\vfill
\begin{bitemize}{}
\item obě rozhraní používají explicitní realizaci
\end{bitemize}
\end{frame}








\pkapitola{Základní rozhraní v knihovně}


\begin{frame}[fragile]
\vfill
\begin{bitemize}{}
\item \lstinline|ICloneable| -- \lstinline|object Clone()|
\begin{itemize}
\item umožňuje provádět mělkou (\lstinline|MemberwiseClone()|) nebo hlubokou kopii objektu
\end{itemize}

\item \lstinline|IComparable| -- \lstinline|int CompareTo(object obj)|
\begin{itemize}
\item porovnání menší/větší/rovno mezi dvěma objekty
\end{itemize}

\item \lstinline|IComparer| -- \lstinline|int Compare(object x, object y)|
\begin{itemize}
\item podobně jako \lstinline|IComparable|, ale objekt funguje jako nezávislý porovnávač jiných objektů
\end{itemize}

\item \lstinline|IDisposable| -- řízené uvolnění prostředků
\item \lstinline|IFormattable| -- převod na řetězec v daném formátu
\end{bitemize}
\vfill
\end{frame}




\begin{frame}[fragile]
\frametitle{IEnumerable a foreach}
\vfill
\begin{bitemize}{}
\item slouží k projití prvků kolekce
\item lze využít cyklus \lstinline|foreach|
\item metoda \lstinline|IEnumerator GetEnumerator()|
\begin{itemize}
\item \lstinline|IEnumerator|
\begin{itemize}
\item \lstinline|Object Current|
\item \lstinline|bool MoveNext()|
\item \lstinline|void Reset()|
\end{itemize}
\end{itemize}
\end{bitemize}
\vfill
\begin{yesblock}
\begin{lstlisting}
var enumerable = new TestEnumerable();
foreach (var obj in enumerable)
{
    Console.WriteLine($"{obj}");
}
\end{lstlisting}
\end{yesblock}
\vfill
\end{frame}




\begin{frame}[fragile]
\frametitle{IEnumerator}
\vfill
\begin{bitemize}{}
\item enumerátor lze napsat ručně nebo lze využít automatické implementace od kompilátoru a klíčového slova \lstinline|yield|
\end{bitemize}
\vfill
\begin{yesblock}
\begin{lstlisting}
class TestEnumerable : IEnumerable
{
    readonly int[] values = new int[] { 1, 2, 3, 4, 5, 6, 7, 8, 9, 10 };

    public IEnumerator GetEnumerator()
    {
        for (int i = 0; i < values.Length; i++)
        {
            if (values[i] > 5)
                yield break;

            yield return values[i];
        }
    }
}
\end{lstlisting}
\end{yesblock}
\vfill
\end{frame}


\begin{frame}[fragile]
\frametitle{IEnumerator}
\vfill
\begin{bitemize}{}
\item \lstinline|yield return ...| -- přidá prvek do výstupu enumerátoru
\item \lstinline|yield break| -- ukončí iterování
\end{bitemize}
\vfill
\begin{yesblock}
\begin{lstlisting}
class TestEnumerable : IEnumerable
{
    ...
    
    public IEnumerable DescendingValues
    {
        get
        {
            for (int i = values.Length - 1; i >= 0; i--)
                yield return values[i];
        }
    }
}
\end{lstlisting}
\end{yesblock}
\vfill
\end{frame}