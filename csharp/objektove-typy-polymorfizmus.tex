
\hkapitola{Objektové typy -- dědičnost, polymorfizmus}

\kapitola{Dědičnost}

\begin{frame}[fragile]
\begin{bitemize}{Dědičnost, polymorfizmus}
\item jednoduchá dědičnost -- je možné dědit z jednoho předka (Java)
\begin{itemize}
\item syntaxe připomíná C++
\item pokud není specifikován předek, třída automaticky dědí z~\lstinline|System.Object|
\end{itemize}
\item vícenásobná realizace rozhraní (Java)
\item explicitní rozlišování polymorfních metod -- časná/pozdní vazba (C++)
\end{bitemize}
\end{frame}



\begin{frame}[fragile]
\begin{bitemize}{Konstrukce objektu}
\item \lstinline|new System.Windows.Forms.Form()|
\item proces postupného volání konstruktorů
\begin{itemize}
\item \lstinline|System.Object|
\item \lstinline|System.MarshalByRefObject|
\item \lstinline|System.ComponentModel.Component|
\item \lstinline|System.Windows.Forms.Control|
\item \lstinline|System.Windows.Forms.ScrollableControl|
\item \lstinline|System.Windows.Forms.ContainerControl|
\item \lstinline|System.Windows.Forms.Form|
\end{itemize}
\item poté je objekt zkonstruován a lze jej používat
\end{bitemize}
\end{frame}



\begin{frame}[fragile]
\frametitle{Dědičnost}
\begin{noteblock}{}
\begin{lstlisting}[basicstyle=\small]
[viditelnost] [modifikátory] class NazevTridy [dědičnostARozhraní] { 
	[složkyTřídy]...
}

dědičnostARozhraní:
	: [názevPředkaNeboRozhraní], ...
\end{lstlisting}
\end{noteblock}
\vfill
\begin{yesblock}
\begin{lstlisting}[basicstyle=\small]
class Person
{
    public string Name { get; set; }
}

class Student : Person
{
    public string StudentID { get; set; }
}
\end{lstlisting}
\end{yesblock}
\end{frame}



\begin{frame}[fragile]
\frametitle{Dědičnost -- příklady}
\begin{yesblock}
\begin{lstlisting}[basicstyle=\small]
class Student : Person, IComparable, ICloneable { }
class Student : IComparable { }
class Student : ICloneable, IComparable { }
\end{lstlisting}
\end{yesblock}
\vfill
\begin{noblock}
\begin{lstlisting}[basicstyle=\small]
class Student : Person, Object { }
class Student : Person, LivingEntity { }
\end{lstlisting}
\end{noblock}
\end{frame}



\begin{frame}[fragile]
\begin{bitemize}{}
\item automaticky se volá konstruktor předka
\begin{itemize}
\item kompilátor umí volat pouze bezparametrický konstruktor
\end{itemize}
\end{bitemize}

\begin{yesblock}
\begin{lstlisting}[basicstyle=\small]
class Person
{
  public Person() => Console.WriteLine("Person()");
  public Person(string name) => Console.WriteLine("Person(string)");
}

class Student : Person
{
  public Student() => Console.WriteLine("Student()");
  public Student(string netid) => Console.WriteLine("Student(string)")
}
\end{lstlisting}
\end{yesblock}
\end{frame}


\begin{frame}[fragile]
\begin{yesblock}
\begin{lstlisting}[basicstyle=\small]
Student student = new Student();
// Person()
// Student()

Student studentWithParam = new Student("netid");
// Person()
// Student(string netid)
\end{lstlisting}
\end{yesblock}
\end{frame}


\begin{frame}[fragile]
\begin{noblock}
\begin{lstlisting}[basicstyle=\small]
class Person
{
  //public Person() => Console.WriteLine("Person()");
  public Person(string name) => Console.WriteLine("Person(string)");
}

// nelze zkompilovat - kompilátor neumí vytvořit objekt
class Student : Person
{
  public Student() => Console.WriteLine("Student()");
  public Student(string netid) => Console.WriteLine("Student(string)")
}
\end{lstlisting}
\end{noblock}
\end{frame}



\begin{frame}[fragile]
\begin{bitemize}{}
\item konstruktor předka lze zavolat pomocí \lstinline|: base([parametry])|
\end{bitemize}

\begin{yesblock}
\begin{lstlisting}[basicstyle=\small]
class Person
{
    //public Person() => Console.WriteLine("Person()");
    public Person(string name) => Console.WriteLine("Person(string name)");
}

class Student : Person
{
    public Student() : base("unknown") 
    {
        Console.WriteLine("Student()");
    }

    public Student(string name, string netid) : base(name) 
        => Console.WriteLine("Student(string, string)");
}
\end{lstlisting}
\end{yesblock}
\end{frame}




\begin{frame}[fragile]
\begin{bitemize}{}
\item lze zavolat i jiný konstruktor z aktuální třídy \lstinline|: this([parametry])|
\end{bitemize}

\begin{yesblock}
\begin{lstlisting}[basicstyle=\small]
class Student : Person
{
    public Student() : this("unknown", "unidentified") 
    { 
    }

    public Student(string name, string netid) : base(name) 
        => Console.WriteLine("Student(string, string)");
}
\end{lstlisting}
\end{yesblock}
\end{frame}