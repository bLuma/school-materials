\hkapitola{Přetěžování operátorů}

\begin{frame}[fragile]
\vfill
\begin{bitemize}{Přetěžování operátorů}
\item podobně jako C++ lze v C\# přetěžovat operátory u svých objektových typů
\item množina přetížitelných operátorů je oproti C++ menší
\item syntax a pravidla pro přetěžování jsou striktnější a snaží se zamezit problémům
\begin{itemize}
\item např. není možné přetížit == bez přetížení !=
\end{itemize}

\end{bitemize}
\vfill
\begin{bitemize}{Přetížitelné operátory}
\item \lstinline|- ! ~ ++ -- true false|
\item \lstinline|- * / % & ^ | | \lstinline| << >> == != > < >= <=|
\item konverzní operátory
\item \lstinline|[ ]| (indexery)
\end{bitemize}
\vfill
\end{frame}


\nezkouskove

\begin{frame}[fragile]
\frametitle{Přetížení oprerátoru}
\vfill
\begin{noteblock}{}
\begin{lstlisting}
public static typ operator symbol ( parametry ) tělo
\end{lstlisting}
\end{noteblock}
\vfill
\begin{bitemize}{}
\item parametry
\begin{itemize}
\item unární operátor -- jeden parametr
\item binární operátor -- dva parametry
\end{itemize}

\item logické dvojice -- vždy je nutné přetížit oba operátory naráz
\begin{itemize}
\item \lstinline|== !=|
\item \lstinline|< >|
\item \lstinline|<= >=|
\item \lstinline|true false|
\end{itemize}

\item inkremetace/dekremetace se přetěžuje jednou metodou, kompilátor ji automaticky použije pro prefixovou i postfixovou variantu
\item operátory \lstinline|true,false| umožňují dát objekt přímo do výrazů očekávající logickou hodnotu (\lstinline|if, while, ...|)

\end{bitemize}
\vfill
\end{frame}


\begin{frame}[fragile]
\frametitle{Přetížení oprerátoru -- konverzní operátory}
\vfill
\begin{noteblock}{}
\begin{lstlisting}
public static implicit operator cílovýTyp ( zdrojovýTyp parametr ) tělo
public static explicit operator cílovýTyp ( zdrojovýTyp parametr ) tělo
\end{lstlisting}
\end{noteblock}
\vfill
\begin{bitemize}{}
\item není možné zároveň definovat pravidla pro implicitní a explicitní konverzi
\item kompilátor tyto operátory volá při výrazech:
\begin{itemize}
\item implicitní
\begin{itemize}
\item \lstinline|cílovýTyp proměnná = objektZrojovéhoTypu|
\end{itemize}

\item explicitní
\begin{itemize}
\item \lstinline|cílovýTyp proměnná = (cílovýTyp)objektZdrojovéhoTypu|
\end{itemize}
\end{itemize}

\end{bitemize}
\vfill
\end{frame}

\zkouskove

\begin{frame}[fragile]
\frametitle{Indexery}
\vfill
\begin{bitemize}{}
\item definují přetížení operátoru \lstinline|[ ]| pro přístup k prvkům/datům
\item definují se jako nestatické vlastnosti
\end{bitemize}
\vfill
\begin{noteblock}{}
\begin{lstlisting}
[modifikátor] návratovýTyp this [ seznamParametrů ] deklaracePřístupovýchMetod
\end{lstlisting}
\end{noteblock}
\vfill
\begin{yesblock}
\begin{lstlisting}
class IndexerExample
{
    readonly int[] values = new int[] { 1, 2, 3, 4, 5, 6, 7, 8, 9, 10 };

    public int this[int index]
    {
        get => values[index];
    }
}
\end{lstlisting}
\end{yesblock}
\vfill
\end{frame}


\begin{frame}[fragile]
\frametitle{Indexery}
\vfill
\begin{bitemize}{}
\item je možné definovat několik různých přetížení indexerů s různými parametry
\item rozhraní mohou definovat indexery
\item indexer může data číst/zapisovat nebo provádět obojí
\end{bitemize}
\vfill
\begin{yesblock}
\begin{lstlisting}
var indexer = new IndexerExample();
int numberThree = indexer[2];
\end{lstlisting}
\end{yesblock}
\vfill
\end{frame}